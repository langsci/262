\documentclass[output=paper]{langsci/langscibook}
\title{Negative conjuncts and negative concord across the board
}  

\author{Manfred Sailer\affiliation{Goethe-Universität Frankfurt a.M.}\lastand Frank Richter\affiliation{Goethe-Universität Frankfurt a.M.}}


\abstract{Negative concord is a prominent one-to-many correspondence
  between form and meaning at the syntax-se\-man\-tics interface, in
  which one semantic function may correlate with several semantic
  exponents. Languages are typically classified as showing negative
  concord or not, yet they all seem to exhibit the same interpretation
  strategy of conjoined negative noun phrases, i.e.\@ cases like
  \emph{no lecture and no seminar}. We will analyze this construction
  within a framework of a constraint-based, underspecified
  syntax-semantics interface (\emph{Lexical Resource Semantics}, LRS,
  \citealt{Richter:Sailer:04}). We will combine an earlier LRS
  analysis of cross-linguistic variation of negative concord with a
  new analysis of coordination. The latter will make it necessary to
  integrate into LRS so-called \emph{equality up-to} constraints,
  which were originally introduced in \citet{Pinkal:99} as a core type
  of constraint for underspecified semantic systems. We show that the
  resulting analysis captures the negative-concord-like behavior of
  conjoined negative noun phrases even in a non-negative concord
  language like Standard German.  }

% \kommentar{
% Keywords:
% negation
% coordination
% semantic underspecification
% Lexical Resource Semantics
% syntax-semantics interface
% negative concord
% collective predicate
% negative polarity item
% plural
% negative indefinite
% bi-propositional coordination
% mono-propositional coordination
% reciprocal pronoun
% equality up-to constraint


% Languages:
% German
% French
% Polish
% English

% }

\begin{document}
%\avmoption{center,topleft}
\maketitle

%project webpage: https://hpsg.hu-berlin.de/Projects/HPSG-handbook/
%limit: 10,000 words = 20 pages?
%deadlines: 
%			outline: 15.3.
%			full paper: 31.7.

\section{Introduction}
\label{Sec-Intro}

The occurrence of multiple potential markers of negation within a single sentence has been a prominent topic within research on the syntax-semantics interface, see \citet{Giannakidou:05} for an overview. 
An important distinction is typically made on the basis of the
interpretation assigned to such constellations: Ιn \emph{negative
  concord} (NC) languages, the sentences receive a single-negation
reading (SN). 
This is illustrated for Polish in \refer{ex-pl}. 
\emph{Non-NC} languages have a double negation reading (DN), as shown for Standard German (StG) in \refer{ex-de}.
The sentences are ambiguous between SN and DN in optional NC languages, such as French, see \refer{ex-fr}.

\ea \label{ex-pl}
\gll \NE{Nikt} \NE{nic} nie powiedział. (Polish)\\
nobody nothing NM said (SN)\\
\glt \mytrans{Nobody said anything.}
\z

\ea \label{ex-de}
\gll \NE{Niemand} hat \NE{nichts} gesagt. (StG)\\
nobody has nothing said (DN)\\
\glt \mytrans{Nobody didn't say anything.}
\z 

\ea \label{ex-fr}
\gll \NE{Personne} n' a \NE{rien} dit. (French)\\
nobody NM has nothing said {(SN, DN)}\\
\z

SN readings in NC-languages are an instance of a many-to-one relation at the syntax-semantics interface: there are several potential markers of negation in syntax, but only one negation in the interpretation. Consequently, this poses a problem for standard views of compositionality -- see \citet{Sailer:16} for an elaboration of this point.
%This is also the reason why NC-languages have been more prominent in the literature on the syntax-semantics interface than non-NC languages.
%\marginpar{Is that so? what is `more prominent'?}
%\marginpar{compared to what? overall, it is not that big a topic, I think} 
%MS: Rausgenommen: wegen Review1
%Only recently is there a growing interest in the way in which speakers of non-NC languages react to such sentences %\marginpar{MS: Review1-Kommentar eingearbeitet}
There are, however, constellations in which non-NC languages show interpretations that are similar to what has been observed for NC-languages, see for example \citet{Puskas:12} and \citet{Larrivee:16}.
In this paper, we are concerned with one of these constellations.

In the present paper, we will investigate the interpretation of a conjunction of negative noun phrases (CNNP), as illustrated for the three languages above in \refer{ex-conj-pl}--\refer{ex-conj-fr}. 
As indicated, we find the same interpretation for all three languages. 
We will show that the interpretation of CNNP is an instance of NC, even in a non-NC language like StG. 


\ea \label{ex-conj-pl}
\gll Alex nie napisał [\NE{żadnego} listu
i \NE{żadnego} e-maila]
(Polish)\\
Alex NM wrote \hphantom{[}no letter and \hphantom{[}no e-mail (SN)\\
\glt \mytrans{Alex didn't write any letter or any e-mail message.}
\z 

\ea \label{ex-conj-de}
\gll Alex hat [\NE{keinen} Brief und \NE{keine} e-Mail] geschrieben. (StG)\\
Alex has \hphantom{[}no letter and \hphantom{[}no {e-mail message} written (SN)\\
\z

\ea \label{ex-conj-fr}
\gll Alex n' a écrit [\NE{aucune} lettre et \NE{aucun} message électronique]. (French)\\
Alex NM has written \hphantom{[}no letter and \hphantom{[}no message electronic (SN)\\
\z 



In Section~\ref{Sec-Data}, we will present the core empirical properties of CNNP in German. We will show that they are problematic for analyses of negation in StG in Section~\ref{Sec-Previous}. We will then
outline our semantic analysis in Section~\ref{Sec-Semantics}. In Section~\ref{Sec-LRS}, the framework of semantic combinatorics of \emph{Lexical Resource Semantics} (LRS) is introduced as a basis for formulating our analysis within this framework in Section~\ref{Sec-Analysis}. 
We will also show how our NC-like analysis of negated conjuncts in StG carries over to languages with very different sentential negation systems such as Polish or French.
In Section \ref{Sec-Anaphor}, we will consider data with an anaphoric relation between the two conjuncts.
We will end with a short conclusion (Section~\ref{Sec-Conclusion}).

\section{Data: Negative conjuncts in Standard German}
\label{Sec-Data}

StG is not an NC language. 
The empirical situation for the interpretation of sentences with two n-words in StG is briefly sketched on the basis of corpus data in \cite[242--245]{Sailer:18}.
This study confirms that the co-occurrence of two n-words in one sentence as in \refer{ex-de} is generally avoided. %\citep[424--425]{Sailer:18}. 
Many speakers do not find such sentences easily interpretable. 
Those who understand them perceive a DN reading, as indicated above.
For examples with CNNP no such problems arise.

For analogous French and English data, \citet[188, footnote~1]{Larrivee:16} quotes a reviewer's comments on CNNP.
Larrivée's reviewer argues that the sentence in \refer{ex-dogcat} has neither a reading in which the second negative NP is interpreted as an indefinite in the scope of negation -- which would correspond to an NC reading, see \refer{ex-dogcat-nc} --  nor does the sentence have a DN reading.
In a DN reading, the two negations would cancel each other out, and the meaning would correspond to  \refer{ex-dogcat-dn}.
The interpretation rather corresponds to that of a conjunction of two clauses with one negative NP each, as in \refer{ex-dogcat-para}.
Larrivée's reviewer indicates that this reading can be derived with a categorial grammar combinatorics as in \citet{Keenan:Faltz:85}.%
\footnote{The basic idea behind the hypothetical paraphrases in \refer{ex-dogcat} are the following logical representations:

\ea
\textnormal{``NC'' reading:} $\lnot \exists x (\phi \land \exists y (\phi' \land \psi))$
\z 

\ea
\textnormal{``DN'' reading:} $\lnot \exists x (\phi \land \lnot \exists y (\phi' \land \psi))$\\
$\equiv \forall x \lnot (\phi \land \lnot \exists y (\phi' \land \psi))$
$\equiv \forall x (\lnot \phi \lor \lnot \lnot \exists y (\phi' \land \psi))$\\
$\equiv \forall x (\phi \supset \lnot \lnot \exists y (\phi' \land \psi))$
$\equiv \forall x (\phi \supset \exists y (\phi' \land \psi))$
\z 

}

\vbox{
\ea \label{ex-dogcat}
I want no dogs and no cats.
\begin{xlist}
\ex $\not=$ I want no dogs and any cats. (``NC'')
\label{ex-dogcat-nc}
\ex $\not=$ I want every dog and some cat(s).  (``DN'')
\label{ex-dogcat-dn}
\ex $=$ I want no dogs and I want no cats.\label{ex-dogcat-para}
\end{xlist}
\z 

}

The StG sentence in \refer{ex-conj-de} has the same kind of reading, shown in \refer{ex-conj-de-para}. 
Below the paraphrase, we provide a formal rendering.%
\footnote{Throughout this paper, we will state the semantic representation of generalized quantifiers in the form ``quantifier variable (restrictor : scope)''.}

\ea \label{ex-conj-de-para}
\gll Alex hat \NE{keinen} Brief geschrieben und Alex hat \NE{keine} e-Mail geschrieben.\\
Alex has no letter written and Alex has no {e-mail mess.} written\\
\glt \mytrans{Alex didn't write a letter and Alex didn't write an e-mail message.}\\
$\lnot \exists x (\co{letter}(x) : \co{write}(\co{alex}, x)) \land $
$\lnot \exists y (\co{e-mail-mess}(y) : \co{write}(\co{alex}, y))$
\z

%\marginpar{MS: Hedde würde das gerne mono-/bi-clausal nennen, weil es Syntax ist, wir aber gerade nicht. Das sollten wir explizit thematisieren}
We will call this analysis \emph{bi-propositional} as it contains a conjunction of two sentential formul\ae. 
A bi-propositional semantic analysis does not require a syntactic analysis in terms of two clauses, i.e., sentence \refer{ex-dogcat} need not be analyzed as being syntactically derived from its paraphrase in \refer{ex-dogcat-para}.
In the system presented in \citet{Keenan:Faltz:85}, for instance, the  bi-propositional reading is derived from a conjunction of two noun phrases.
We will pursue a similar syntactic structure below.


There is, however, evidence that such a bi-propositional analysis of CNNP is not always possible. 
In \refer{ex-refl} we see that a reciprocal pronoun may take the entire conjunction as its antecedent.%
\footnote{We will mark reflexive and reciprocal pronouns with a wavy underline.}
No bi-clausal paraphrase can be given for such constructions, which is demonstrated by the oddity of example \refer{ex-refl-bi}.%
\footnote{See for example \citet{Winter:01} for a number of cases in which no bi-propositional analysis is possible. In the semantics literature, it is common to distinguish between \emph{boolean} and \emph{non-boolean} coordination instead of bi- and mono-propositional coordination. We prefer to stick to the latter terminology, though.}

\ea \label{ex-refl}
\gll Ich habe gestern [\NE{keinen} Hund und \NE{keine} Katze] \reci{miteinander} streiten hören.\\
I have yesterday \hphantom{[}no dog and no cat {with each other} quarrel heard\\
\glt \mytrans{Yesterday I heard [no dog and no cat] quarrel with one another.}
\z 

\ea 
[*]{\gll Ich habe gestern \NE{keinen} Hund \reci{miteinander} streiten
  hören und ich habe gestern \NE{keine} Katze \reci{miteinander}
  streiten hören.\\
  I have yesterday no dog {with each other} quarrel heard and I have yesterday no cat {with each other} quarrel heard\\ 
  \glt % \mytrans{* I heard no dog quarrel with each other yesterday and I
  % heard no cat quarrel with each other yesterday.}
   \label{ex-refl-bi}}
\z 

Standard tests confirm that the negation in the conjunction expresses a clausal negation rather than a constituent negation. First, we can add a negative polarity item (NPI) such as \bspT{jemals}{ever}.%
\footnote{NPIs are written in italics in our examples.}
The adjusted version of example \refer{ex-refl} is given in \refer{ex-refl-npi}.

\ea \label{ex-refl-npi}
{\gll Ich habe [\NE{keinen} Hund und \NE{keine} Katze] \npi{jemals}
  \reci{miteinander} streiten hören.\\
  I have  \hphantom{[}no dog and no cat ever {with each other} quarrel heard\\
  \glt \mytrans{I heard [no dog and no cat] ever quarrel with one another.}}
\z 

Second, we can continue sentence \refer{ex-refl} with the German equivalent of \bsp{and neither does X}, see \refer{ex-refl-neither}.%
\footnote{This negativity test is also applied in \citet{Zeijlstra:17}.}

\ea \label{ex-refl-neither}
\gll \refer{ex-refl} und Alex auch nicht.\\
{} and Alex also not\\
\glt \mytrans{\ldots{} and neither did Alex.}
\z 

This shows that the negation in example \refer{ex-refl} takes clausal scope. At the same time, the conjunction as a unit serves as the antecedent for the reciprocal pronoun. Consequently, we need to pursue a mono-propositional analysis of CNNP. 

%HIER SCHON MAL DIE MONO-PROPOSITIONALE ANALYSE ZEIGEN UND AUF SPÄTER VERWEISEN!

However, we cannot discard the option of a bi-propositional analysis entirely. 
%Example \refer{ex-eltern} shows that in some cases we do not easily obtain a reading in which we can get the joint set of the two conjuncts. 
%\marginpar{unklar, was das joint set negativer quantoren ist} (\#\# \ldots reading which involves the sum of the two conjuncts. ?\#\#)
In example \refer{ex-eltern}, all speakers obtain a bi-propositional reading, i.e.\@ a reading in which there is a disagreement among the children and a disagreement among the adults, see \refer{ex-eltern-bi}.
Many speakers do not accept the reading \refer{ex-eltern-mono}, in which the quarrel happens across the two groups.


\ea \label{ex-eltern}
\gll [\NE{Keine} Kinder und \NE{keine} Erwachsenen] haben gestritten.\\
\hphantom{[}no children and no adults have quarreled\\
\glt \mytrans{No children and no adults quarreled.}
\begin{xlist}
\ex [$=$]{The children did not quarrel with one another and the adults did not quarrel with one another.
\label{ex-eltern-bi}}
\ex [$\not=$]{The children did not quarrel with the adults and the other way around. (for many speakers)
\label{ex-eltern-mono}}
\end{xlist}
\z

\begin{sloppypar}
  If we put the conjuncts in singular the sentence is often
  uninterpretable, marked with ``\#''.
\end{sloppypar}
\ea \label{ex-eltern2}
\gll \# [\NE{Kein} Kind und \NE{kein} Erwachsener] haben gestritten.\\
{} \hphantom{[}no child and no adult have quarreled\\
\glt $\not=$ \mytrans{No child and no adult quarreled.} (for many speakers)
\z

The verb \bspT{streiten}{to quarrel} requires a group as its subject when used intransitively. 
Example \refer{ex-eltern2} shows that many speakers consider such a group formation impossible in this constellation. 
In \refer{ex-eltern}, the conjuncts are in plural, so each conjunct provides the required group argument. However, at least some speakers do seem to obtain a reading for \refer{ex-eltern2}, and this reading can be emphasized by adding \bsp{miteinander} to the sentence, which other speakers consider degraded or unacceptable, marked with ``\%''.

\ea \label{ex-eltern3}
\gll \% [\NE{Kein} Kind und \NE{kein} Erwachsener] haben \reci{miteinander} gestritten.\\
{} \hphantom{[}no child and no adult have with.each.other quarreled\\
\glt $=$ \mytrans{No child and no adult quarreled with one another.}
\z

The same judgment pattern emerges for universally quantified conjuncts: for many speakers the plural version in \refer{ex-eltern-all} lacks the reading in which there is a cross-group quarrel, and the singular version in \refer{ex-eltern-jed} is not interpretable for these speakers. Other speakers, who seem to be in the minority, have an additional cross-group reading for \refer{ex-eltern-all}, and do get a reading for \refer{ex-eltern-jed}. The reading they obtain for \refer{ex-eltern-jed} can be emphasized by adding \bsp{miteinander} to the sentence, as in the corresponding \refer{ex-eltern3}. For this reading \bsp{haben} must have plural agreement with the coordinated subject.

\ea \label{ex-eltern-all}
\gll Alle Kinder und alle Erwachsenen haben gestritten.\\
all children and all adults have quarreled\\
\glt \mytrans{All children quarreled among themselves and all adults quarreled among themselves.}\\
\glt $\not=$ \mytrans{All children quarreled with all adults.} (for many speakers)
\z 

\ea \label{ex-eltern-jed}
\gll Jedes Kind und jeder Erwachsene \%haben/ *hat gestritten.\\
every child and every adult \hphantom{\%}have/ \hphantom{*}has quarreled\\
\z 

We do not know the conditions under which a bi-pro\-po\-si\-tional reading seems required (for some speakers) or strongly preferred (for others). 
It seems clear to us, however, that there are two readings, one mono-propositional and one bi-pro\-po\-si\-tional. 
Consequently, we will assume that CNNPs are in principle ambiguous, but that there are factors enforcing a mono-propositional reading (such as reciprocals with singular conjuncts), and also factors enforcing a bi-propositional reading (for many speakers). 
While these factors are not clear to us at present, an adequate theory must certainly provide representations for both. 
Restrictions that explain majority preferences or completely exclude one of the readings under certain circumstances can be added to this general theory as they are being worked out. They might be additional grammatical constraints or processing constraints.%
\footnote{Some of these restrictions will follow from our treatment of distributive and collective predicates.}
%\marginpar{processing effects more plausible for preferences than grammatical constraints?}

%We managed to identify classes of examples in which only a mono-propositional reading is available, and we found examples in which only a bi-propositional reading is available. 




\bigskip%
So-called \emph{split-readings} are an interesting property of German negative NPs, which became prominent in formal semantic discussion through \citet{Jacobs:80}. 
According to a favored analysis, a sentence with a negative indefinite will have a semantic representation involving a negation and an existential quantifier. 
However, \citet{Jacobs:80} showed that the existential quantifier need not be in the immediate scope of the negation. \citet{Penka:Stechow:01} illustrate this with the example in \refer{ex-tie} with an intervening modal operator.

%\marginpar{finde Seitenzahl!}
\ea \label{ex-tie}
\gll Monika \npi{braucht} \NE{keinen} Vortrag zu halten.\\
Monika need no lecture to give\\
\glt \mytrans{Monika need give no lecture.}\\
$=$ \mytrans{It is not the case that it is necessary that Monika gives a lecture.}
%\gll Fritz muss keinen Schlips anziehen.\\
%Fritz must no tie {put on}\\
%\glt \mytrans{It is not necessary that Fritz put on a tie.}
\z 

The verb \bspT{brauchen}{need} in \refer{ex-tie} is an NPI expressing a necessity modality. Consequently, it enforces wide scope of the negation. The semantic representation corresponding to the relevant reading is given in \refer{ex-tie-lf}.

\ea \label{ex-tie-lf}
$\lnot \Box (\exists x (\co{lecture}(x) : \co{present}(\co{monika},x)))$
%$\lnot \co{MUST} (\exists x (\co{tie}(x) : \co{put-on}(\co{fritz},x)))$
\z 

We can modify example \refer{ex-tie} slightly to show that CNNP has the same type of split reading.


\ea \label{ex-tie-conj}
\gll Monika \npi{braucht} [\NE{keinen} Vortrag und \NE{kein} Seminar] zu halten.\\
Monika need \hphantom{[}no lecture and no seminar to give\\
\glt 
\mytrans{It is not the case that Monika is obliged to give a lecture and it is not the case that Monika is obliged to give a seminar.}
%\mytrans{It is not the case that Monika is obliged to give a lecture or a seminar.}
\z 

%MS alt: To ensure that we are not dealing with a bi-propositional analysis, we construct a sentence with \bsp{brauchen} and a reciprocal.
We used a bi-propositional paraphrase in \refer{ex-tie-conj}. 
To show that split readings are also available with mono-propositional readings, we construct a sentence with \bsp{brauchen} and a reciprocal.

\ea
\gll Du \npi{brauchst} [\NE{keinen} Vortrag und \NE{kein} Seminar] \reci{miteinander} zu vergleichen.\\
you need \hphantom{[}no lecture and no seminar {with each other} to compare\\
\glt \mytrans{You don't need to compare any lecture with any seminar.}\\
$=$ \mytrans{It is not the case that you are obliged to compare a lecture and a seminar with each other.}
\z 

To sum up the discussion so far, a CNNP can serve as antecedent to a reciprocal pronoun, it expresses a clausal negation, and this negation can have wide scope over the existential quantifier
(originating from \bspT{kein-}{no}) and the intervening material.

\bigskip%
Before closing the data discussion, we would like to point to another intriguing property of CNNP. 
For many cases of CNNP a natural paraphrase would contain a negation plus a \emph{disjunction} of indefinite noun phrases rather than a conjunction. Such a disjunctive paraphrase can be given for example \refer{ex-tie-conj} above, see \refer{ex-tie-conj-paraOr}.

\ea \label{ex-tie-conj-paraOr}
It is not the case that Monika is obliged to give a lecture or a seminar.\\
\hspace*{\fill} $=$ \refer{ex-tie-conj} 
\z 

In fact, using a conjunction in a mono-clausal paraphrase would not yield the correct interpretation. Such a hypothetical paraphrase of \refer{ex-tie-conj} is given 
in \refer{ex-tie-para-dis}. 

\ea \label{ex-tie-para-dis}
%\refer{ex-tie-conj} $\not=$ 
It is not the case that Monika must give a lecture and a seminar.
\hfill $\not=$ \refer{ex-tie-conj} 
\z 

Sentence \refer{ex-tie-para-dis} expresses the idea that Monika is not obliged to give both a lecture and a seminar. 
Sentence \refer{ex-tie-conj}, however, expresses the idea that Monika is not obliged to do either of the two. 
We call this property the ``disjunction'' effect of CNNP. 

It is important that the predicate used in \refer{ex-tie-para-dis} is distributive, i.e.\@ we cannot distinguish between a bi-propositional and a mono-propositional analysis on the basis of the truth conditions. If we insert a reciprocal, as in \refer{ex-tie-para-reci}, there is no bi-propositional reading and, consequently, there is 
no equivalence between a bi- and a mono-propositional analysis.

\ea \label{ex-tie-para-reci}
It is not the case that Monika must compare a lecture and a seminar with each other.
\z

%The most straightforward interpretation of these observations is that the conjunction of noun phrases can have a mono-propositional or a bi-propositional interpretation in principle. A reciprocal pronoun is only compatible with the mono-propositional analysis. 

Related to the disjunction effect is another observation:
CNNP is missing a reading that is available for a negated sentence with conjoined indefinite noun phrases, namely the ``not-both'' reading.

% \kommentar{
% The ``flip'' of a conjunction into a disjunction in the scope of negation immediately suggests the application of de Morgan's law, i.e.\@ the equivalence of $\lnot A \land \lnot B$ and $\lnot (A \lor B)$. 

% Unfortunately, de Morgan's law cannot be applied here since we are dealing with the conjunction of noun phrases rather than clauses. 
% This can be shown by a simple contrast: a version of \refer{ex-tie-conj} with a verbal negation and simple indefinites has the reading as paraphrased in \refer{ex-tie-para-dis}, i.e., it does not show the disjunction effect.
% }

We can use a neg-raising constellation \citep{Horn:78} to show a contrast between CNNP and negated occurrences of conjoined indefinite noun phrases. 
In such a constellation the negation is in the higher clause but its effect is visible in the embedded clause -- as shown by the licensing of the NPI \bspT{brauchen}{need}. 
The example in \refer{ex-tie-glaub-both} is compatible with two readings: one reading in which the speaker thinks that Monika needs to teach neither a lecture nor a seminar, and a second reading in which the speaker thinks that she is not obliged to teach both types of classes, but maybe one of them.%
\footnote{We find a disambiguating effect of stress in \refer{ex-tie-glaub-both}, as observed for English in \cite[226]{Szabolcsi:Haddican:04}: Reading 2 requires stress on \bspT{und}{and}, whereas Reading 1 allows for no stress on the conjunction particle.}

\ea \label{ex-tie-glaub-both}
\gll Ich glaube \NE{nicht}, dass Monika [einen Vortrag und ein Seminar] zu halten \npi{braucht}.\\
I think not that Monika \hphantom{[}a lecture and a seminar to teach need\\
%\glt \mytrans{I don't think that Monika need teach both, a lecture and a seminar.}\\
\glt Reading 1: \mytrans{I think that Monika is not obliged to teach either a lecture or a seminar.}
\glt Reading 2: \mytrans{I think that Monika is not obliged to do both: teach a lecture AND a seminar.}
\z 

In contrast to the data with negated indefinites in a neg-raising constellation, CNNP only allows for the first reading, i.e.\@ the reading that Monika needs to teach neither type of class. 
This is shown in \refer{ex-tie-glaub-neither}.

\ea \label{ex-tie-glaub-neither}
\gll 
Ich glaube, dass Monika [\NE{keinen} Vortrag und \NE{kein} Seminar] zu halten \npi{braucht}.\\
I think that Monika no lecture and no seminar to teach need\\
\glt 
Reading 1: \mytrans{I think Monika is not obliged to teach either a lecture or a seminar.}\\
Reading 2: \# \mytrans{I think that Monika is not obliged to do both: give a lecture AND give a seminar.}
\z 

If we enforce a mono-propositional reading, the two constellations are paraphrases, i.e., the sentences in \refer{ex-glaub-notEX-reci} and \refer{ex-glaub-CNNP-reci} have the same truth conditions:  the speaker thinks that there is no pair consisting of a lecture and a seminar such that the two need to be compared. This corresponds to the English sentence in \refer{ex-glaub-X-reci-comp}.

\ea \label{ex-glaub-notEX-reci}
\gll Ich glaube \NE{nicht}, dass Monika einen Vortrag und ein Seminar (\reci{miteinander}) vergleichen muss.\\
I think not that Monika a lecture and a seminar {with each other} compare must\\
\z 

\ea \label{ex-glaub-CNNP-reci}
\gll Ich glaube, dass Monika \NE{keinen} Vortrag und \NE{kein} Seminar (\reci{miteinander}) vergleichen muss.\\
I think that Monika no lecture and no seminar {with each other} compare must\\
\z 

\ea \label{ex-glaub-X-reci-comp}
I believe that Monika need not compare a(ny) lecture and a(ny) seminar.
\z 

%Our English translation indicates that  the sentence has the ``not-both'' reading, not the ``neither-nor'' reading, which is given in \refer{ex-tie-glaub-neither}



%\ea \label{ex-tie-glaub-neitherX}
%Ich glaube nicht, dass Monika [einen Vortrag und ein Seminar] zu halten \npi{braucht}\\
%\mytrans{I don't think that Monika need give either a lecture or a seminar}\\
%\mytrans{I think that Monika is not obliged to give either a lecture or a seminar.}
%\z 

To summarize these observations, the disjunction reading seems to be obligatory with CNNP independently of whether we are forced to have a mono-pro\-po\-si\-tional analysis or not. 
For non-negative indefinites in the scope of negation, the disjunction reading is not obligatory. %and might even be dispreferred.
The difference in readings between \refer{ex-tie-glaub-both} and \refer{ex-glaub-notEX-reci}
can be taken as additional support for our decision to
assume that both a mono-propositional and a bi-propositional reading should be derivable for conjoined noun phrases.

\bigskip%
This leaves us with a number of challenging properties of CNNP: 
(i) we cannot analyze it as a bi-propositional construction in all cases, 
(ii) we must permit split readings of the negation component and the existential component of the determiner, 
and (iii) we have to account for the disjunction effect. In addition, since CNNP uses no construction-specific lexical items nor a special syntactic form, no special apparatus should be required in its analysis.


%\subsection{Negation in Standard German}
%\label{Sec-NegStG}
%\subsection{Negated conjuncts}

\section{Related Analyses}
\label{Sec-Previous}

% \kommentar{
% HIER FEHLT NOCH GANZ VIEL. WENN FRANK AN DEN ANDEREN TEILEN SCHREIBT, KANN ICH DA NOCH WAS MACHEN, ALLERDINGS BRAUCHT DAS ZEIT, DIE WIR NICHT HABEN...

% The most prominent recent approaches to negative noun phrases in StG are formulated within the framework of \emph{Transparent Logical Form} (TLF), presented in \citet{Stechow:93}  or \citet{Heim:Kratzer:98}.
% Within TLF, there is a level of syntactic representation, called \emph{Logical Form} (LF), that reflects the scopal relations of the operators in a sentence by its c-command relations. 
% Given this assumption, any negative clause must have a syntactic position that is associated with the scope of the negation. 
% Consequently, it follows from the availability of split readings that the position of the negation-node must be higher in the LF tree than the position marking the scope of the indefinite. 

% A prominent recent approach within this research tradition is formulated in \citet{Penka:12}. 
% Penka treats  expressions like \bspT{kein}{no} semantically as indefinites that carry a syntactic requirement 
% to occur in the appropriate constellation with 
% %which will be checked by 
% a negation-expressing node.%
% \footnote{Technically, she assumes an uninterpretable NEG feature on negative indefinites that will be checked by an interpretable NEG feature in a certain syntactic constellation.} 
% In StG, negation
% %the interpretable NEG feature 
% is typically contributed by a phonologically empty element.




% WIR MUESSEN HIER NOCH DIE DINGE VON JOVANA GAIJC AUS GÖTTINGEN ZU \bsp{neither nor} anschauen und reinnehmen.

% %\citet{Penka:12,Penka:15}
% }

To our knowledge, CNNP has not been studied in the formal syntactic and semantic literature. 
For this reason, we will not be able to compare our approach to a concrete existing proposal.
%Related to this, there is no  approach for which we can directly observe the interaction of the analysis of negation and that of coordination.
%
%For these reasons
Consequently, we will limit ourselves here to the following questions: (i) How do existing proposals treat the difference between NC and non-NC languages? 
(ii) How do they derive split readings? 

The introduction of split readings into the discussion of StG negation in \citet{Jacobs:80} encouraged analyses that treat determiner \bspT{kein-}{no} as an indefinite  in the scope of a negation. 
%
The most prominent recent approaches to negative noun phrases in StG are formulated within the framework of \emph{Transparent Logical Form} (TLF), presented in \citet{Stechow:93}  and \citet{Heim:Kratzer:98}.
Within TLF, a level of syntactic representation, called \emph{Logical Form} (LF), displays the scope relations of the operators in a sentence by their c-command relations. 
Given this assumption, any negative clause must have a syntactic position that is associated with the scope of the negation. 
It thus follows from the availability of split readings that the position of the negation-node must be higher in the LF tree than the position marking the scope of the indefinite. 
An overview of the analyses of negation within this research strand is given in \citet{Zeijlstra:16}.

If the indefinite associated with the n-word is treated in exactly the same way semantically as the indefinite article, we would predict that there is no difference in meaning between an overt negation marker with an indefinite and the occurrence of the negative indefinite. We saw above with the examples in \refer{ex-tie-glaub-both} and \refer{ex-tie-glaub-neither} that this is not the case for CNNP. Thus any analysis of this type must still be able to distinguish semantically between a plain indefinite and a negative indefinite.

A further challenge of this type of approach lies in the syntactic constellation that must hold between the abstract negation-node and the node marking the scope of the existential.
%n-word. 
Because of the availability of split readings, this constellation cannot be one of immediate scope. Surface adjacency is a good candidate.%
\footnote{Surface adjacency is mentioned in \citet{Penka:11} as a licensing condition on negative indefinites (NI) in German, where Op$\lnot$ stands for a (phonologically empty) negation that occurs as a terminal node in the structure. The condition in \refer{penka-adjacent} is taken from \cite[112]{Penka:11}.

\ea \label{penka-adjacent}
Licensing condition for NIs in German:\\
NIs have to be adjacent to an abstract negation Op$\lnot$ in the surface syntax.
\z 

}

The adjacency condition is illustrated in \refer{often-homework}. Given the word order in the sentence, the scope of the negation must be below \bspT{öfters}{several times}, i.e., the negation-expressing (covert) node 
%``invisible negation'' 
must be adjacent to the n-word.%
\footnote{When the adverb overtly follows the n-constituent, as in
  \refer{not-often}, only the reading in \refer{nicht-oefters} is
  possible.
  
\ea \label{not-often}
\gll Alex hat für die Sitzungen \NE{kein} Buch öfters gelesen.\\
Alex has for the {class meetings} no book {several times} read\\
\z


}

\ea \label{often-homework} 
%}[Even though Alex should regularly read books before classes]
\gll Alex hat für die Sitzungen öfters \NE{kein} Buch gelesen.\\
Alex has for the {class meetings} {several times} no {book} read\\
\begin{xlist} 
\ex 
$=$ \mytrans{It was several times the case that Alex did not read a book before the class meetings.}\label{oefters-nicht}
\ex 
$\not=$ \mytrans{It is not the case that Alex read a book several times before the class meetings.}\label{nicht-oefters}
\end{xlist}
\z 


% \kommentar{
% \ea \label{often-homework}
% \gll Alex hat häufig \NE{keine} Hausaufgaben gemacht.\\
% Alex has frequently no {homework} done\\
% \glt 
% $=$ \mytrans{It was often the case that Alex didn't make homework.}\\
% $\not=$ \mytrans{It is not the case that Alex often made homework.}
% \z  
% }

% Alex hat kein Buch öfters gelesen.
% Alex hat öfters kein Buch gleesen


The adjacency requirement cannot mean the adjacency of the indefinite word and the negation-expressing node, as the indefinite may be embedded inside a larger noun phrase. 
This is shown in \refer{kennedy}. 
We use an NPI in the sentence to show that there is a negation taking sentential scope.

\ea \label{kennedy}
\gll {}[Der Besuch \NE{keines} amerikanischen Präsidenten] hat \npi{jemals} so viel Begeisterung ausgelöst wie der von Kennedy in Berlin.\\
\hphantom{[}the visit {of no} American president has ever so much enthusiasm caused as {that} by Kennedy in Berlin\\
\glt \mytrans{The visit of no American president has ever caused as much enthusiasm as that of Kennedy in Berlin.}
\z 

We saw in \refer{ex-tie-conj} that conjoined n-constituents can license NPIs. This effect is also observed with n-words deeply embedded in conjuncts. This is shown in \refer{cnnp-kennedy}. 

\ea \label{cnnp-kennedy}
\gll Maria hat sich [[über  Geschenke von \NE{keinem} Verwandten] und [über Glückwünsche von \NE{keinem} Freund]] \npi{jemals} so sehr gefreut wie bei ihrer Hochzeit.\\
Maria has \textsc{refl} \hphantom{[[}about presents from no relative and about wishes from no friend ever so much {been excited} as on her wedding\\
\glt \mytrans{Maria was never as excited [[about any relative's presents] and [about any friend's wishes]] as at her wedding.}
\z 

These data show that 
an analysis in which n-constituents are decomposed syntactically into a negation-expressing node and an existential determiner
needs to be both restrictive and flexible with respect to the semantic and syntactic relation holding between the 
two components.%
%indefinite and the abstract negation node.%
\footnote{Zeijlstra (personal communication) points out that the adjacency requirement is also problematic for English in examples such as \refer{no-tie}. English being an SVO language, the non-finite verb stands between the negation and the direct object.

\ea \label{no-tie}
You need wear no tie.\\
\mytrans{It is not the case that you are obliged to wear a tie.}
\z 

}

\medskip%
We can now turn to two concrete proposals within the TLF tradition.
One line of research within this tradition is the work of Penka and her co-authors \citep{Penka:Stechow:01,Penka:Zeijlstra:11,Penka:11,Penka:12}.
Penka treats expressions like \bspT{kein}{no} semantically as indefinites that carry a syntactic requirement to occur in the right constellation with
%which will be checked by 
a negation-expressing node.% 
\footnote{Technically, she assumes an uninterpretable NEG feature on negative indefinites that must be checked by an interpretable NEG feature in a certain syntactic constellation.} 
In StG, 
%the interpretable NEG feature 
negation is typically contributed by a phonologically empty element.
The fact that n-words carry this special licensing requirement  can be used to distinguish between a negative indefinite and a plain indefinite. 
%While this is not exploited in the cited work, the difference can clearly be incorporated in this approach.

\begin{sloppypar}
  To account for the non-NC character of StG, Penka assumes that each
  n-word needs to satisfy its licensing requirement
  % check its NEG feature
  against a separate negation-node.  CNNP might be problematic for
  this assumption as we have two n-words but only one negation.  There
  would, of course, not be a problem for the bi-propositional readings
  that could be derived from an underlying bi-clausal syntactic
  analysis.  As we have argued, however, we have empirical evidence
  that a mono-propositional analysis is required as well.
\end{sloppypar}
\medskip%
A second approach to n-words is found in the work of Zeijlstra, starting with \citet{Zeijlstra:04}. 
Our presentation will be based on \citet{Zeijlstra:15}, which is a recent and technically precise formulation of his theory.
Zeijlstra assumes that n-words in non-NC languages are lexically specified as being semantically negative. In addition, he proposes syntactic features, uNEG and iNEG, to capture the language- and item-specific distribution of n-words and negative markers. 
He accounts for the split readings of StG by postulating
two features on n-words: one being responsible for negation, one for the existential interpretation. These two features can be checked in different places in the syntactic tree. These places, then, mark the scope of the two components.

It is important for our discussion here that Zeijlstra treats n-words in NC languages as different from n-words in non-NC languages. 
As in the case of Penka's approach, it is not clear how his approach generalizes to CNNP as we do not know his analysis of coordination. 
CNNP might, however, not be straightforward to capture: since n-words contribute a semantic negation in his analysis and each contributed negation needs to be interpreted, Zeijlstra might be forced into a bi-propositional analysis of CNNP and might not be able to describe data that require a mono-propositional semantic representation.

% \kommentar{

% Negation and coordination have been studied in the literature with respect to the negative coordination particles such as English \bsp{neither nor}.

% \citet{deSwart:01ni} looks at French \bsp{ni \ldots ni}{neither \ldots nor}. She argues 

% \citet{Gajic:16} 

% %WIR MUESSEN HIER NOCH DIE DINGE VON JOVANA GAIJC AUS GÖTTINGEN ZU \bsp{neither nor} anschauen und reinnehmen.

% }

\bigskip% 
\begin{sloppypar}
  The challenge of CNNP in the current state of discussion of negation
  and non-NC languages lies in the combination of two properties:
  First, we are forced to assume a mono-propositional analysis -- at
  least for cases in which a bi-propositional analysis is not
  possible. Second, as a consequence thereof, there can only be one
  negation in the interpretation of CNNP, even if StG usually exhibits
  1-to-1 correspondence between n-words and semantic negations.
\end{sloppypar}
%In addition, the mono-propositional analysis appears to be restricted  to cases in which a bi-propositional analysis is not possible. Consequently, we will need both types of analysis. 



\section{The semantics of conjunctions of negative noun phrases}
\label{Sec-Semantics}

In this section, 
%Before we turn to our combinatorial analysis of CNNP, we 
we will discuss the semantic representation that we consider adequate for the CNNP construction. 
In particular, we will emphasize that the proposed representations are motivated by the observations in Section~\ref{Sec-Data}.
We will not be concerned with the question of how these representations can be connected to a syntactic analysis of the CNNP sentences until Section~\ref{Sec-Analysis}.
%
In Section \ref{Sec-Conjunction},
we adopt the analysis of mono-propositional noun phrase conjunction from \citet{Chaves:07}, in which 
 the conjunction introduces a new, plural discourse referent -- which will account for the data on reciprocals as in \refer{ex-refl}.
In Section \ref{Sec-DisjunctionEffect}, we propose that a negation can take wide scope over the conjunction to account for NPI-licensing. 
We will show that we can capture the disjunction effect.  %with this assumption and the way in which we specify a plural discourse referent for the conjoined noun phrases.

\subsection{Conjunction}
\label{Sec-Conjunction}

In this subsection, we will propose an analysis of the semantics of the conjunction of quantified noun phrases.
Negation will not play a role in this subsection.
We assume a division of labor between the mono- and the bi-propositional analyses: 
While the bi-propositional analysis may be considered more basic, the mono-propositional analysis is available whenever there is no possible bi-propositional analysis,
as in 
%that a bi-propositional analysis is possible for conjoined noun phrases, but a mono-propositional analysis is available, too, and required for 
cases with a collective predicate or some other indication of collectivity, such as a reciprocal pronoun. 
Our mono-propositional analysis will be a variant of the analysis developed in \citet{Chaves:07}. 
%for non-negative conjuncts.

The semantic representation of the bi-propositional reading of a conjunction is straightforward and does not require special discussion here. 
%new semantic material. 
For the mono-propositional analysis, however, we need to introduce plural individuals and tuples.
%
Since we cannot present a semantic analysis of plural here, we will keep this discussion as general as possible. 
For our examples, it is enough if we treat plural individuals as sets, in contrast to collective individuals such as \bsp{committee} or \bsp{deck of cards} \citep{Link:83}. 
%When we work with sets as arguments of predicates, it is a natural assumption that a predicate holds trivially when applied to an empty set. 

Whether a predicate is interpreted collectively, distributively, or has both readings in a given sentence is determined lexically or contextually.%
\footnote{See \citet[46]{Winter:01} for a classification of various lexically and constructionally conditioned collective interpretations of verbs, adjectives, and nouns.}
%
An  obligatorily distributive predicate such as \co{sleep} is true of a set if 
and only if every element is in the set of sleepers.
We need tuples to account for collective predicates and the reciprocal readings. 
To give a simple example, the predicate \co{meet} takes a set of pairs as its argument and holds of this set of pairs if and only if every pair in the set is such that the first element of the pair meets the second element.%
\footnote{As discussed in \citet{Sabato:Winter:12}, for reciprocal readings, predicates differ with respect to the exact requirements on which tuples need to be included in their denotation. 
The initial example in \citet{Sabato:Winter:12} is the contrast between \bsp{know each other} and \bsp{be standing on each other}.
See also \citet{Winter:16} for detailed considerations of various types of collective predicates.
}
%, as we find them in mono-propositional readings.


% \kommentar{
% We assume that there is a tuple-operator, $\tuple{i}$, where $i$ indicates the arity of the tuple. 

% \ea For each natural number $i$,\\
% for each set \obt{S}, 
% $\tuple{i}\obt{S} 
% \subseteq \obt{S}^î$,\\
% for each $i$-tuple $\obt{S}^î$, $\tuple{i}\obt{S} = \obt{S}$,
% and undefined else.
% \z 
% }

%The predicate \co{student}, as in \refer{dog-sleep}, is true of a set if  
%Similarly, the predicate \co{sleep} is


The denotation of the predicates \co{sleep} and \co{meet} is given
in \refer{def-sleep-gather}, where $S$, $S_1$, and $S_2$ are sets of individuals.
The denotation of the predicate \co{sleep} is defined in \refer{def-sleep} by distribution over all its elements.
In contrast to this, the predicate \co{meet} in \refer{dog-gather} is obligatorily collective.
To determine whether we can say of a set that its elements met, we need to look at all non-reflexive pairs of this set and determine whether all of these pairs met.
%For a set to meet, it is required that all -- or at least almost all -- of the members of the set come together.%
%\footnote{\citet{Winter:16} presents detailed considerations of various types of collective predicates.}
Consequently, the predicate \co{meet} ranges over sets of pairs. There is however the option of type coercion for \co{meet}: if its argument is a simple set, this set can be treated as if it is a (non-reflexive) subset of the Cartesian product with itself.
%


\ea \label{def-sleep-gather}
\begin{xlist}
\ex \label{def-sleep}
For each set $\obt{S}$, $\denotes{\co{sleep}}(\obt{S}) = \obt{1}$ iff
%$\obt{S} \not= \emptyset$ and 
for each $\obt{o} \in \obt{S}$, $\obt{o}$ is asleep.
\ex {\label{def-gather}
For each set $\obt{S}_1, \obt{S}_2$,
\begin{xlist} 
    \ex \label{def-gather-S2}
$\denotes{\co{meet}} (\obt{S}_1 \times \obt{S}_2) = \obt{1}$\\
iff 
%$\obt{S} \not= \emptyset$ and all elements in $\obt{S}$ gather.
for each $\langle \obt{x}_1, \obt{x}_2 \rangle \in \obt{S}_1 \times \obt{S}_2$ such that $\obt{x}_1 \not= \obt{x}_2$, 
$\obt{x}_1$ and $\obt{x}_2$ meet, and
\ex  \label{def-gather-S}
$\denotes{\co{meet}}(\obt{S}_1) = \obt{1}$ iff
$\denotes{\co{meet}} (\obt{S}_1 \times \obt{S}_1) = \obt{1}$
\end{xlist}}
\end{xlist}
\z 


The difference in the denotation of the predicates allows us to have no difference in the formul\ae. This is shown in \refer{ex-sleep-gather}.

\ea \label{ex-sleep-gather}
\begin{xlist}
\ex \label{dog-sleep}
Some students slept in the library.\\
$\exists z (|z| \geq 1 \land \co{student}(z) : \co{sleep}(z))$.  
\ex \label{dog-gather}
Some students met in the library.\\
$\exists z (|z| \geq 1 \land \co{student}(z) : \co{meet}(z))$.
\end{xlist}
\z


%Not only predicates have an influence on the interpretation of plurals, but other 

Besides predicates, other elements have an influence on the interpretation of plurals as well,
such as markers of distributivity (\bsp{each}), collectivity (\bsp{together}), or reciprocity (\bsp{each other}), see \citet{Sternefeld:98}.

\citet{Chaves:07} shows how conjuncts contribute to the discourse referent of the overall conjunction. 
He assumes a new discourse referent for the conjunction. This referent is a set containing the elements denoted by the conjuncts.
We can illustrate this first with the conjunction of two proper nouns as in \refer{AlexKimTalk}.
As shown in the semantic representation, the conjunction is specified in such a way that each conjunct must be a member of the set $z$.%
\footnote{Since $z$ is existentially quantified over, we do not need to enforce that $z$ be exhaustively specified through the conjuncts, i.e., there might be more elements in $z$.}

\ea \label{AlexKimTalk}
Alex and Kim met.\\
 $\exists z 
((\co{alex} \in z
\land \co{kim} \in z)
: \co{meet}(z))$
\z

To combine our assumptions about collective predicates with Chaves's theory of coordination, 
we need to depart from \citeauthor{Chaves:07}'s analysis slightly: instead of assuming that there is a simple set built by the conjunction, we assume that there is tuple formation, i.e., in \refer{AlexKimTalk}, $z$ is not  $\{\denotes{\co{alex}}, \denotes{\co{kim}}\}$,
but rather the Cartesian product
%a set containing a pair, i.e.\@  
$\{ \denotes{\co{alex}, \ldots} \} \times \{\denotes{\co{kim}, \ldots} \}$.%

%\langle \denotes{\co{kim}}, \denotes{\co{alex}} \rangle,
%\langle \denotes{\co{alex}}, \denotes{\co{alex}} \rangle,
%\langle \denotes{\co{k}}, \denotes{\co{kim}} \rangle$.
Applying this to our example, we arrive at the semantic representation in \refer{AlexKimTalk-sum}. 
We use $\pi_i z$ to identify the $i$-th position in the tuple $z$ -- and, by extension, if $z$ is a set of tuples, $\pi_i z$ is the set of all elements that occur in the $i$-th position in any of the tuples in $z$.

%\footnote{Note that we do not need the symmetric pair, $\{ \langle \denotes{\co{kim}}, \denotes{\co{alex}} \rangle \}$ to be included in $z$ as the symmetry of the predicate \co{meet} follows from its lexical meaning.}

\ea \label{AlexKimTalk-sum}
$\exists z 
((\co{alex} \in \pI z
\land \co{kim} \in \pi_2 z)
: \co{meet}(z))$
\z

Given the way we have defined the denotation of the predicate \co{sleep} in \refer{def-sleep}, there is no mono-propositional analysis for an analogous sentence with \bsp{sleep}. 
This is shown in \refer{AlexKimSleep-sum}.

\ea \label{AlexKimSleep-sum}
Alex and Kim slept.\\
$\exists z 
((\co{alex} \in \pI z
\land \co{kim} \in \pi_2 z)
: \co{sleep}(z))$
\hspace*{\fill}(type clash!)
\z 

The formula in \refer{AlexKimSleep-sum} is ill-formed: $z$ must refer to a subset of a Cartesian product of two sets, but \co{sleep} is only defined for sets of objects, not for sets of tuples of objects.

Similarly, there is no bi-propositional analysis for the sentence in \refer{AlexKimTalk}.
The hypothetical formula is given in \refer{AlexKimTalk-bi}.
This formula is not well-formed as the predicate \co{meet} requires a set as its argument, not an individual.

\ea \label{AlexKimTalk-bi}
Hypothetical bi-propositional analysis of \refer{AlexKimTalk}:\\
$\co{meet}(\co{alex}) \land
\co{meet}(\co{kim})
$
\z 

If the conjoined noun phrases are plural, we do, of course, get both a mono- and a bi-propositional analysis. This is shown in \refer{ex-kinder-erw-streit}.

\ea \label{ex-kinder-erw-streit}
\gll Die Kinder und die Erwachsenen haben gestritten.\\
the children and the adults have quarreled.\\
\begin{xlist}
\ex 
 Mono-propositional reading: 
\mytrans{The kids quarreled with the adults.}
\ex  Bi-propositional reading: 
\mytrans{The kids quarreled among themselves and the adults quarreled among themselves.}
\end{xlist}
\z 




% \kommentar{
% %MS: raus, 14.5.19; Grund: Analyse geändert
% As the predicate \co{sleep} receives a strictly distributive interpretation, the formula has the same truth conditions as the bi-propositional representation of the sentence, given 
% %the conjunction of simple formul\ae{} 
% in \refer{AlexKimTalk2}.

% \ea \label{AlexKimTalk2}
% $\co{sleep}(\co{alex}) \land \co{sleep}(\co{kim})$
% \z 

% For a collective predicate with conjoined singular noun phrases, 
% no such equivalent bi-propositional formula would exist, since only a mono-propositional reading is possible.%
% %there is, of course a difference between the mono-propositional and the bi-propositional formul\ae, since only the mono-propositional formula is possible.%
% \footnote{If the conjoined noun phrases are plural, we do, of course, get both a mono- and a bi-propositional analysis:

% \ea
% \gll Die Kinder und die Erwachsenen haben sich gestritten.\\
% the children and the adults have themselves quarreled.\\
% \glt \textnormal{Mono-propositional reading: The kids quarreled with the adults.}
% \glt \textnormal{Bi-propositional reading: The kids quarreled among themselves and the adults quarreled among themselves.}
% \z 

% }

% }% Ende Kommentar

% \kommentar{
% % MS, raus 14.5.19, Grund: Disjunktion jetzt nicht mehr mono-prop
% In the case of a disjunction instead of a conjunction, we will just use logical disjunction in the restrictor of the existential quantifier, as shown in \refer{AlexKimOR}. Again, this has the same truth conditions as a disjunction of simple formul\ae, as in \refer{AlexKimOR2}

% \ea \label{AlexKimOR}
% Alex or Kim slept.\\
% $\exists z
% ((\co{alex} \in z 
% \lor \co{kim} \in z)
% : \co{sleep}(z))$
% \z 

% \ea \label{AlexKimOR2}
% $\co{sleep}(\co{alex}) \lor \co{sleep}(\co{kim})$
% \z 
% }


An advantage of the analysis in \citet{Chaves:07} is that it  carries over directly to quantified noun phrases. %This carries over directly to our approach.
In \refer{AlexManyCall} we give an example with the conjunction of a proper noun and a quantified noun phrase. 
As can be seen, the quantified noun phrase \bsp{many students} is integrated into the semantic representation in such a way that it takes conjunction-internal scope with just the membership requirement in the discourse referent of the conjunction, $z$, as its scope, i.e., $\co{Many}\, y \, (\co{student}(y) : y \in \pi_2 z)$

\ea \label{AlexManyCall}
Alex and many students met in the library.\\
$
\exists z
((\co{alex} \in \pI z
\land 
(\co{Many}\, y\, (\co{student}(y): y \in \pi_2 z)))
: \co{meet}(z))
$
\z 

Example \refer{AlexManyCall} also points to a final adjustment that we need to make. 
%Sentence \refer{AlexManyCall} 
It can be understood in such a way that the argument of the predicate \co{meet} is the set containing Alex and many students. This means that the students meet one another as well as Alex, not just Alex
meeting each of the many students.
We can derive this reading using the truth conditions of \co{meet} in \refer{def-gather-S}.

% \kommentar{
% To capture this, we will allow that two conjuncts may end up in the same slot of a tuple. 
% This results in $z$ being a 1-tuple $\obt{S}^1$, 
% where $\obt{S}^1 = \{\langle \obt{x} \rangle | \obt{x} \in \obt{S} \}$.


% This is not yet covered by our definition in 
% \refer{def-gather}.
% To include this case, we can add the following clause to the above definition.

% \ea Clause to be added to \refer{def-gather}:\label{def-gather-1}\\
% $\denotes{\co{meet}}(\obt{S}^1) = \obt{1}$ iff
% $\denotes{\co{meet}}(\obt{S} \times \obt{S}) = \obt{1}$ 
% \z 
% }

We will illustrate this with the example in  \refer{alex-kim-robin}.
This sentence has a reading in which the predicate \co{meet} would just take a set of one-tuples as its argument. 
This can be expressed in the semantic representation given below the sentence.

\ea \label{alex-kim-robin}
Alex, Kim, and Robin met in the library.\\
$\exists z ((
\co{alex} \in \pI z
\land \co{kim} \in \pI z
\land \co{robin} \in \pI z
): \co{meet}(z)
)$
\z 


In order to capture the systematic ambiguity of either keeping the conjuncts separate or merging them into a set of one-tuples,
we will write $\pII z$ instead of $\pi_2 z$ to indicate the position in the tuple to which the second conjunct makes its contribution.
In \refer{alex-kim-robin}, $z$ is, consequently just a set of one-tuples, which we can treat as a simple set. We can interpret $\co{meet}(z)$ according to the truth conditions given in  \refer{def-gather-S}, i.e., as equivalent to $\denotes{\co{meet}} (\denotes{z} \times \denotes{z})$. 
%
The interpretation of sentence \refer{AlexManyCall} as many students meeting one another and Alex follows in the same way. 
The semantic representation given above needs to be changed slightly, using $\pII z$ instead of $\pi_2 z$ in the second conjunct.

%\footnote{
%This simple notation works for the cases considered in the present paper, where the collective predicates have at most two arguments over which they build a unit.
%}





% \kommentar{%MS: raus: 11.5.69
% As our core data are about indefinite negated noun phrases, which will be treated as an existential quantification over sets, we need to take into consideration whether an empty set should be a possible value for a conjunct.



% In contrast to the overall conjunction, we will allow for empty sets in the conjunctions. 
% Therefore, for set-like conjuncts, we assume that they be \emph{subsets} of the conjunction object rather than elements thereof.

% }



% \kommentar{% Raus, Lesarten nicht so klar
% \citet{Chaves:07} also looks at possible scope ambiguities arising with conjoined quantifiers. 
% He discusses, among others, the example in \refer{inmates} \citep[67--69]{Chaves:07}. 
% He argues that such examples are scopally ambiguous: there is a reading, (a), with no scopal interaction among the two conjuncts, a wide-scope reading for the universal quantifier from the first conjunct over the existential in the second conjunct, given in (b), and a reading in which  the existential quantifier in the second conjunct has wide scope, see (c). 


% \ea \label{inmates}
% Every inmate and a guard met.
% \begin{itemize}
% \item [a.] $\exists z ((\forall x (\co{inmate}(x) : x \in z))
% \land 
% \exists y (\co{guard}(y) : y \in z)
% ) :
% \co{meet}(z)
% )$
% \item [b.] $\exists z (\forall x (\co{inmate}(x) : (x \in z \land 
% \exists y (\co{guard}(y) : y \in z)
% )) :
% \co{meet}(z)
% )$
% \item [c.] $\exists z (\exists y (\co{guard}(y) : ( 
% \forall x (\co{inmate}(x) : x \in z) \land y \in z
% )) :
% \co{meet}(z)
% )$
% \end{itemize}
% \z 

% Independently of the scope ambiguity, in all readings, the quantifiers contributed by the conjuncts make their ordinary contribution and have the expression $x\in z$, respectively $y\in z$, in their scope.%
% \footnote{Compatible with this formulation, \cite[69--70]{Chaves:07} assumes that the quantifiers contributed by the conjuncts can take scope over the existential quantifier of the conjunction.}
% Though we will not discuss such examples in the following, our combinatorial account will allow us to derive these readings as well.

% }

% \kommentar{% raus: Mai 2018
% This analysis correctly accounts for the observation that a quantifier in one conjunct cannot bind a pronoun in the rest of the sentence, as illustrated in \refer{ex-PN-QN-x}.%
% \footnote{\citet{Chaves:07} argues that the scope of the conjuncts is more flexible than we allow it to be. 
% }

% \ea \label{ex-PN-QN-x}
% \gll Alex erzählt Bert und jeder$_i$ Studentin, dass sie$_i$ den Kurs bestanden hat.\\
% Alex tell Bert.\textsc{masc} and every student.\textsc{fem} that she the course passed has\\
% \glt \#\mytrans{Alex is telling Bert and every$_i$ (female) student that she$_i$ passed the course.}\\
% $
% \exists z ((
% \co{bert} \in z
% \land (\forall y (\co{female-student}(y) : y \in z)
% ) : \co{tell}(\co{alex},z, \co{passed}(y))
% $
% \z 
% }

\bigskip%
\begin{sloppypar}
  In this subsection, we have presented a semantic analysis for
  mono-propositional readings of conjoined noun phrases.  While our
  approach relies on the insights of \citet{Chaves:07}, we provided a
  tuple-based formulation of some of his core ideas.
  % conjunction presented in \citet{Chaves:07} in the way in which we
  % are going to use it in the rest of this paper.
  We can, now, combine the analysis of conjunction with an analysis of
  n-words.
\end{sloppypar}

\subsection{Wide-scope negation and the disjunction effect}
\label{Sec-DisjunctionEffect}

%\marginpar{MS: angepasst!}
N-words are often analyzed as existential quantifiers in the scope of negation, which is exactly what we will do here. We saw in the data discussion that CNNP introduces a negation that takes scope over the rest of the sentence. 
In \refer{alex-cnnp-lf}, we show the semantic representation for a simple CNNP-sentence.

\vbox{
\ea \label{alex-cnnp-lf}
\gll Alex vergleicht/ liest [\NE{keinen} Brief und \NE{keine} e-Mail].\\
Alex compares/ reads no letter and no {e-mail message}\\
\glt{Alex is comparing / reading no letter and no e-mail message.}
\begin{xlist}
\ex \label{alex-cnnp-mono}
Mono-propositional (for \bspT{vergleichen}{compare}): \\
$\lnot \exists z ((
\exists x (\co{letter}(x) : x \in \pI z)
\land 
\exists y (\co{e-mail-mess}(y): y \in \pII z))$\\
\qquad $: \co{compare}(\co{alex},z))$
\ex \label{alex-cnnp-bi}
Bi-propositional (for \bspT{lesen}{read}): \\
$\lnot \exists x (\co{letter}(x) : \co{read}(\co{alex},x))
\land 
\lnot \exists x (\co{e-mail-mess}(x) : \co{read}(\co{alex},x))
$
\end{xlist}
\z 
}

According to the mono-propositional reading, there is no set of pairs $z$ that contains pairings of letters with e-mail messages such that Alex is comparing any of the items in this tuple. 
%
For the strictly distributive interpretation of the complement of the predicate \co{read}, we find a coordination of two negated formul\ae{} with identical or parallel expressions in their scope.


%two negated formulae that share the same scope.
%, the indicated reading, therefore, has exactly the same truth conditions as the bi-propositional reading.


The formul\ae{} in \refer{alex-cnnp-lf} also account for the NPI-licensing potential of CNNP: 
there is a negation in the semantic representation that takes scope over the contribution of the NPI.
%the highest operator in the formula is the negation, so
Consequently, we expect NPIs to be possible in each conjunct and in the rest of the clause. This is the case, as shown in \refer{jemals3}. 

\ea \label{jemals3}
\gll [[\NE{Kein} Student, der \npi{jemals} in meinem Kurs war,] und [\NE{kein} Student, der \npi{jemals} in deinem Kurs war,]] wird \npi{jemals} vergessen, was dort unterrichtet wurde.\\
\hphantom{[[}no student who ever in my course was and \hphantom{[}no student who ever in your course was will ever forget what there taught was\\
\glt \mytrans{No student who has ever been in my class and no student who has ever been in your class will ever forget what was taught there.}
\z 


% \kommentar{
% $\lnot \exists x (\co{letter}(x) \land  \co{write}(\co{alex},x))
% \land \lnot \exists x (\co{e-mail-mess}(x) \land  \co{write}(\co{alex},x))
% $\\
% $\equiv 
% \lnot (\exists x (\co{letter}(x) \land  \co{write}(\co{alex},x)) 
% \lor \exists x (\co{e-mail-mess}(x) \land  \co{write}(\co{alex},x)))
% $\\
% $\equiv$
% }


Finally, we will show how the disjunction effect follows from the introduced representations. 
%We saw that the mono-propositional and the bi-propositional analyses are truth-conditionally equivalent for distributive readings. 
The bi-propositional formula from \refer{alex-cnnp-lf} is logically equivalent to the one given in \refer{alex-cnnp-lf3a}, in which there is a disjunction in the restrictor of the existential quantifier, i.e., the quantification takes any $x$ into consideration that is a letter or an e-mail message.


\ea \label{alex-cnnp-lf3a}
$\lnot \exists x ((\co{letter}(x) \lor \co{e-mail-mess}(x)) \land \co{read}(\co{alex},x))
$
\z 

The equivalence between \refer{alex-cnnp-bi} and \refer{alex-cnnp-lf3a} follows directly: if nothing that is either a letter or an e-mail message is being read by Alex, this is the same as saying that Alex is neither reading a letter nor an e-mail message. This means, there is no letter such that Alex is reading it and there is no e-mail message such that Alex is reading it, either.


% \kommentar{Das ist nicht der Punkt:
% This is not equivalent to an intepretation that combines the two conjuncts with a logical conjunction, i.e., the formular in \refer{alex-cnnp-lf4}. In this reading, $x$ would be required to be both a letter and an e-mail message.

% \ea \label{alex-cnnp-lf4}
% $\lnot \exists x ((\co{letter}(x) \land \co{e-mail-mess}(x)) \land \co{write}(\co{alex},x))
% $
% \z 
% }



The formula in \refer{alex-cnnp-lf3a} shows that the expression in \refer{alex-cnnp-lf} captures the disjunction effect, i.e., while the conjoined noun phrases are combined logically with a conjunction, the overall interpretation is rather like a disjunction. 

As Zeijlstra (personal communication) pointed out to us, a mono-pro\-po\-si\-tional analysis would lead to a ``not both'' reading for a CNNP sentence with a distributive predicate: if Alex were reading a letter but no e-mail message, there would not be a pair or a plural object containing both a letter and an e-mail message being read by Alex. 
We saw in Section \ref{Sec-Conjunction}, example \refer{AlexKimSleep-sum}, that a distributive predicate cannot take a tuple as its argument. 
Consequently, a mono-propositional analysis of a CNNP sentence with a verb like \bsp{read} would lead to a type clash.

\medskip%
We can now turn to the contrast between CNNP and the negation of conjoined indefinite noun phrases, illustrated in examples \refer{ex-tie-glaub-both} and \refer{ex-tie-glaub-neither} above. 
The contrast only arose in the cases in which a bi-propositional reading is possible. 
We observed that CNNP excludes a ``not both'' reading, which is readily available for negated conjoined indefinites.

\ea \label{lf-notboth}\label{lf-neither}
\begin{xlist}
\ex 
``not both'' reading:\\ 
\hspace*{-1.5em}
$\lnot (\exists x (\co{lecture}(x) : \co{teach}(\co{monika},x))$\\
\hspace*{1em}$
\land \exists x (\co{seminar}(x) : \co{teach}(\co{monika},x))$)\\
\ex ``neither'' reading:\\ 
\hspace*{-1.5em}
$\lnot \exists x (\co{lecture}(x) : \co{teach}(\co{monika},x))$\\
\hspace*{1em}$
\land \lnot \exists x (\co{seminar}(x) : \co{teach}(\co{monika},x))$
\end{xlist}
\z 

\begin{sloppypar}
  The difference between the two readings lies in the scope of the
  negation: for the ``not both'' reading, the negation has wide scope
  over the two conjoined propositions, in the ``neither'' reading each
  of the conjuncts keeps its negation. We will have to provide an
  analysis that allows for split readings with CNNP on the one hand
  but, on the other, blocks wide-scope negation for the
  bi-propositional interpretation.
\end{sloppypar}

\bigskip
In this section, we introduced and discussed a semantics of negated, potentially plural, noun phrases. We showed that this semantics is  compatible with our observations on  CNNP. 
In the next section, we will present the framework  of the semantic combinatorics that we will adopt for our analysis of the data.

\section{Lexical Resource Semantics}
\label{Sec-LRS}

Lexical Resource Semantics (LRS, \citealt{Richter:Sailer:04}) is a system of constraint-based, underspecified semantic combinatorics.
It has been developed to account for problems with a traditional concept of compositionality. 
The basic idea behind any LRS analysis is that the syntactic structure should be determined by syntactic considerations and the semantic representation by semantic considerations. 
This sets LRS apart from LF-approaches as those mentioned in Section~\ref{Sec-Previous}, which assume a syntactic representation that directly reflects the semantic representation. It is also different from categorial grammar, which questions the entire notion of an independent syntactic constituent structure.
From its first publications on, negation and negative concord, as well as other cases of semantic concord, played an important role in the development of LRS.% 
\footnote{\citet{Richter:Sailer:01.1} look at the occurrence of multiple interrogatives, 
\citet{Sailer:04.sub} discusses temporal concord, and \citet{Sailer:10} proposes a semantic concord-analysis of cognate objects.
}
We will present the necessary background on LRS in Section~\ref{Sec-LRS-Basics}. 
In Section~\ref{Sec-LRS-One2Many}, we will go through four aspects of \emph{one-to-many} correspondences that follow from the general architecture of LRS. We will use these to introduce the LRS treatment of negation and coordination. 
%and the LRS-analysis of NC from \citet{Richter:Sailer:06} in Section~\ref{Sec-LRS-NC}.

\subsection{Underspecified constraint-based combinatorics}
\label{Sec-LRS-Basics}

In LRS, we use a standard semantic representation language, like the one used in Section~\ref{Sec-Semantics}. 
We enrich this language with \emph{metavariables}, which we will write as upper case letters.% 
\footnote{We will be using a variant of the notation introduced in the computational implementation of LRS in \citet{Penn:Richter:04,Penn:Richter:05}.}
A metavariable can denote any formula of the underlying semantic representation language.
For any formul\ae{} $\phi_1, \ldots \phi_n$ of our extended language and any metavariable $A$, $A[\phi_1, \ldots \phi_n]$ restricts the denotation of $A$ to formul\ae{} containing all of $\phi_1, \ldots \phi_n$ as subexpressions. 
When convenient, we may write $\phi \compo A$ to express that $A$ must refer to an expression from our underlying representation language that contains the denotation of $\phi$.

LRS is a constraint-based framework in the sense that all words and phrases \emph{constrain} the possible semantic representation of a sentence. 
There are two basic types of constraints: \emph{contribution constraints} and \emph{component constraints}. 
Contribution constraints determine which constants, variables, predicates, and operators of the representation language occur. 
For example, the name \bsp{Alex} determines that whenever it is used in a sentence, the semantic representation of this sentence will contain an occurrence of the constant \co{alex}. In LRS, contribution constraints can only be made by lexical elements, i.e., LRS heavily relies on ``lexical resources''.

Component constraints indicate which expressions must be a component of other expressions. All meta-expressions of the form $A[\phi_1, \ldots \phi_n]$ or $\phi \compo A$ are component constraints. 
Component constraints restrict the possible readings of a sentence. They can be imposed by lexical elements but also by the syntactic structure, i.e., by the principles of semantic combinatorics.

In \refer{le-niemand}, the semantic constraints of the word \bspT{niemand}{nobody} are shown. Whenever the word is used, there will be a negation in the sentence, an existential quantification binding the variable $x$, the variable $x$ itself, and the formula $\co{person}(x)$. In addition to these contribution constraints, there are also a number of component constraints: (i) the existential quantifier is in the scope of the negation -- though not necessarily in its immediate scope, (ii) the formula $\co{person}(x)$ occurs in the restrictor of the existential quantifier, and (iii) the scope of the existential quantifier contains variable $x$ at least once.

\ea \label{le-niemand}
\bspT{niemand}{nobody}: $\lnot A[\exists x (B[\co{person}(x)] : B'[x])]$
\z

The semantic constraints of a verb are shown in \refer{le-schlaeft}. The verb contributes a predicate, \co{sleep}, and its application to the discourse referent of its subject. However, it
does not contribute that discourse referent. This is an indirect contribution constraint, i.e. the occurrence of some expression $x$ is required but the expression is not contributed. 
We indicate indirect contribution constraints 
by using a gray background with white font instead of black on white, i.e., \noc{x} instead of $x$.
%MS alt: with the notation $\noc{x}$. 
All expressions from our semantic representation language that are not included in a contribution constraint in a given linguistic sign will be marked in this way.


\ea \label{le-schlaeft}
 \bspT{schläft}{is asleep}: $C[\co{sleep}(\noc{x})]$
\z 

For the purpose of semantic combinatorics, we add three more diacritic markings to our metaformul\ae. 
%MS: alter Test: 
%Each nominal expression will mark its discourse referent by putting a box around it, i.e.\@ \dr{x} in \refer{le-niemand}. 
For each nominal expression, we will mark its discourse referent by a wavy underlining, 
i.e.\@ \dr{x}. 
The semantics associated with a phrase will be called its \emph{external content}, marked as $\exc{\phi}$. The \emph{internal} content will be the part of the semantic contribution of the head of a phrase 
that is scoped over by all semantic operators that occur as non-heads in this phrase. 
This 
is displayed as $\inc{\phi}$. The discourse referent, the external content, and the internal content percolate along the syntactic head projection.

We will enhance the two lexical specifications we have so far by these three additional markings.

\ea \label{le-niemand-schlaeft-diacritic}
\bspT{niemand}{nobody}: $\lnot A[\exc{\exists \dr{x} (B[\inc{\co{person}(x)}] : B'[x])}]$\\
\bspT{schläft}{is asleep}: $\exc{C[\inc{\co{sleep}(\noc{x})}}]$
\z


When we combine the two words, we get the clause in \refer{niemand-schlaeft}. Since the verb is the syntactic head of the clause, the external and internal content of the clause are the same as those of the verb as given in \refer{le-niemand-schlaeft-diacritic}.

\ea \label{niemand-schlaeft}
\gll (dass) niemand schläft\\
that nobody is.asleep\\
\glt \mytrans{that nobody is asleep}\\
Constraints: \begin{tabular}[t]{l}
$\exc{C[\lnot A[\exists x (B[\co{person}(x)] : B'[x])], \inc{\co{sleep}(\noc{x})}]}$\\
$\co{sleep}(\noc{x}) \compo B'$
\end{tabular}
\z 

There are some more combinatorial constraints. In this paper, we need the principles for quantified expressions and the so-called \emph{External Content Principle}. We will briefly illustrate these.

First, we assume a number of combination-specific principles. When a quantified noun phrase is the non-head combining  with a head, then the head's internal content is a component of the quantifier's scope. This can be seen in the second constraint given in \refer{niemand-schlaeft}. When a quantificational determiner combines with the rest of a noun phrase, the internal content of the rest of the noun phrase will be a component of the determiner's restrictor.

Second, the \emph{External Content Principle} constrains the external content. It has various clauses, which are contingent on 
 the structural completeness of a linguistic sign. 
For each phrase, there will be some expression that satisfies all constraints contributed by the daughters. 
This is the expression denoted by the metavariable $C$ in \refer{niemand-schlaeft}. 
For a complete utterance, there is an even stronger constraint: The external content of an utterance is a formula that consists all and only of those logical expressions mentioned in contribution constraints and satisfies all component constraints. Given the constraints in \refer{niemand-schlaeft}, there is exactly one formula satisfying the External Content Principle on utterances: $\lnot \exists x (\co{person}(x) : \co{sleep}(x))$.

We can verify that this formula is a possible semantic representation of the sentence by assigning  subexpressions of the formula to the metavariables in \refer{niemand-schlaeft}. 
If we get an assignment that is consistent with the constraints, the formula is a possible reading of the sentence. 
We will call such an assignment of expressions to metavariables a \emph{plugging}, following the terminology of \citet{Bos:96}.
The relevant plugging for our example is given in \refer{schlaeft-plug}.

% \kommentar{
% The semantic representation of an utterance can be determined by a \emph{plugging} \citep{Bos:96}. 
% A plugging assigns each metavariable an expression of the underlying semantic representation language in such a way that the component constraints are all satisfied.

% }

\ea \label{schlaeft-plug}
\begin{tabular}[t]{ll}
$B' = \co{sleep}(x)$ & 
$B = \co{person}(x)$\\
$A = \exists x (\co{person}(x):\co{sleep}(x))$ & 
$C = \lnot \exists x (\co{person}(x):\co{sleep}(x))$
\end{tabular}

\z



%\citet{Richter:Sailer:04.polish}\citet{Richter:Sailer:04}

\subsection{One-to-many relations in LRS}
\label{Sec-LRS-One2Many}

The basic mechanism of LRS is sufficient to capture one-to-many relations at the syntax-semantics interface. 
We will go through the following four of such one-to-many relations in this subsection: (i) scope ambiguity, (ii) split readings, (iii) semantic concord, and (iv) implicit semantic material. 
We will go through these four in this subsection.

\subsubsection{Scope ambiguity}
\label{Sec-LRS-Ambig}

Scopally ambiguous sentences have been the primary motivation for the development of underspecified semantics in computational linguistics, see \citet{Pinkal:95} and \citet{Bos:96}. 
Such sentences are instances of one-to-many correspondences, as there is one syntactic form associated with more than one semantic representation.  
Our metaformul\ae{} are ambiguous if and only if there is more than one possible plugging. 

This can be illustrated with the following example sentence. The semantic constraints contributed by the words are given in \refer{every-not-le}.

\ea
\gll Jeder schläft nicht.\\
everyone is.asleep not\\
\glt \mytrans{Everyone is not asleep.}
\z 

\ea \label{every-not-le}
\bspT{jeder}{every}: $\exc{\forall \dr{x}(B[\inc{\co{person}(x)}]:B'[x])}$\\
\bspT{nicht}{not}: $\lnot A$
\z

Combining the lexical constraints with those for \bspT{schläft}{is.asleep} in the standard way, we arrive at the metaformula in \refer{every-not-lf}.

\ea \label{every-not-lf}
$\exc{C[\lnot A, \forall x (B[\co{person}(x)] : B'[x]), \inc{\co{sleep}(\noc{x})}]}$\\
$\co{sleep}(\noc{x}) \compo B'$
and
$\co{sleep}(\noc{x}) \compo A$
\z

In this metaformula, the relative scope of the negation and the universal quantifier is not constrained. Consequently, there are two possible pluggings. In \refer{every-not-plug-wide}, the reading with wide scope for the negation is given, in \refer{every-not-plug-narrow}, the negation is interpreted in the scope of the universal quantifier.

\ea \label{every-not-plug-wide}
\begin{tabular}[t]{ll}
$B = \co{person}(x)$ & 
$B' = \co{sleep}(x)$\\
$A = \forall x (\co{person}(x): \co{sleep}(x))$ & 
$C = \lnot \forall x (\co{person}(x): \co{sleep}(x))$
\end{tabular}\\
Resulting reading: $\lnot \forall x (\co{person}(x) : \co{sleep}(x))$
\z 

\ea \label{every-not-plug-narrow}
\begin{tabular}[t]{ll}
$B = \co{person}(x)$ & 
$B' = \lnot \co{sleep}(x)$\\
$A = \co{sleep}(x)$ & 
$C = \forall x (\co{person}(x): \lnot \co{sleep}(x))$
\end{tabular}\\
Resulting reading: $\forall x (\co{person}(x) : \lnot \co{sleep}(x))$
\z 

As this example illustrates, we can derive more than one reading, depending on how we interpret the metavariables. In \refer{every-not-plug-wide}, the quantified formula is in the immediate scope of the
negation, it equals A. In the second reading, \refer{every-not-plug-narrow}, the negation is in
the scope of the quantifier, $B'$, and the negated expression equals $B'$.

\subsubsection{Split readings}
\label{Sec-LRS-Split}

In the narrow-negation reading, the negation is interpreted as taking scope over the atomic formula $\co{sleep}(x)$ and within the scope of the quantifier. This way of talking about the reading in \refer{every-not-plug-narrow} characterizes this reading as a form of ``intervention'' or, in fact, as some ``split reading''. 
 A split reading can always arise when a word contributes lexical constraints with at least 
 one operator and does not fully specify the scope of this operator.

 
 %two operators %$O_1$ and $O_2$ 
 %and do not enforce that one of them is in the immediate scope of the other.
 
% that contain a metavariable which is required to contain some expression from our ordinary semantic representation level. 
 
 Let us give a very simple example for illustration. We assume a purely epistemic interpretation of the modal verb \bspT{müssen}{must} in \refer{niemand-muss}. There are three such epistemic readings, differing with respect to the scope of the negation, as indicated by the paraphrases and by the semantic representations.%
 \footnote{We write ``$\Box$'' for the necessity operator. Of course, \refer{niemand-muss} has deontic readings as well.}%
 $^,$\footnote{Readings in which the necessity operator has scope over negation are clearly dispreferred in German.
 As Zeijlstra (personal communication) pointed out to us, the reading in \refer{niemand-muss-BoxNotEx} is certainly not common, if available at all. 
 For us, it seems available in principle, though we assume that its degradedness follows from other, general scope preferences of the modal operator.
 }
 
 \ea \label{niemand-muss}
\gll Niemand muss schlafen.\\
nobody must sleep\\
\begin{xlist}
\ex \glt \mytrans{For nobody is it necessary to sleep.}
$\lnot \exists x (\co{person}(x) : \Box \co{sleep}(x))$%\\[0.5ex]
\ex \glt \mytrans{It is not necessary that anybody sleeps.}
$\lnot \Box \exists x (\co{person}(x): \co{sleep}(x))$%\\[0.5ex]
\ex \glt \mytrans{It is necessary that nobody sleeps.}
$\Box \lnot \exists x (\co{person}(x): \co{sleep}(x))$
\label{niemand-muss-BoxNotEx}
\end{xlist}
 \z 



We give a very simple set of semantic constraints for the verb \bspT{müssen}{must} in \refer{le-must}.%
%\footnote{In addition, the semantics of the verb embedded by the modal must be within the scope of the modal operator `$\Box$".}

\vbox{
\ea \label{le-must}
\bspT{müssen}{must}: $\Box (D)$
\z 

}

Given the lexical and combinatorial constraints, we arrive at the metaformula in \refer{niemand-muss-lf} for sentence \refer{niemand-muss}. Semantically, the modal behaves like the negation in \refer{every-not-lf}. 
It introduces a propositional operator and requires that the internal content of the verb \bspT{schlafen}{sleep} be in its scope. The resulting underspecified formula is given in \refer{niemand-muss-lf}.

\ea \label{niemand-muss-lf}
$\exc{C[\lnot A[\exists x (B[\co{person}(x)] : B'[x])], \inc{\co{sleep}(\noc{x})}, \Box(D)]}$\\
$\co{sleep}(\noc{x}) \compo B'$
and
$\co{sleep}(\noc{x}) \compo D$
\z 

\begin{sloppypar}
  There are three possible pluggings for this metaformula: the scope
  of the modal operator can contain only the verb's internal content,
  that plus the existential quantifier, or the entire negated
  formula. In \refer{niemand-muss-plug}, the readings are shown
  together with the relevant parts of these three pluggings.
\end{sloppypar}
\ea \label{niemand-muss-plug}
\begin{xlist}
\ex Reading 1: $\lnot \exists x (\co{person}(x) : \Box (\co{sleep}(x)))$\\
\hspace*{5em} $D = \co{sleep}(x)$ (i.e., $D = B'$)
\ex Reading 2: $\lnot \Box (\exists x (\co{person}(x) : \co{sleep}(x)))$\\
\hspace*{5em} $D = \exists x (\co{person}(x) : \co{sleep}(x))$ (i.e.\@ $D = A$)
\ex Reading 3: $\Box (\lnot \exists x (\co{person}(x) : \co{sleep}(x)))$\\
\hspace*{5em} $D = \lnot \exists x (\co{person}(x) : \co{sleep}(x))$
\end{xlist}
\z 

The difference between the first and the second reading is real but subtle: In the first reading, the predicate \co{person} is not interpreted in the scope of the modal operator. If it is a world-dependent predicate, as is often assumed in intensional semantics, there might be an individual $a$ that is a person in one world but not in another world. In Reading 1 we quantify existentially over persons in the world of evaluation, in Reading 2 over individuals that are persons in the modally quantified world.%
\footnote {
This contrast is clearer in examples like \refer{modal-not}:
\ea \label{modal-not}
%HIER EIN SCHÖNES BSP REIN
\gll Michelle wollte keinen Präsidenten heiraten.\\
Michelle wanted no president marry\\
\glt \mytrans{Michelle did not want to marry a president.}\\
Reading 1: It is not the case that there is a current president such that Michelle wanted to marry him.\\
Reading 2: It is not the case that Michelle wanted to marry someone who was a president at the time of their wedding.\\
Reading 3: What Michelle wanted was not to get married to a person who was president at the time of their wedding.
\z 
}

\subsubsection{Semantic concord}
\label{Sec-LRS-NC}

The examples discussed so far show that the underspecification mechanism accounts for both scope ambiguity and split readings. 
We can now turn to concord, which we will illustrate with negative concord.
We assume that the analysis of all languages with n-words is based on 
%We assume that n-words have 
the same lexical semantic contribution independently of a language's NC-type, i.e. whether it is an NC language like Polish, a non-NC language like German, or an optional NC language like French.
The languages have the same underspecified semantic representations of sentences with n-words, but different types of languages use different interpretation strategies, i.e.\@ impose different constraints on the kinds of pluggings they allow \citep{Richter:Sailer:06}. As mentioned at the beginning of this chapter, French allows for a SN and a DN reading of a sentence with two n-words. This is shown in \refer{personne-personne}.

\ea \label{personne-personne}
\gll Personne (ne) connaît personne.\\
nobody NE knows nobody\\
\begin{xlist}
\ex 
\glt SN: \mytrans{Nobody knows anybody}\\
$\lnot \exists x (\co{person}(x) : \exists y (\co{person}(y) : \co{know}(x,y)))$%\\[0.5ex]
\ex 
\glt DN: \mytrans{Everyone knows someone}\\
$\lnot \exists x (\co{person}(x) : \lnot \exists y (\co{person}(y) : \co{know}(x,y)))$\\
$\equiv \forall x (\co{person}(x) : \exists y (\co{person}(y): \co{know}(x,y)))$
\end{xlist}
\z 

Let us consider how we derive these readings.
Ignoring the pre-verbal negation marker \bsp{ne}, we assume the following lexical constraints for the words in the sentence.

\ea \label{le-personne-personne}
\begin{xlist}
\ex Subject: \bspT{personne}{nobody}: $\lnot A[\exc{\exists \dr{x} (B[\inc{\co{person}(x)}]: B'[x])}]$
\ex Complement: \bspT{personne}{nobody}: $\lnot D[\exc{\exists \dr{y} (E[\inc{\co{person}(y)}]: E'[y])}]$
\ex Verb: \bspT{(ne) connaît}{NE knows}: $\exc{C[\inc{\co{know}(\noc{x},\noc{y})}]}$
\end{xlist}
\z 

When these lexical constraints are combined in a sentence, we arrive at the semantic constraints in \refer{personne-personne-lf}. 
This metaformula contains all constraints from the lexical entries. In addition, the combinatorial principles enforce that the verb's internal content be in the scope of each of the two quantified noun phrases.

\ea \label{personne-personne-lf} 
$\exc{C}[$\\
$\lnot A[\exists x (B[\co{person}(x)]: B'[x])],$ \hfill (subject)\\
$\lnot D[\exists y (E[\co{person}(y)]: E'[y])],$ \hfill (object)\\
$\inc{\co{know}(\noc{x},\noc{y})}
]$ \hfill (verb)\\
and $\co{know}(\noc{x},\noc{y}) \compo B'$
and $\co{know}(\noc{x},\noc{y}) \compo E'$
\z 

For simplicity, we will ignore the possible ambiguity of the relative scope of existential quantifiers contributed by the subject and the object and assume that the subject outscopes the object here. What is relevant for us, however, is to consider the negation(s). We need to remember that we are working in a \emph{constraint-based} framework. This means that our metaformul\ae{} impose constraints on what the real formul\ae{} can look like. An n-word therefore states that the semantic representation in which it occurs must contain a negation and that this negation must take scope over the existential quantifier that binds the discourse referent associated with the n-word.

Under the DN reading of sentence \refer{personne-personne}, this constraint is satisfied for both of the n-words: for each n-word, we have a negation scoping over the corresponding existential quantifier. Note that the outmost negation, in fact, has both of these quantifiers in its scope. We can now turn to the SN reading. Maybe surprisingly, it also satisfies the constraints of the n-words: each of the existential quantifiers is in the scope of a negation in the semantic representation. In \refer{personne-personne-plug}, we indicate the pluggings responsible for the two readings.

\ea \label{personne-personne-plug}
\begin{xlist}
\ex DN:\label{pers-pers-DN}\\
$A = \exists x (\co{person}(x) : \lnot \exists y (\co{person}(y) : \co{know}(x,y)))$\\
$B = \co{person}(x)$ \qquad 
$B' = \lnot \exists y (\co{person}(y) : \co{know}(x,y))$\\
$C = \lnot \exists x (\co{person}(x) : \lnot \exists y (\co{person}(y) : \co{know}(x,y)))$\\
$D = \exists y (\co{person}(y) : \co{know}(x,y))$\\
$E = \co{person}(y)$ \qquad 
$E' = \co{know}(x,y)$
\ex SN:\label{pers-pers-SN}\\
$A = \exists x (\co{person}(x) : \exists y (\co{person}(y) : \co{know}(x,y)))$\\
$B = \co{person}(x)$ \qquad 
$B' = \exists y (\co{person}(y) : \co{know}(x,y))$\\
$C = \lnot \exists x (\co{person}(x) : \exists y (\co{person}(y) : \co{know}(x,y)))$\\
$D = \exists x (\co{person}(x) : \exists y (\co{person}(y) : \co{know}(x,y)))$\\
$E = \co{person}(y)$ \qquad 
$E' = \co{know}(x,y)$
\end{xlist}
\z 

The relevant parts of the pluggings are the values for the scopes of the negations, $A$ and $D$. In the DN reading, these are assigned distinct formul\ae. In the SN reading, they are identical.%
\footnote{\citet{Egg:10} notes that
LRS is the only system of underspecified semantic combinatorics that allows this type of identity of the interpretation of metavariables. %More computational approaches might be concerned that this makes the system less resource sensitive and, therefore, less efficient.
}

LRS is a genuinely ambiguity-friendly system. Therefore, the
ambiguity that we find for optional NC languages is accounted for without any additional assumptions. 
For strict NC languages and for non-NC languages, we need to impose constraints that reflect the interpretation strategies of these languages. 
In other words, such languages have additional principles that filter out one of the pluggings from \refer{personne-personne-plug}. 
The constraints required for this are elaborated in some detail in \citet{Richter:Sailer:06} and we will just summarize them briefly here.

For a strict NC language like Polish, we require that the external content of a verb contain at most one negation that takes scope over the verb's internal content. This constraint  excludes the DN-plugging in \refer{pers-pers-DN}.
The interpretive strategy of NC languages is very simple and leads to slim semantic representations. This might account for the fact that NC is the typologically most frequent interpretation strategy for sentences with two n-words.

A non-NC language like German, on the other hand, employs a different strategy, which \citet{Richter:Sailer:06} call \emph{negation faithfulness}, alluding to the optimality theoretic account of negation systems in \citet{deSwart:10}. This faithfulness constraint is given in \refer{neg-faith} in a form that is adapted to the present notation and relativized to headed phrases.

\vbox{
\ea \label{neg-faith}
Negation Faithfulness Constraint (NFC, adapted from \citealt{Richter:Sailer:06})\\
In every headed phrase, whenever one daughter has a constraint $\lnot A$ and another daughter has a constraint $\lnot B$, the overall phrase has a constraint $A \not= B$.
%In every (binary headed) phrase: there is no element of the form $\lnot \alpha$ which is on the \attr{parts} list of both the head-daughter and the nonhead-daughter.
\z 

}

Given the NFC, the German equivalent of sentence \refer{personne-personne} has a constraint on its semantic representation that requires that $A$ (the scope of the negation contributed by the subject) and $D$ (the scope of the negation contributed by the complement) be distinct. This rules out the plugging in \refer{pers-pers-SN}, which leads to the SN reading.

\subsubsection{Implicit semantic material: Identical material}
\label{Sec-ImplicitSemMat}

%DAS SOLLTE VIELLEICHT ANDERS HEIßEN, WEIL WIR JA NICHTS ÜBER ELLYPSE SAGEN WOLLEN

%The header of this subsection is certainly less clear than those of the previous ones. 
In this subsection, we will discuss cases in which there seems to be more material required in the semantic representation than is apparently contributed by the elements overtly occurring in syntax. An obvious case in point is ellipsis, but more relevant to us here is the bi-propositional analysis of sentences with NP conjunction.

%
%\footnote{\citet{Pinkal:96} already included this type of phenomena in his list of cases for which underspecified semantic approaches can offer interesting solutions. He looked at more general cases than we, introducing what he called ``equality up-to'' constraints.}

While the three one-to-many phenomena discussed earlier in this section have been studied intensely in LRS, no work on elliptical constructions exists so far.
However, the technique that we will use to account for elliptical data has been applied in previous approaches: in \citet{Sailer:05.hpsg} for LRS and in \citet{Bonami:Godard:07} for a version of \emph{Minimal Recursion Semantics}.
We will concentrate here only on bi-propositional interpretations of sentences with conjoined noun phrases, i.e., there is not necessarily any material missing in syntax, but we have one sentence that receives the same semantic representation as a conjunction of two sentences.

We can illustrate this with the sentence in \refer{hund-katze-schlaf}, for which we intend to derive the bi-propositional reading given below the example.

\ea \label{hund-katze-schlaf}
\gll Ein Hund und eine Katze schlafen.\\
a dog and a cat are.asleep\\
\glt $(\exists x (\co{dog}(x) : \co{sleep}(x)) \land \exists x (\co{cat}(x) : \co{sleep}(x)))$
\z 

The important aspect here is the lexical specification of the coordination particle. 
We can safely assume that the particle selects its conjuncts. In HPSG, a selector has access to syntactic and semantic information of the selected elements. 
We argued in \citet{Richter:Sailer:04} and \citet{Sailer:04.cssp} that the discourse referent marker of the selected element should be visible for selection.%
\footnote{\label{fn-main}
We also assume that the ``main'' lexical semantic predicate contributed by a word should be visible. We will ignore this ``main'' content in the present paper, though.}
%\footnote{In fact, we motivate a further distinction between the \emph{internal content} and the \emph{main} content and make only the latter visible for selection. For the present paper, however, we can ignore this distinction as the main content and the internal content will be the same for the examples discussed here.}

With these assumptions, we can provide the semantic constraint of the 
coordination particle \bspT{und}{and} in \refer{le-und1}. 

% \kommentar{%raus: verwendet MAIN
% \ea \label{le-und1}
% \bspT{und}{and}: $H[\inc{(F[F',\dr{x}] \land G[G',\dr{x}])}]$\\
% where $x$ is the discourse referent marker of both conjuncts, $F'$ is the internal content of the first conjunct, and $G'$ is the internal content of the second conjunct.
% \z 
% }

\ea \label{le-und1}
\bspT{und}{and}: $\inc{(F[\dr{x}] \land G[\dr{x}])}$\\
where $x$ is the discourse referent marker of both conjuncts.
%, $F'$ is the internal content of the first conjunct, and $G'$ is the internal content of the second conjunct.
\z 


The word \bsp{und} contributes a logical coordination. It states that the two conjuncts and the overall conjunction use the same variable for their discourse referents.
%Furthermore, the internal contents of the conjoined elements must appear within the semantic representation of the respective conjuncts.

In \refer{le-hund-katze-schlaf}, we provide the semantic constraints for the two conjuncts 
in \refer{hund-katze-schlaf}.

\ea \label{le-hund-katze-schlaf}
\begin{xlist}
\ex \bspT{ein Hund}{a dog}: $A[\exc{\exists \dr{x} (B[\co{dog}(x)]: B'[x])}]$
\ex \bspT{eine Katze}{a cat}: $D[\exc{\exists \dr{x} (E[\co{cat}(x)]: E'[x])}]$
%\bspT{schlafen}{sleep}: $\co{sleep}(\noc{x})$
\end{xlist}
\z 

These combine into the conjoined noun phrase \bsp{ein Hund und eine Katze}, whose constraint is given in \refer{le-hund-und-katze}. The resulting constraint collects the constraints of the coordination particle and the two conjuncts.

% Old version:
% \ea \label{le-hund-und-katze}
% \bspT{ein Hund und eine Katze}{a dog and a cat}:\\
% $H[
% %F[\co{dog}(x),x] \land G[\co{cat}(x),x],$ \hfill (coordination particle)\\
% F[x] \land G[x],$ \hfill (coordination particle)\\
% $A[\exists x (B[\{\co{dog}(x)\}]: B'[x])],$ \hfill (first conjunct)\\
% $D[\exists x (E[\{\co{cat}(x)\}]: E'[x])]$ \hfill (second conjunct)
% \z 

\ea \label{le-hund-und-katze}
\bspT{ein Hund und eine Katze}{a dog and a cat}:\\
$H[
F[x] \land G[x],$ \hfill (coordination particle)\\
$A[\exists x (B[\{\co{dog}(x)\}]: B'[x])],$ \hfill (first conjunct)\\
$D[\exists x (E[\{\co{cat}(x)\}]: E'[x])]]$ \hfill (second conjunct)
\z 

When this combines with the verb, we arrive at the metaformula in \refer{hund-katze-schlaf-lf}.

\ea \label{hund-katze-schlaf-lf}
$\exc{C} [\inc{\co{sleep}(x)},$ \hfill (verb)\\
%$H[F[\co{dog}(x),x] \land G[\co{cat}(x),x],$ \hfill (coordination particle)\\
$H[F[x] \land G[x],$ \hfill (coordination particle)\\
$A[\exists x (B[\co{dog}(x)]: B'[x])],$ \hfill (first conjunct)\\
$D[\exists x (E[\co{cat}(x)]: E'[x])]]]$ \hfill (second conjunct)
\z 

There are two pluggings that satisfy the constraints expressed in the metaformula in \refer{hund-katze-schlaf-lf}. Let us focus on the variant in \refer{hund-katze-schlaf-plug} first. 

\vbox{
\ea \label{hund-katze-schlaf-plug}
Plugging for \refer{hund-katze-schlaf}:
\begin{xlist}
\ex first conjunct:\label{hund-katze-schlaf-plug-conj1}\\
$A = \exists x (\co{dog}(x) : \co{sleep}(x))$\\
$B = \co{dog}(x)$ \qquad $B' = \co{sleep}(x)$
\ex second conjunct:\label{hund-katze-schlaf-plug-conj2}\\
$D = \exists x (\co{cat}(x) : \co{sleep}(x))$\\
$E = \co{cat}(x)$ \qquad $E' = \co{sleep}(x)$
\ex conjunction:\\
$F = A = \exists x (\co{dog}(x) : \co{sleep}(x))$\\
$G = D = \exists x (\co{cat}(x) : \co{sleep}(x))$\\
$H = (\exists x (\co{dog}(x) : \co{sleep}(x)) \land \exists x (\co{cat}(x) : \co{sleep}(x)))$
\item overall sentence:\\
$C = H = (\exists x (\co{dog}(x) : \co{sleep}(x)) \land \exists x (\co{cat}(x) : \co{sleep}(x)))$
\end{xlist}
\z 

}

This plugging is exactly the intended, bi-propositional semantic representation that should be associated with sentence \refer{hund-katze-schlaf}. 



%This plugging leads to the bi-propositional semantic representation for sentence \refer{hund-katze-schlaf}. 
An important aspect of this plugging is that the same formula, $\co{sleep}(x)$ occurs in both the scope of the first and the scope of the second conjunct ($B'$ and $E'$ respectively). This might be a surprising result but, again, it follows directly from our constraint-based view on semantic combinatorics: the verb constrains the overall logical form in such a way that it must contain the formula $\co{sleep}(x)$, but it does not limit the number of occurrences of this formula to exactly one.%
\footnote{See \citet{Sailer:05.hpsg} for a use of the same technique for some non-standard cases of idiom modification and \citet{Bonami:Godard:07} for an application in an analysis of evaluative adverbs.}

As mentioned earlier, there is a second plugging for sentence \refer{hund-katze-schlaf}. It is like the first one for the metavariables $A$, $B$, $D$, and $E$. The diverging values for the other metavariables are given in \refer{hund-katze-schlaf-plug2}.

\vbox{
\ea 
Alternative plugging for \refer{hund-katze-schlaf}: \label{hund-katze-schlaf-plug2}
\begin{xlist}
\ex first conjunct: see \refer{hund-katze-schlaf-plug-conj1}
\ex second conjunct: see \refer{hund-katze-schlaf-plug-conj2}
\ex conjunction:\\
$F = D = \exists x (\co{cat}(x) : \co{sleep}(x))$\\
$G = A = \exists x (\co{dog}(x) : \co{sleep}(x))$\\
$H = (\exists x (\co{cat}(x) : \co{sleep}(x)) \land \exists x (\co{dog}(x) : \co{sleep}(x)))$
\ex overall sentence:\\
$C = H = (\exists x (\co{cat}(x) : \co{sleep}(x)) \land \exists x (\co{dog}(x) : \co{sleep}(x)))$
\end{xlist}
\z 

}

The difference between the two pluggings is just in the order in which the two conjuncts occur. While truth-conditionally equivalent, the order in the semantic representation should reflect the syntactic order.%
\footnote{This is particularly relevant when using a dynamic semantic representation language such as the one of \emph{Discourse Respresentation Theory} \citep{Kamp:Reyle:93} or \emph{Dynamic Predicate Logic} \citep{Groenendijk:Stokhof:91}.}

The reason for the existence of the second plugging is the fact that the lexical entry of the conjunction particle only mentions the discourse referent markers of the two conjuncts, which are constrained to be identical. Therefore, there is nothing connecting the syntactic order of the conjuncts to their order in the semantic representation.%
\footnote{Using the ``main'' content, mentioned in footnote~\ref{fn-main} would allow us to establish this connection, as these would be \co{dog} and \co{cat} for the two conjuncts, respectively. This solution is, however, not general enough, as it would not solve the problem illustrated with example \refer{ex-QANQAN} below.}

We will first propose a constraint to eliminate the plugging in 
\refer{hund-katze-schlaf-plug2} and then consider additional arguments in favor of our analysis. 
We introduce the 
\emph{Conjunct Integrity Constraint} (CIC)
%\emph{Conjunction Confusion Blocking} constraint (CCB) 
in \refer{CCB}, a constraint that will connect the semantic contribution of the conjuncts to their syntactic position in the conjunction.

% \kommentar{%Formulierung aus den Folien:
% \ea Conjunct Confusion Blocking (CCB)%\label{CBB}\\
% If the \textsc{dr} value of a conjunction with \textsc{exc} $\kappa_1 \land \kappa_2$ and its conjuncts is identical, then every element of the first conjunct's \textsc{parts} list must be in $\kappa_1$ and every element of the second conjunct's \textsc{parts} list must be in $\kappa_2$.
% \z 
% }

\vbox{
\ea \label{CCB}
%Conjunct Confusion Blocking (CCB)\\
Conjunct Integrity Constraint (CIC)\\
If the discourse referent marker of a conjunction with internal content $\kappa_1 \land \kappa_2$ and those of its conjunct daughters are identical, then every element contributed within the first conjunct daughter must be in $\kappa_1$ and every element contributed within the second conjunct daughter must be in $\kappa_2$.
\z 

}

The effect of the \CCB{} is that all elements contributed by the NP \bspT{ein Hund}{a dog} in \refer{hund-katze-schlaf} must be in the first semantic conjunct and those contributed by \bspT{eine Katze}{a cat} in the second conjunct. This makes the plugging in \refer{hund-katze-schlaf-plug} the only possible interpretation of the metavariables in the underspecified representation.

This constraint has additional important effects. 
Consider example \refer{ex-QANQAN}, in which we use the same head noun in the two conjuncts but have different adjectives and determiners. 
Below the example, we indicate two potential readings. 
Both readings respect the lexical and structural constraints of LRS, but the second reading violates the \CCB.

%MUSS JETZT NOCH AUS DEN FOLIEN REIN UND ETWAS HÜBSCH DARGESTELLT WERDEN:

\ea \label{ex-QANQAN}
{}[Every big dog and some 
small dog] ran through the yard.%\\[1ex]
\begin{xlist}
\ex \CCB{} conform reading:\label{ex-CIC-ok}\\ 
$\forall x ((\co{dog}(x) \land \co{big}(x)) : \co{run}(x)) 
\land 
\exists x ((\co{dog}(x) \land \co{small}(x)): \co{run}(x))
$
\ex  \CCB{} non-conform reading:\label{ex-CIC-ko}\\ 
\# $
\forall x ((\co{dog}(x) \land \co{small}(x)) : \co{run}(x))
\land 
\exists x ((\co{dog}(x) \land \co{big}(x)) :  \co{run}(x))
$
\end{xlist}
\z 

In \refer{ex-CIC-ko}, the contributions of the adjectives occur in the wrong conjuncts.
Because of the different determiners, this actually leads to a truth-conditional difference between the two readings. The \CCB{} will rule out \refer{ex-CIC-ko}: as the constant \co{big} is contributed within the first syntactic conjunct, it must occur in the first semantic conjunct, and analogously for \co{small}.

\medskip%
A natural objection to the \CCB{} would be that the problem it is supposed to solve is an artifact of the decision to have identical discourse referent markers for all conjuncts and the overall conjunction in the bi-propositional analysis. 
Our analysis might be perceived as counter-intuitive if one associates the discourse referent marker directly with the entity in the world that a conjunct refers to. After all, the conjoined noun phrases do not refer to the same entity -- even if one pursues a referential approach to quantification as in \citet{Luecking:Ginzburg:19}.
Our examples show that the variable $x$ in the semantic representations in \refer{hund-katze-schlaf} and \refer{ex-QANQAN} is bound by two different quantifiers within the formul\ae. 
Consequently, the variable $x$ only has bound occurrences and its occurrences in one conjunct are independent of those in the other conjunct. 
A referential identity is not implied semantically. 

The use of identical discourse referent markers has two important advantages: 
First, there is a uniform, surface-oriented syntactic analysis for sentences with conjoined noun phrases, i.e., both the mono-propositional and the bi-pro\-po\-si\-tional analysis are treated the same. 
Second, the ordinary semantic combinatorics and the ordinary linking mechanism apply when the conjoined NPs combine with the verb.% 
\footnote{In Section~\ref{Sec-Anaphor}, we will consider cases of anaphoric relations across conjuncts, which seem problematic for this assumption.}
%\marginpar{MS: Verweise auf Section \ref{Sec-Anaphor}}

\bigskip%
Just as we saw with the interpretation strategies for sentences with multiple n-words, we can -- and in fact need to -- impose constraints on the possible pluggings of bi-propositional conjunction. 
The Conjunction Integrity Constraint
%Conjunction Confusion Blocking constraint 
makes it possible to derive a bi-propositional reading from a mono-clausal, surface-oriented syntactic analysis and the ordinary argument-identification, i.e.\@ linking, mechanism of LRS.


% \kommentar{
% Besides this technical point, there are also some interesting empirical consequences of our analysis. \marginpar{gibt es da noch eine Motivation?}

% MIR FALLEN NUR GERADE KEINE EIN!!!
% }



\subsubsection{Implicit material: Equality up-to constraints}
\label{Sec-EqualityUpto}
%\bigskip%
We need to consider not only how the semantic contributions of the individual conjuncts are integrated, but also how these contributions interact with material outside the conjunction. 
As shown in example \refer{ex-QANQAN}, each of the conjoined quantifiers takes scope over the semantic contribution of the verb. 
However, we have not looked at a situation yet in which something takes scope over the conjunction. 
A simple example of this case is given in \refer{ex-notAND-bi}.

\vbox{
\ea %Alex must feed a dog and a cat.
\label{ex-notAND-bi}
Alex might eat a salad and a dessert.
%$\exists x (\co{dog}(x): \co{own}(\co{alex},x)) \land \exists x (\co{cat}(x): \co{own}(\co{alex},x))$
\begin{xlist}
\ex \label{ex-part-bi}Partially bi-propositional reading:\\
%$\Box (\exists x (\co{dog}(x): \co{feed}(\co{alex},x)) \land \exists x (\co{cat}(x): \co{feed}(\co{alex},x)))$\\
$\Diamond (\exists x (\co{salad}(x): \co{eat}(\co{alex},x)) \land \exists x (\co{dessert}(x): \co{eat}(\co{alex},x)))$
\ex  \label{ex-full-bi}Fully bi-propositional reading:\\
%$\Box (\exists x (\co{dog}(x): \co{feed}(\co{alex},x)))$\\ \quad $\land \Box (\exists x (\co{cat}(x): \co{feed}(\co{alex},x)))$
$\Diamond (\exists x (\co{salad}(x): \co{eat}(\co{alex},x)))$
%\\ \quad 
$\land \Diamond (\exists x (\co{dessert}(x): \co{eat}(\co{alex},x)))$
\end{xlist}
\z 
}

%\marginpar{matching funding idea}

Below the example, we indicate two potential bi-propositional readings. In the \emph{partially bi-propositional reading} in \refer{ex-part-bi}, the modal operator, $\Diamond$, takes scope over the entire representation of the rest of the conjunction. 
We call it ``partially bi-propositional'', because the modal operator is the highest operator in the representation of the sentence, but the two conjuncts still represent the semantics of propositions related to the clause, not only to the material from the overtly conjoined noun phrases.
The second reading is fully bi-propositional: the conjunction is the highest operator and the modal appears in both conjuncts. 

The partially bi-propositional reading can be derived easily, without any new constraints. 
It is the fully bi-propositional reading that poses a challenge: 
since the word \bsp{might} occurs only once in the sentence, the modal operator $\Diamond$ 
is contributed just once. 
However, the two occurrences of the operator $\Diamond$ have different formul\ae{} in their scope.
%do not take identical scope. 
The first occurrence includes the predicate \co{salad} in its scope, the second the predicate \co{dessert}.

\citet{Niehren:al:97} and \citet{Pinkal:99} introduce \emph{equality up-to constraints} for cases of ellipsis as in \refer{ex-Pinkal}.%
\footnote{While \citet{Pinkal:99} writes \bsp{equality upto}, we adopt the hyphenated version used in \citet{Niehren:al:97}.
} %
Such constraints capture the observation that whatever the relative scope of the two quantifiers in the first part of the sentence,  will also be the relative scoping in the representation of the elided part. 

\ea \label{ex-Pinkal}
Two European languages are spoken by every linguist, and two Asian languages are, too.
\z 

The basic idea is to say that an elliptic construction specifies that the two conjuncts have the same semantic representation with the only difference that the occurrence of the translation of \bsp{two European languages} in the first conjunct will be replaced with the translation of \bsp{two Asian languages} in the second conjunct. 
Pinkal's notation is ``$X/U \sim Y/V$'', which stands for: the formula $Y$ is just like $X$ except for containing the subformula $V$ where $X$ has the subformula $U$. The characterization shows that an equality up-to constraint is a \emph{resource multiplier} since, of course, all subexpressions of $X$ and $Y$ that contain $U$ and $V$ as subparts, respectively, are not identical.

\ea
The LRS-version of equality up-to:\\
$X/U \sim Y/V$ is a contribution constraint saying that:\\
for every expression $X'$ such that $X[X'[U]]$, there is a contribution constraint requiring the occurrence of an expression $Y'$,  $Y[Y'[V]]$,  which is just like $X'$ but having $V$ as a subexpression where $X'$ has $U$.
\z 

Note that $X/U\sim Y/V$ is not symmetric: it adds contribution constraints to $Y$, but does not add any component to $X$. 
This is intended as $Y$ represents the part that is not overtly present in the sentence.

We use an equality up-to constraint in the lexical entry for the bi-propositional conjunction particle, 
shown in the revised lexical semantic specification in \refer{le-biprop-and-Pinkal}. 
In this lexical sign, we have augmented the entry from \refer{le-und1} with an equality up-to constraint requiring that there be some subexpression $U$ of the first conjunct and some subexpression $V$ of the second conjunct such that the two conjuncts are equal up to the difference between $U$ and $V$.

\vbox{
%\marginpar{Haben wir main hier überhaupt benutzt?}
\ea Lexical specification of the conjunction particle (revised)\label{le-biprop-and-Pinkal}
\label{le-und1-Pinkal}
\\
\bsp{and}: $\inc{(F[\dr{x}] \land G[\dr{x}])}$ and $F/U\sim G/V$\\
where $U$ and $V$ are such that
%the first conjunct's main content is a component of $U$ and the second conjunct's main content is a component of $V$, and
$F[U[x]]$ and $G[V[x]]$. %and $F/U\sim G/V$
\z 

}

If we look at the two readings of \refer{ex-notAND-bi}, we find the following contribution constraints for the conjunction particle. 

\vbox{
\ea \label{constr-notAND-bi}
\begin{xlist}
\ex Partially bi-propositional reading: 
\bsp{and}: $F[x] \land G[x]$
\ex Fully bi-propositional reading: 
\bsp{and}: $F[x] \land G[x, \Diamond(V)]$
\end{xlist}
\z 

}

As shown in \refer{constr-notAND-bi}, in the case of the partially bi-propositional reading, we have a situation in which $F=U$ and $G=V$. Consequently, no additional contribution constraints are added by the equality up-to constraint. 

For the fully bi-propositional reading, the modal operator is added as having scope over both conjuncts separately. 
The two conjuncts are equal with respect to the implicitly added operators. They differ, however, with respect to the rest in that 
the first conjunct contains the expression $\exists x (\co{salad}(x) : \co{eat}(\co{alex},x))$ as the scope of the modal operator and the second conjunct has the expression $\exists x (\co{dessert}(x) : \co{eat}(\co{alex},x))$ in the parallel position in the second conjunct.


% \kommentar{%MS: raus 24.6.
% What is missing for our analysis, is a principle that specifies further embedding constraints when the two conjuncts are combined with the conjunction particle. This is done in the \emph{Conjunction Parallelism Constraint} (CPC) in \refer{CPC}.

% \vbox{
% \ea 
% Conjunction Parallelism Constraint (CPC, preliminary version)\label{CPC}\\
% In a conjunction phrase
% with an internal content of the form $F \land G$,\\
% %there exist expressions $U$ and $V$ contributed by the first and the second conjunct respectively, such that $F$ is just like $G$ with $U$ instead of $V$.
% \begin{itemize}
% %\item the external content of the first conjunct is in $F$, and the external content of the second conjunct in $G$,

% \item there exist expressions $U$ and $V$ contributed by the first and the second conjunct respectively, such that $F$ is just like $G$ with $U$ instead of $V$,

% %\item 
% %for each $H$ contributed by one conjunct, $H$ occurs inside this conjunct.
% %%TO BE ADDED LATER FOR ATB: 
% %%or $H$ is contributed by the other conjunct as well and $H[F\land G]$. 

% \item for each $H$ contributed in a conjunct, $H$ must not occur in a conjunct in which it is not contributed nor have scope over the external content of the conjunction.%
% %nor have scope over a conjunct in which it is not contributed.
% \footnote{
% In this paper, look at two types of conjunction. For bi-propositional conjunction, $F\land G$ will also be the external content. For mono-propositional conjunction, the external content will be $\exists z (A:B)$, see \refer{le-und2} below.
% }

% \end{itemize}
% \z 

% }
% }%kommentar

\medskip%
It is important to note that the readings discussed in this subsection do not violate the CIC from \refer{CCB}. 
In each reading, the semantic material contributed in the first conjunct daughter appears within the first conjunct, and the material from the second conjunct daughter within the second conjunct.

What is missing for our analysis is a principle that specifies further embedding constraints when the two conjuncts are combined with the conjunction particle. This is done in the \emph{Conjunction Parallelism Constraint} (CPC) in \refer{CPC}.

\vbox{
\ea 
Conjunction Parallelism Constraint (CPC)\label{CPC}\\
In a conjunction phrase
with an internal content of the form $F \land G$,\\
for each expression $H$ which occurs only in one conjunct,
%there exist expressions $U$ and $V$ contributed by the first and the second conjunct respectively, such that $F$ is just like $G$ with $U$ instead of $V$.
\begin{enumerate}
%\item the external content of the first conjunct is in $F$, and the external content of the second conjunct in $G$,

\item %verson 24.6.19
 $H$ is contributed by the conjunct daughter linked to that conjunct, or 
 \item 
there is an expression $H'$, where either $H$ or $H'$ is contributed by the conjunction particle, such that for some $J, J'$,  $H/J \sim H'/J'$.
%version until 23.6.19: there exist expressions $U$ and $V$ contributed by the first and the second conjunct respectively, such that $F$ is just like $G$ with $U$ instead of $V$,
%\item 
%for each $H$ contributed by one conjunct, $H$ occurs inside this conjunct.
%%TO BE ADDED LATER FOR ATB: 
%%or $H$ is contributed by the other conjunct as well and $H[F\land G]$. 

%\item for each expression $H$ contributed in a conjunct daughter, $H$ must not occur in a conjunct in which it is not contributed nor have scope over the external content of the conjunction.%
%nor have scope over a conjunct in which it is not contributed.
%\footnote{In this paper, look at two types of conjunction. For bi-propositional conjunction, $F\land G$ will also be the external content. For mono-propositional conjunction, the external content will be $\exists z (A:B)$, see \refer{le-und2} below.}

\end{enumerate}
\z 
}


The CPC encodes the observation that the conjuncts may only differ with respect to material that has been explicitly contributed by the conjunct daughters or that embeds such material.
The first clause of this constraint requires that all contributions of a conjunct daughter actually occur in the conjunct to which this daughter is linked.%
\footnote{In its version in \refer{CPC}, the first clause of the CPC covers the effect of the \CCB{} in \refer{CCB}. 
However, we will see later that 
%the second clause of the \SemATB{} is too strong and that, consequently 
the \CCB{} still has its place in our analysis of conjunction. 
}
%For implicit semantic material there are two options, expressed in the two clauses of the CPC.
%Firstly, it can be fully identical in both conjuncts. 
%This is the case in the partially bi-propositional reading of \refer{ex-notAND-bi}, where both  
Implicit material is material that is contributed by the conjunction particle.
Such material can be equal up-to the material contributed in the conjunct daughters -- as the modal operator in the fully bi-propositional reading of \refer{ex-notAND-bi}. 
%The CPC requires that the equality up-to relation between the two conjuncts actually holds with respect to an element contributed by the words in the conjuncts.


%$\co{MUST}(\co{alex}, \exists x (\co{dog}(x): \co{feed}(\co{alex},x)))$\\ \quad $\land \co{MUST}(\co{alex}, \exists x (\co{cat}(x): \co{feed}(\co{alex},x)))$



\bigskip%
% \kommentar{%raus: nach oben kopiert 
% Just as we saw with the interpretation strategies for sentences with multiple n-words, we can -- and in fact need to -- impose some constraints on the possible pluggings of bi-propositional conjunction. 
% The Conjunction Integrity Constraint
% %Conjunction Confusion Blocking constraint 
% makes it possible to derive a bi-propositional reading from a mono-clausal, surface-oriented syntactic analysis and the ordinary argument-identification mechanism of LRS.
% }

%HIER NOCH WAS ZUSAMMENFASSENDES ZUM EQUALITY UP-TO-TEIL SCHREIBEN
The concept of equality up-to constraints has not been implemented in LRS so far.%
%
\footnote{In a recent talk, \citet{Park:al:18} propose an LRS-analysis of gapping. We repeat their running example in \refer{gapping}, adapting the semantic representations to our notation.

\ea \label{gapping}
John can't live in LA and Mary in New York.
\begin{xlist}
\ex Distributive-scope reading:
$\lnot \Diamond (\co{live-in}(\co{john},\co{la}))
\land \lnot \Diamond (\co{live-in}(\co{mary},\co{ny}))$
\ex Wide-scope reading: 
$\lnot \Diamond (\co{live-in}(\co{john},\co{la}) 
\land \co{live-in}(\co{mary},\co{ny}))$
\end{xlist}
\z 

In the distributive-scope reading, the two occurrences of the negation and the modal operator have distinct formul\ae{} in their scope in the two conjuncts. However, apart from the material contributed in the gapped clause, \co{mary} and \co{ny}, their scope is identical. In the spirit of the present paper, gapping would be seen as another application of equality up-to constraints. \citet{Park:al:18} do not elaborate on the equality up-to aspect of their analysis.

}
The version we presented here tries to capture the original intuitions formulated in \citet{Pinkal:99}. 
As mentioned above, Pinkal introduces this type of constraint for elliptical constructions as in \refer{ex-Pinkal}, but we use them for simple NP-coordinations. 
It should also be noted that the equality up-to contribution constraint in \refer{le-biprop-and-Pinkal} is different in nature from the previous contribution constraints:
instead of specifying a concrete contribution, it is an abstract characterization of what is to be contributed. 
While this is different from what we have seen in this paper so far, it is not completely new for LRS: a similar kind of semantic underspecification in the lexicon is used 
in \citet{Lahm:18} for the optional presence of pluralization operators in the semantics of verbs. 

\bigskip%
In this section, we showed that LRS allows for various types of one-to-many correspondences at the syntax-semantics interface such as 
scope ambiguity, split reading, semantic concord, and semantically implicit material of two types.
We will make use of all of them in our analysis of CNNP.

% \kommentar{
% (i) For SN readings with multiple n-words, we were in a constellation in which more than one word required a negation, but there was, eventually, only one negation in the overall semantic representation. 
% (ii) In the conjunction case, the occurrence of the verb's semantics is only imposed once, but it it is present more than once in the resulting (linear) representation. 
% However, we did not need to introduce any additional techniques to derive these effects. 
% The only new technique that we introduced in this section are equality up-to constraints, which are an instance of underspecification in the lexicon.
% %They are consistent with  the constraint-based view on semantic contributions as implemented in our notion of a plugging.
% }

%\bigskip%
%In this section, we %presented the framework of LRS just in enough detail to be able to 
%illustrated the kinds of one-to-many correspondences that LRS can capture. 


\section{Analysis}
\label{Sec-Analysis}



\subsection{Conjunction}
\label{Sec-AnalyisConjunction}

We have already seen how we can derive a bi-propositional reading of sentences with conjoined noun phrases. In the present subsection, we will extend our analysis of conjunction to mono-propositional readings.
In \refer{KimAlexTalk-lf2}, we repeat the mono-propositional semantic representation of sentence \refer{AlexKimTalk}, which contained conjoined proper nouns.

\vbox{
\ea \label{KimAlexTalk-lf2}
Alex and Kim met. \hfill(= \refer{AlexKimTalk})
\glt $\exists z 
((\co{alex} \in \pI z
\land \co{kim} \in \pII z)
: \co{meet}(z))$
\z
% \kommentar{
% \ea \label{KimAlexTalk-lf2}
% Alex and Kim met. (= \refer{AlexKimTalk})\\
% $\exists z (
% (\co{alex} \in z
% \land 
% \co{kim} \in z
% ) :
% \co{meet}(z)
% )$
% \z 
% }
}

Our lexical specification for the coordination particle in \refer{le-und1} will not be sufficient to derive this reading, therefore we introduce a new, plural discourse referent, $z$, and define what elements need to be in $z$.%
\footnote{It is quite common to assume two readings for English \bsp{and}, one corresponding to logical conjunction -- our bi-propositional \bsp{and} -- and one to some group/plurality formation -- our mono-propositional \bsp{and}. Such an assumption can be found, for example, in \citet{Partee:Rooth:83}. More recently, \citet{Mitrovic:Sauerland:16} argue for it on the basis of typological evidence.
}

% \kommentar{%raus: MAIN nicht benötigt
% \ea \label{le-und2}
% Mono-propositional \bsp{and}:\\
% $\exc{\exists \dr{z} (
% (F[F', x  \in  z]
% \land 
% G [G', y \in z]
% ) : H[z]
% )}$\\
% where $F'$ is the internal content of the first conjunct and $x$ its discourse referent, and
% $G'$ is the internal content of the second conjunct and $y$ its discourse referent.
% \z 
% }

\vbox{
\ea \label{le-und2}
Mono-propositional \bsp{and}:\\
$\exc{\exists \drr{z} (
\inc{(F[\noc{x}  \in \pI z]
\land 
G [\noc{y} \in \pII z]
)} : H[z]
)}$\\
where $x$ is the discourse referent marker of the first conjunct, and
$y$ the discourse referent marker of the second conjunct.
\z 

}

Using this specification for the coordination particle, we can derive the mono-propositional representation in \refer{KimAlexTalk-lf2}.

The lexical specification in \refer{le-und2} is sufficient to derive all mono-propositional representations from Section~\ref{Sec-Conjunction}. We shall illustrate this with example \refer{AlexManyCall}, in which one of the conjuncts is a  quantifier, repeated as 
\refer{AlexManyCall-lf2}.

\ea \label{AlexManyCall-lf2}
Alex and many students met in the yard. \hfill(= \refer{AlexManyCall})\\
$
\exists z 
((\co{alex} \in \pI z 
\land 
(\co{Many}\, y\, (\co{student}(y): y \in \pII z)))
: \co{meet}(z))
$
\z 

The semantic constraints of the conjuncts are given in \refer{le-AlexManyCall}.

\ea \label{le-AlexManyCall}
\begin{xlist}
\ex \bsp{Alex}: $\exc{\inc{\dr{\co{alex}}}}$
\ex \bsp{many students}: $\exc{\co{Many}\, \dr{x}\, (B[\inc{\co{student}(x)}]: B'[x])}$
\end{xlist}
\z 
Together with the translation of the conjunction particle, the previous
two constraints lead to the following constraint for the entire
conjunction phrase \bsp{Alex and many students}.

\vbox{
\ea \label{le-AlexManyCall-and}
$A [ \exc{\exists} \drr{z}$\\
$(\{(F[\co{alex} \in \pI z] $ \hfill (first conjunct)\\
$\land 
G [x \in \pII z]
]
)\} $ \hfill (second conjunct)\\
$: H[z]),$ \hfill (scope of the conjunction) \\
$\co{Many}\, x\, (B[\co{student}(x) : B'[x])]$
\z 

}

The only plugging that is compatible with these constraints is given in \refer{AlexManyCall-plug}.

\ea \label{AlexManyCall-plug}
$F = \co{alex} \in \pI z$ \hfill (first conjunct)\\
$B = \co{student}(x)$ \hfill (restrictor of \co{Many})\\
$B' = x \in \pII z$ \hfill (scope of \co{Many})\\
$G = \co{Many}\, x\, (\co{student}(x) : x \in \pII z)$ \hfill (second conjunct)\\
$H = \co{meet}(z)$ \hfill (scope of the conjuction)\\
$A = \exists z ((\co{alex} \in \pI z) \land (\co{Many}\, x\, (\co{student}(x) : x \in \pII z)) :
\co{meet}(z))$
\z 

We do not need a constraint such as the \CCB{}, \refer{CCB}, for the mono-propositional conjunction because the discourse referent markers of the conjuncts and the overall conjunction are all distinct. 
Therefore, the order of the conjuncts within the semantic representation can be fixed in the lexical entry of the conjunction particle. 
Furthermore, any modifiers or determiners within a conjunct will be connected to the conjunct-specific discourse referent marker.
The CPC, \refer{CPC}, does not have an effect in the mono-propositional case either, as there is no shared, implicit material in the two conjuncts.


\subsection{Negated conjuncts}
\label{Sec-AnalysisCNNP}


All LRS techniques that we have introduced above come together in our analysis of CNNP. 
We will assume the lexical entries of n-words and coordination particles motivated in the preceding sections as well as the combinatorial constraints illustrated so far.
We will first look at the general syntactic and semantic conditions and show how we can derive the mono-propositional and the bi-propositional readings of CNNP. 
We will then explain how our analysis leads to the properties of CNNP from Section~\ref{Sec-Data} such as the availability of split readings and the disjunction effect. 

\subsubsection{Semantic Across-the-Board exception}
\label{Sec-SemATB}

%\marginpar{MS: die Sätze mit geb. pron. im 2. Konjunkt stellen ein Problem für strikten ATB dar. Wahrscheinlich kommt man aus dem ersten Konjunkt doch raus. Das macht dann auch manche Fälle von NPI-Liz im 2. Konjunkt}
We gave a brief characterization of German as a non-NC language in Section~\ref{Sec-Intro}
and its LRS analysis in Section~\ref{Sec-LRS-NC}. We accounted for the non-NC-hood of StG by assuming a \emph{Negation Faithfulness Constraint} (NFC) in \refer{neg-faith}. According to this constraint, whenever more than one daughter contributes a negation in a headed phrase, the negations have to be distinct.

Independently of our concrete assumptions about the syntax of coordination, it is uncontroversial that coordination has its own syntactic structure and should not be treated as an ordinary headed phrase. 
As the NFC only enforces negation faithfulness in headed structures, it does not have an effect in coordination structures in general, also including StG. Thus StG may show an NC-like behavior in exactly these structures.

In our semantic analysis of the mono-propositional readings of CNNP in Section~\ref{Sec-Semantics}, we provided semantic representations in which (i) the negation has wide scope over the existential quantifier contributed by the coordination particle, (ii) there is only one negation in the resulting semantic representation, and (iii) this reading  is only possible if each conjunct contains an n-word.
%This characterization can be rephrased in the following way: 
%
To enforce these three properties, we will assume a semantic analogue of the syntactic \emph{Across-the-Board} (ATB) exception to the \emph{Coordinate Structure Constraint} (CSC), the ban of syntactic movement out of a conjunct from \citet{Ross:67}. The ATB exception says that material may be moved out of a conjunct as long as it is moved out of every conjunct.% 
\footnote{
\citet{Chaves:12} shows that the CSC and its ATB exception can be reduced to a semantic requirement, using a semantic combinatorial framework similar to ours. 
In his approach, symmetric coordination is analyzed as the formation of a plural event, i.e.\@ via a conjunction analogous to the effect of our mono-propositional conjunction particle. He, then, assumes that syntactically extracted elements are obligatorily distributed over all conjuncts. 
}

We can, now, rephrase the conditions on CNNP as an ATB phenomenon: a negation from one conjunct can only have scope over the entire conjunction if all conjuncts contribute the same negation. 
%
Such a semantic ATB exception is independently motivated. For the ATB exception to make sense, we must show that there is a semantic CSC. This has been argued for in 
\cite[83]{Winter:01}, for example.
\cite[323]{Copestake:al:05} 
 provide example \refer{stay-asleep} to show that the modal adverb \bsp{probably} cannot take scope over both conjuncts if it occurs in one.

\ea \label{stay-asleep}
Sandy stayed and probably fell asleep.\\
($\not=$ Sandy probably stayed and fell asleep.)
\z

\cite[86--89]{Chaves:07} argues against the applicability of the CSC to scope. Instead, he considers conjunct-internal scope as a reading preference and allows for wide scope of individual conjuncts. He provides examples such as \refer{romeojuliet}, for which a wide-scope interpretation of the modal adverb is available even if it only occurs in one conjunct.

\ea \label{romeojuliet}
Kim probably is playing Juliet and Fred is playing Romeo.
\z 

% \kommentar{
% If we translate Chaves' example to German, this wide-scope reading is lost, see \refer{romeojuliet-de}. 

% \ea \label{romeojuliet-de}
% \gll weil Kim wahrscheinlich Julia spielt und Fred Romeo spielt.\\
% because Kim probably Juliet plays and Fred Romeo plays\\
% \glt \$ \mytrans{because, probably, Kim plays Juliet and Fred plays Romeo.}
% \z 

% }


We suspect that the adverb in \refer{romeojuliet} is treated as a parenthetical. This is confirmed by the sentence in \refer{romeojuliet2}, where it follows the finite verb. 
In this position, 
the adverb is usually phonologically integrated and, thus, has a non-parenthetical interpretation. 
The wide-scope interpretation of \bsp{probably} is not available for this sentence.%
\footnote{\citet{Chaves:07} also provides examples in which the second conjunct contains a pronoun that is interpreted as coreferential to or bound by an NP in the first conjunct. 
We will address these data in Section~\ref{Sec-Anaphor}.
%would hope to derive these data by an appropriate dynamic semantic representation language.
}


\ea
\label{romeojuliet2}
Kim is probably playing Juliet and Fred is playing Romeo.
\glt $\not=$ Probably, Kim is playing Juliet and Fred is playing Romeo.
\z 

\citet{Chaves:07} uses \emph{Minimal Recursion Semantics} \citep{Copestake:al:05}, a
 framework that does not allow two words to make identical semantic contributions. Consequently,
he cannot derive a real ATB reading, i.e., an interpretation with two syntactic occurrences of \bsp{probably} but a single interpretation.  
If we alter example \refer{stay-asleep} in such a way that there is the same adverbial in both conjuncts, we can find a reading in which there is a single ATB-interpretation of the adjunct's scope. This is illustrated in \refer{stay-asleep-ATB}. The second reading is the relevant ATB interpretation. According to our intuitions, this reading is not available if the adverb \bspT{wahrscheinlich}{probably} occurs in only one of the conjuncts.

\ea \label{stay-asleep-ATB}
\gll Sandy ist wahrscheinlich geblieben und ist wahrscheinlich eingeschlafen.\\
Sandy has probably stayed and has probably fallen.asleep\\
\glt \mytrans{Sandy probably stayed and probably fell asleep.}
\glt Reading 1: Sandy probably stayed and Sandy probably fell asleep.
\glt Reading 2: Probably, Sandy stayed and fell asleep.
\z 

% \kommentar{
% Hey Leute. Ich hab mal eine Frage. Anfang nächsten Jahres muss ich ein Referat ueber Hollister halten. Denkt ihr ich kann da mal bei dem Kundendienst nachfragen ob sie mor eventuell Werbung also Prospekte und eventuell B-Ware schicken koennten da ich selber noch keine Hollister Klamotten besitze? 
% https://www.gutefrage.net/frage/hollister-kundensienst-nach-prospekten-und-eventuell-b--ware-fragen
% 29.2.2020
% }


\bigskip%
Having given some negation-independent empirical motivation of semantic CSC with a corresponding ATB exception, we can turn to the formulation of the relevant constraints.
%The extraction mechanism of HPSG, following that of \citet{GKPS}, is built in such a way that the ATB exception follows without further stipulation.
%The \emph{Coordination Parallelism Constraint} in \refer{CPC} 
The \emph{Conjunct Integrity Constraint} in \refer{CCB}
expresses exactly the observation behind the Coordinate Structure Constraint, i.e., the insight that the material contributed within a conjunct needs to stay within this conjunct. 
What is missing so far, however, is a semantic analogue to the ATB exception. %
%\footnote{A semantic version of the ATB-exception has been proposed in \citet{Winter:01} as well.}
This is stated in the reformulation of the CIC in \refer{SemATB}.
As we will see in the discussion of individual examples, the final part of the CIC in this version will allow for CNNP. The negation contributed within one conjunct can take wide scope over the entire conjunction if and only if it is contributed within both conjuncts.

% \kommentar{%MS: raus, 25.6., Grund: jetzt Teil des CIC
% \vbox{
% \ea 
% Conjunction Parallelism Constraint with Semantic ATB exception (CPC, second version)\label{CPC-ATB}\label{SemATB}\\
% In a conjunction phrase
% with an internal content of the form $F \land G$,\\
% %there exist expressions $U$ and $V$ contributed by the first and the second conjunct respectively, such that $F$ is just like $G$ with $U$ instead of $V$.
% \begin{itemize}
% %\item the external content of the first conjunct is in $F$, and the external content of the second conjunct in $G$,

% \item %there exist expressions $U$ and $V$ contributed by the first and the second conjunct respectively, such that $F$ is just like $G$ with $U$ instead of $V$,
% For each expression $H$ which occurs only in one conjunct, $H$ is contributed by the conjunct daughter linked to that conjunct, or 
% there is an expression $H'$, where either $H$ or $H'$ is contributed by the conjunction particle, such that for some $J, J'$,  $H/J \sim H'/J'$.

% \item 
% For each $H$ contributed by one conjunct daughter, $H$ must not occur in a conjunct in which it is not contributed and may only have scope over the conjunction's external content if it is contributed by the other conjunct daughter as well.
% %$H$ occurs inside this conjunct,
% %TO BE ADDED LATER FOR ATB: 
% %alt: or $H$ is contributed by the other conjunct as well and $H[F\land G]$. 

% \end{itemize}
% \z 
% }
% }%kommetar

\vbox{
\ea 
Conjunct Integrity Constraint with Semantic ATB exception (CIC, second version)\label{SemATB}\\

%If the discourse referent marker of a conjunction with internal content $\kappa_1 \land \kappa_2$ and those of its conjunct daughters are identical, then every element contributed within the first conjunct daughter must be in $\kappa_1$ and every element contributed within the second conjunct daughter must be in $\kappa_2$.
In every coordination phrase, 
for each $H$ contributed by one conjunct daughter, $H$ must not occur in a conjunct in which it is not contributed and may only have scope over the conjunction if 
%(i) the conjuncts have distinct discourse referent markers and (ii) 
it is contributed by the other conjunct daughter as well.



%$H$ occurs inside this conjunct,
%TO BE ADDED LATER FOR ATB: 
%alt: or $H$ is contributed by the other conjunct as well and $H[F\land G]$. 

\z 
}

% \kommentar{
% This is stated in a somewhat generalized form as a \emph{Semantic Across-the-Board Constraint} (SemATB) in \refer{SemATB}.

% \kommentar{%raus: Version der Folien
% \ea Semantic ATB-constraint (SemATB):\\
% In each conjunction phrase,\\
% if an element from \textbf{one} conjunct-daughter's \attr{parts} list takes scope over the entire conjunction,\\ 
% then this element must be on \textbf{every} conjunct-daughter's \attr{parts} list.
% \z 
% }

% \ea Semantic Across-the-Board constraint (SemATB):\label{SemATB}\\
% In each coordination phrase,\\
% if an element contributed by \textbf{one} conjunct-daughter takes scope over the entire coordination,\\ 
% then this element must contributed by \textbf{every} conjunct-daughter.
% \z 


% The extraction mechanism of HPSG, following that of \citet{GKPS}, is built in such a way that the ATB exception follows without further stipulation.
% At present, we think that \SemATB{} must be formulated as an independent constraint and does not follow from other constraints.
% However, its clear parallelism to the syntactic constraints on extraction out of coordination structures gives us hope that a more general principle can be found that will comprise both the syntactic and the semantic aspects of the ATB exception.
% }

\subsubsection{A simple example}
\label{Sec-AnalyisSimpleExample}

With all constraints in place, we can now analyze a sentence with CNNP. We use a version of our running example but use a collective predicate, see \refer{ex-brief-mail-ana}, to illustrate the mono-propositional reading.

\ea \label{ex-brief-mail-ana}
\gll Alex vergleicht [\underline{\NE{keine} Briefe} und \uwave{\NE{keine} Mails}].\\
Alex compares {\hphantom{[}no letters} and {no {e-mail messages}}\\
\z 

The semantic constraints of the two conjuncts are given in \refer{ex-brief-mail-conj}. 
The noun phrases are interpreted exactly in the way illustrated for n-constituents in Section~\ref{Sec-LRS-NC}.

\ea \label{ex-brief-mail-conj}
\begin{xlist}
\ex
\bsp{keine Briefe}: $\lnot A[\exc\exists x (B[\{\co{letter}(x)\}]: B'[x])]$
\ex
\bsp{keine Mails}: $\lnot D[\exc\exists y (E[\{\co{mess}(y)\}]: E'[y])]$
\end{xlist}
\z 

The two conjuncts combine with the mono-propositional coordination particle, which leads to the following overall constraint for the conjunction.

\ea \label{ex-brief-mail-coord}
\begin{xlist}
\ex
\bsp{und}: $\exc \exists \drr{z} (\inc{(F[\noc{x} \in \pI z] \land G[\noc{y} \in \pII z])} : H[z])$\\
\ex \bsp{keine Briefe und keine Mails}:\\
\qquad $I[
\exc \exists \drr{z} (\inc{(F[x \in \pI z] \land G[y \in \pII z])} : H[z]),$
\qquad $\lnot A[\exists x (B[\co{letter}(x)]: B'[x])],$\\
\qquad $\lnot D[\exists y (E[\co{mess}(y)]: E'[y])]
]$
\end{xlist}
\z 

When we add the verb and the subject, we arrive at the  constraint in \refer{ex-brief-mail-lf}.

\ea \label{ex-brief-mail-lf}
$\exc J[I[
\exists z ((F[\noc{x} \in \pI z] \land G[\noc{y} \in \pII z]) : H[z]),$
\hfill (coordination particle)\\
$\lnot A[\exists x (B[\co{letter}(x)]: B'[x])],$
\hfill (first conjunct)\\
$\lnot D[\exists y (E[\co{mess}(y)]: E'[y])],$
\hfill (second conjunct)\\
$\inc{\co{compare}(\noc{\co{alex}},\noc{z})},$ \hfill (verb)\\
$\co{alex} 
]]$ \hfill (subject)
\z 

The intended mono-propositional reading can be derived with the following plugging.

\ea \label{ex-brief-mail-plug}
$B = \co{letter}(x)$ \qquad$B' = x \in \pI z$\\
$E = \co{mess}(y)$ \qquad $E' = y \in \pII z$\\
$F = \exists x (\co{letter}(x):x \in \pI z)$\\
$G = \exists y (\co{mess}(y) : y \in \pII z)$\\
$A = D = \exists z ((\exists x (\co{letter}(x):x \in z) \land 
\exists y (\co{mess}(y) : y \in z))$\\
\hspace*{\fill}$: \co{compare}(\co{alex},z))$\\
$J = I = \lnot \exists z ((\exists x (\co{letter}(x):x \in \pI z) \land 
\exists y (\co{mess}(y) : y \in \pII z))$\\
\hspace*{\fill}$ : \co{compare}(\co{alex},z))$
\z 

In this plugging, the two conjuncts both introduce a contribution constraint for a negation, $\lnot A$ and $\lnot D$ respectively. 
Eventually, we end up with just a single negation, as the plugging assigns the same formula to both $A$ and $D$. 
The first conjunct constraints the negation $\lnot A$ to take scope over the existential quantification over letters, the second conjunct constrains $\lnot D$ to scope over the existential quantification over e-mail messages. 
By having wide scope over both conjuncts, both these requirements can be satisfied by a single negation.

The plugging in \refer{ex-brief-mail-plug} also satisfies the \SemATB: 
while there is a semantic operator contributed by one conjunct that takes scope over the entire conjunction, this very operator is contributed by all conjuncts.

% \kommentar{%Version aus den Folien
% \begin{tabular}{lll}
% \bsp{keine Briefe}: & $\underline{\lnot} \alpha, \idx{A} \underline{\exists x_1 (\co{letter}(x_1): \alpha')}$ & $\idx{A} \compo \alpha$\\[0.5ex]
% \bsp{keine Mails}: & $\uwave{\lnot} \beta, \idx{B} \uwave{\exists x_2 (\co{message}(x_2): \beta')}$
% & $\idx{B} \compo \beta$\\
% \bsp{und}: & $\idx{1} (x_1\in X), 
% \idx{2} (x_2 \in X),
% \idx{K}(\kappa_1 \land \kappa_2),
% \exists X (\idx{K} : \sigma)$ &
% %$\idx{K} = \rho$,\\
% $\idx{1} \compo \kappa_1$,\\
% && $\idx{2} \compo \kappa_2$\\
% %NP \bsp{und} NP: & & 
% \end{tabular}
% }

% \kommentar{%Version aus den Folien
% Effect of SemATB:
% \begin{itemize}
% \item $\idx{A}$ must be in $\kappa_1$
% \item $\idx{B}$ must be in $\kappa_2$
% \item If $\lnot \alpha$ takes scope over the entire conjunction, 
% then $\alpha = \beta$.
% \end{itemize}

% \ea 
% $\uwave{\underline{\lnot}} \exists X ((
% \underline{\exists x_1 (\co{letter}(x_1) : x_1 \in X)}
% \land 
% \uwave{\exists x_2 (\co{message}(x_2) : x_2 \in X)}$\\
% \hspace*{\fill}$: \co{answer}(\co{alex},X))
% $
% \z 
% }

When we look at the constraints gathered in \refer{ex-brief-mail-lf}, we could imagine a plugging in which both negation contributions have wide scope over the coordination but are not identical. This would result in the semantic representation in \refer{ex-brief-mail-plugDN}. 
This semantic representation violates the \SemATB, because the negation operators differ, i.e., this is not an ATB exception.

\vbox{
\ea \label{ex-brief-mail-plugDN}
$\lnot\lnot \exists z ((\exists x (\co{letter}(x):x \in \pI z) \land \exists y (\co{mess}(y) : y \in \pII z))$\\
\hspace*{\fill}$: \co{compare}(\co{alex},z))$
\z 

}



Another case that is excluded by \SemATB{} is given in \refer{ex-brief-mail-plug3}. Here, the negation contributed by the first conjunct takes wide scope. The one contributed by the second conjunct, however, takes conjunct-internal scope. Even though this semantic representation satisfies the constraints collected in \refer{ex-brief-mail-lf}, it violates \SemATB.

\ea \label{ex-brief-mail-plug3}
$\lnot \exists z ((\exists x (\co{letter}(x):x \in \pI z) \land 
\lnot \exists y (\co{mess}(y) : y \in \pII z))$\\
\hspace*{\fill}$: \co{compare}(\co{alex},z))$
\z 

%\bigskip%
We should also consider the derivation of the bi-propositional reading of a sentence with CNNP.
%like \refer{ex-brief-mail-ana}. 
To ensure that we have a bi-propositional reading, we replace the collective verb in \refer{ex-brief-mail-ana} with a non-collective one, \bspT{beantworten}{answer}.
For this reading, the syntactic analysis is the same, but we need to choose a different interpretation of the conjunction particle, namely the one in 
%\refer{le-und1}
\refer{le-und1-Pinkal}.
This choice has the effect that the discourse referent markers in both conjuncts and for the overall conjunction are identical. 
Since our example sentence does not contain semantic material that will take scope over the conjuncts, the equality up-to constraint does not add additional contribution constraints and we can ignore it.


\ea \label{ex-brief-mail-conj2}
\begin{xlist}
\ex
\bsp{und}: $\inc{(F[\dr{x}] \land G[\dr{x}])}$ and $F/U\sim G/V$
\ex
\bsp{keine Briefe}: $\lnot A[\exc\exists x (B[\{\co{letter}(x)\}]: B'[x])]$
\ex 
\bsp{keine Mails}: $\lnot D[\exc\exists x (E[\{\co{mess}(x)\}]: E'[x])]$
\end{xlist}
\z 

When we combine these conjuncts with the coordination particle, we get the following overall constraint for the conjunction.

\vbox{
\ea \label{ex-brief-mail-coord2}
%\bsp{und}: $F[x] \land G[x]$ and $F/U\sim G/V$\\
\bsp{keine Briefe und keine Mails}:\\
\qquad $I[
(F[x] \land G[x]),$
\hfill (coordination particle)\\
\qquad $\lnot A[\exists x (B[\co{letter}(x)]: B'[x])],$
\hfill (first conjunct)\\
\qquad $\lnot D[\exists x (E[\co{mess}(x)]: E'[x])]
]$
\hfill (second conjunct)
\z 

}

The \CCB{} in its first version in \refer{CCB} 
allows us to constrain this further: we know that all contribution constraints of the first conjunct must be within $F$ and all those of the second conjunct within $G$. We can incorporate this into the constraint above, which results in the contraint in \refer{ex-brief-mail-coord3}.

\vbox{
\ea \label{ex-brief-mail-coord3}
\bsp{keine Briefe und keine Mails}:\\
 $I[
\{(F[\dr{x},\lnot A[\exists x (B[\co{letter}(x)]: B'[x])]]$\\
\hspace*{1.5em} $\land G[\dr{x}, \lnot D[\exists x (E[\co{mess}(x)]: E'[x])] ])\}] $ \\
\z  

}

With the verb and the subject, we arrive at the overall constraint in \refer{ex-brief-mail-lf2}.

\vbox{
\ea \label{ex-brief-mail-lf2}
$\exc J[I[
F[x,\lnot A[\exists x (B[\co{letter}(x)]: B'[x])]]$
\hfill (first conjunct)\\
\hspace*{1.5em}$\land G[x, \lnot D[\exists x (E[\co{mess}(x)]: E'[x])] ]),$ \hfill (second conjunct)\\
$\inc{\co{answer}(\noc{\co{alex}},\noc{x})},$ \hfill (verb)\\
$\co{alex} 
]]$ \hfill (subject)
\z 

}

In this constraint, each conjunct must contain a negation of its own. Therefore, 
 the two negations cannot be identical, i.e., there is no plugging in which $A=D$. Instead, we get a plugging that leads to the reading in \refer{ex-brief-mail-bi}.

\ea \label{ex-brief-mail-bi}
$\lnot \exists x (\co{letter}(x) : \co{answer}(\co{alex},x))
\land \lnot \exists x (\co{mess}(x) : \co{answer}(\co{alex},x))$
\z 

For this reading, we must use the semantic material contributed by the subject and the verb in both conjuncts, i.e., both conjuncts have the same formula as their scope. 
In other words: $B' = E' = \co{answer}(\co{alex},x)$. We had seen in Section~\ref{Sec-ImplicitSemMat} that this is possible and necessary for phenomena in which semantic material is used more often than its contributing syntactic elements occur in the structure.

% \kommentar{%MS: raus, 25.6.19, CIC und SemATB jetzt anders
% It is important to note that \CCB{} prevents the negations contributed by the individual conjuncts from taking wide scope over the entire conjunction. 
% I.e., there is no reading as in \refer{ex-brief-mail-biWide}.
% %, even though such a reading would be compatible with the ATB-exception as included in the \SemATB{} in \refer{CPC-ATB}.

% \ea \label{ex-brief-mail-biWide}
% $\lnot (\exists x (\co{letter}(x) : \co{answer}(\co{alex},x))
% \land \exists x (\co{mess}(x) : \co{answer}(\co{alex},x)))$
% \z 

% This representation would be compatible with the \SemATB{} in \refer{SemATB}, but it violates the \CCB: since the negation is contributed within the syntactic conjuncts, it must be part of the semantic representation of the conjuncts.
% }

\medskip%
If we use the refined version of the \CCB{} in \refer{SemATB}, 
a semantic ATB exception is allowed in principle.
This licenses a second potential bi-propositional reading, the one given in \refer{ex-brief-mail-biWide}.
In this reading, the negations contributed by the two conjuncts are assumed to be identical and to take wide scope over the conjunction. This corresponds to a semantic ATB exception for the bi-propositional coordination. As such, it is compatible with the \SemATB{} from \refer{SemATB}. 
As this is not a possible reading of the sentence, we will show how it can be blocked.

\ea \label{ex-brief-mail-biWide}
$\lnot (\exists x (\co{letter}(x) : \co{answer}(\co{alex},x))
\land \exists x (\co{mess}(x) : \co{answer}(\co{alex},x)))$
\z 

The examples with \bsp{probably} in \refer{stay-asleep-ATB} showed that we do not want to exclude an ATB exception for a bi-propositional coordination in general. It, thus, seems that unavailability of the reading derives from the properties of the n-words.

N-words are special in that they express indefinites that are in the scope of a sentential negation. 
The basic intuition of our explanation is that the negation contributed by an n-word is confined to the clause containing the n-word. 
We can define a semantically negative clause in English as a clause in which the internal content of the highest verb of the sentence is in the scope of negation within its external content. 
The internal content of the verb need not be in the immediate scope of the negation, though: there may be quantifiers or modal operators taking intermediate scope between the negation and the internal content of the verb. However, there must not be an intervening logical connective.%
%\footnote{This is reminiscent of the licensing condition for clause-bounded weak negative polarity items (NPIs), such as German \bspT{brauchen}{need}, in the DRT-based NPI-analysis in \citet{Sailer:habil,Sailer:sub08}. There, a clause-bounded NPI needs to be embedded within an accessible negative DRS within the clause in which it occurs.}

This is reminiscent of the situation found in languages which require a negative marker on the verb in negated sentences, such as Polish. For Polish, \citet{Richter:Sailer:06} formulate an LRS version of the Neg Criterion from \citet{Haegeman&Zanuttini:96}, requiring that whenever a verb is in the scope of negation in its external content, that negation must be contributed by the verb. 
For English, there is no such contribution requirement. Nonetheless, there is a similar connection between the verb's semantics and the negation.
In \refer{def-negatedclause} we attempt a definition of what is a negated clause.

\ea English negated clause:\label{def-negatedclause}\\
An English clause is \emph{negated} iff its internal content is in the scope of negation within its external content and there are no intervening connectives.
\z 

This independently relevant characterization of a negated sentence is sufficient to exclude the reading in \refer{ex-brief-mail-biWide}. 
The internal content of the sentence is $\co{answer}(\co{alex},x)$. 
While the semantic ATB exception allows the negations contributed in the conjuncts to take wide scope over the overall conjunction, this leads to a constellation that does not express a negated sentence.

\bigskip%
We have seen how we can derive the mono-propositional and the bi-pro\-po\-si\-tional readings for CNNP in LRS. 
To do this, we did not have to change anything in the analysis of StG as a non-NC language. We modified the \SemATB{} to include the semantic analogue of the empirically well-motivated Coordinate Structure Constraint with the ATB exception to extraction from conjuncts. 
%We also saw that the \CCB{} prevents wide scope negation and semantic ATB in bi-propositional readings.

\subsubsection{Split readings}
\label{Sec-AnalysisSplitReadings}

We have seen in Section~\ref{Sec-LRS-Split} that LRS allows us to capture split readings of n-words. The important part of the lexical specification on an n-word is that there is a metavariable between the negation and the existential quantification contribution constraints, i.e., the specification is of the form $\lnot A[\exists x (\ldots)]$. We will show that the same is true for CNNP, in both the mono-propositional and the bi-propositional reading.

We will analyze the example sentence in \refer{brauch-brief-mail} in this subsection. Since the example uses the NPI \bspT{brauchen}{need}, the negation must take scope over the semantic contribution of the modal verb.%
\footnote{The NPI requirement of \bspT{brauchen}{need} can be expressed as an indirect contribution constraint in LRS, see \refer{le-brauchen}.
The modal verb contributes a necessity operator and requires that this operator be in the scope of a negation, though it does not contribute the negation.
This encoding was proposed in \citet{Penn:Richter:05}.

\ea \label{le-brauchen}
\bsp{brauchen}: $\noc{\lnot}A[\Box(B)]$
\z 

A more refined approach to NPIs within LRS is pursued in \citet{Richter:Soehn:06} and \citet{Sailer:habil}.

}
Furthermore, narrow scope of the existential quantifier contributed by the n-constituents is the most natural reading. 
Below the example, we provide the semantic representation for the mono-propositional  reading.

\vbox{
\ea \label{brauch-brief-mail}
\begin{xlist}
\ex 
\gll Alex \npi{braucht} [\NE{keine} Briefe und \NE{keine} e-Mails] (\reci{miteinander}) zu vergleichen.\\
Alex needs \hphantom{[}no letters and no {e-mail messages} {with each other} to compare\\
\ex Mono-propositional reading:\\
$\lnot \Box \exists z ((
\exists x (\co{letter}(x) : x \in \pI z)
\land \exists y (\co{mess}(y) : y \in \pII z))$\\
\hspace*{\fill}$: \co{compare}(\co{alex},z)
)$
\end{xlist}
\z 
}

The analysis of this example is more or less parallel to that of the mono-propo\-si\-tional reading of \refer{ex-brief-mail-ana}. The two conjuncts both contribute constraints of the form $\lnot A [\exists x (\ldots)]$. 
So, they both leave room between the negation and the existential quantifier. For the sentence without an additional modal operator, the set-valued discourse referent $z$ is introduced in the scope of this negation.
Consequently, nothing speaks against also adding the modal operator contributed by \bspT{brauchen}{need}.

The modal verb \bsp{brauchen} requires that the core meaning of the verb it embeds occur in its scope. In this example, the formula $\co{compare}(\co{alex},z)$ is required to be in the scope of $\Box$. This constraint is satisfied in the indicated reading as well. 

The \SemATB{} is equally satisfied: both conjuncts contribute a negation, so this negation can outscope the overall conjunction. \SemATB{} does not require that the outscoping operator have immediate scope over the conjunction, so intervening material is not excluded.

% \kommentar{%raus: aus den Folien
% \ea  
% $\uwave{\underline{\lnot}} \co{\npi{NEED}}$\\
% \qquad $(\exists X ((
% \underline{\exists x_1 (\co{letter}(x_1) : x_1 \in X)}$\\
% \qquad \qquad $\land 
% \uwave{\exists x_2 (\co{message}(x_2) : x_2 \in X)}
% )$\\
% \qquad $: \co{answer}(\co{alex},X)))
% $
% \z 


% \begin{itemize}
% \item SemATB allows for negation over the conjunction.
% \item The NPI (\npi{\bsp{braucht}}) must be in the scope of negation.
% \item The negation can take scope over the predicate.
% \end{itemize}
% }

We can equally derive a bi-propositional analysis of the split reading. 
The semantic representation of such a reading is given in \refer{brauch-brief-mail-bi}.

\ea \label{brauch-brief-mail-bi}
\begin{xlist}
\ex 
\gll Alex \npi{braucht} [\NE{keinen} Brief] und [\NE{keine} e-Mail] zu beantworten.\\
Alex need \hphantom{[}no letter and \hphantom{[}no {e-mail message} to answer\\
\glt \mytrans{Alex need not answer any letter and Alex need not answer any e-mail message.}
\ex 
Bi-propositional reading:\\
$\lnot \Box \exists x (\co{letter}(x) : \co{answer}(\co{alex}, x))$\\
\hspace*{\fill}$\land 
\lnot \Box \exists x (\co{mess}(x) : \co{answer}(\co{alex}, x)) 
$
\end{xlist}
\z 

In this representation, the modal operator $\Box$ occurs twice, but the two occurrences have different scopes. 
For this purpose, the equality up-to extension of the lexical entry of the coordination particle is needed.



The constraints of the two conjuncts are as given above in \refer{ex-brief-mail-conj2}. 
Combining them with the bi-propositional coordination particle leads to the constraint in \refer{NandN-split-bi}.
This constraint already contains the occurrence of the modal operator in the second conjunct, $\Box(V)$. 
This anticipates the combination with the modal verb in the sentence and the occurrence of the modal operator in the first conjunct. The constraint $\Box(V)$ is contributed by virtue of the equality up-to extension of the coordination particle. 

\ea \label{NandN-split-bi}
\bsp{keinen Brief und keine e-Mail}:\\
$I[\{F[\dr{x}, \lnot A[\exists x (B[\co{letter}(x)] : B'[x])]]$\\ 
\hspace*{1em}$\land 
G[\dr{x},\Box(V), \lnot D[\exists x (E[\co{mess}(x)]: E'[x])]]\}]$
\z 

When the coordinated noun phrases combine with the verb \bspT{beantworten}{answer}, we get the following constraint.

\ea \label{VP-split-bi}
\bsp{keinen Brief und keine e-Mail zu beantworten}: \\
$\exc J[I[F[x, \lnot A[\exists x (B[\co{letter}(x)] : B'[x])]]$
\hfill (first conjunct)\\
$\land 
G[x,\Box(V), \lnot D[\exists x (E[\co{mess}(x)]: E'[x])]]],$
\hfill (second conjunct)\\
$\inc{\co{answer}(\noc{\co{alex}},\noc{x})}
]$ \hfill (verb)
\z 

The modal \bspT{brauchen}{need} contributes a modal operator that takes scope over the internal content of the VP, $\co{answer}(\co{alex},x)$, which is also the internal content of the modal verb.
The subject, \bsp{Alex}, only contributes the name constant \co{alex}. 
The constraint for the overall sentence is given in \refer{braucht-VP-split-bi}.

\ea \label{braucht-VP-split-bi}
$K[J[I[F[x, \lnot A[\exists x (B[\co{letter}(x)] : B'[x])]]$
\hfill (first conjunct)\\
\hspace*{1.5em}$\land 
G[x,\Box(V), \lnot D[\exists x (E[\co{mess}(x)]: E'[x])]]],$
\hfill (second conjunct)\\
$\inc{\co{answer}(\noc{\co{alex}},\noc{x})}
],$ \hfill (verb)\\
$\Box(U[\co{answer}(\noc{\co{alex}},\noc{x})]),$
\hfill (modal verb)\\
$\co{alex}]$ \hfill (subject)
\z 

Finally, we provide the plugging that leads to the intended reading in \refer{plug-brauch-bi}.

\ea \label{plug-brauch-bi}
$A = \Box(U) = \Box(\exists x (\co{letter}(x) : \co{answer}(\co{alex},x)))$\\
$B = \co{letter}(x)$
\qquad $B' = \co{answer}(\co{alex},x)$\\
$D = \Box(V) = \Box(\exists x (\co{mess}(x) : \co{answer}(\co{alex},x)))$\\
$E= \co{mess}(x)$
\qquad $E'= \co{answer}(\co{alex},x)$\\
$F= \lnot \Box(\exists x (\co{letter}(x) : \co{answer}(\co{alex},x)))$\\
$G= \lnot \Box(\exists x (\co{mess}(x) : \co{answer}(\co{alex},x)))$\\
$I=J=K= F\land G$\\
\hspace*{3em} $= 
\lnot \Box(\exists x (\co{letter}(x) : \co{answer}(\co{alex},x)))$\\
\hspace*{\fill} $\land 
\lnot \Box(\exists x (\co{mess}(x) : \co{answer}(\co{alex},x)))
$\\
$U= \exists x (\co{letter}(x) : \co{answer}(\co{alex},x))$\\
$V= \exists x (\co{mess}(x) : \co{answer}(\co{alex},x))$
\z 

The plugging in \refer{plug-brauch-bi} satisfies the constraint from \refer{braucht-VP-split-bi}.
The conjunction is the highest operator in the resulting representation. The two negations are interpreted within their respective conjuncts. The equality up-to constraint allows us to use the modal operator $\Box$ twice, though with not fully identical formul\ae{} in the scope of the two occurrences.


Without the equality up-to extension,
the only possible reading would be a non-split reading, i.e., a reading in which the modal operator is in the scope of the existential quantifiers, given in \refer{brauch-brief-mail-biWide}.
We can still derive this reading, as the equality up-to part is optional.

\ea \label{brauch-brief-mail-biWide}
Bi-propositional reading with narrow scope of the modal operator:\\
$\lnot \exists x (\co{letter}(x) : \Box \co{answer}(\co{alex}, x))$%\\ 
$\land 
\lnot \exists x (\co{mess}(x) : \Box \co{answer}(\co{alex}, x)) 
$
\z 



% \kommentar{
% %MS raus, 19.5.19: Grund, passt irgendwie nicht mehr
% To derive this reading in \refer{brauch-brief-mail-biWide}, the conjunction particle comes with the following constraint.

% \ea \label{and-split-bi}
% \bspT{und}{and}: $\inc{F[x] \land G[x]}, F/U\sim G/V$\\
% in \refer{brauch-brief-mail-biWide}: 
% $F[x] \land G[x,\Box(V)]$
% \z 
% }

Just as shown above for the structurally simpler example \refer{ex-brief-mail-biWide}, we do not get a bi-propositional analysis in which there is just one negation in the overall semantic representation. In other words, the formula in \refer{ATB-brauch-bi} cannot occur as the semantic representation of our example sentence since there is a coordination intervening between the internal content of the verb, $\co{answer}(\co{alex},x)$, and the negation, i.e., this semantic representation does not express a negated sentence.

\ea \label{ATB-brauch-bi}
$\lnot (\Box(\exists x (\co{letter}(x) : \co{answer}(\co{alex},x)))$\\
\hspace*{\fill} $\land 
\Box(\exists x (\co{mess}(x) : \co{answer}(\co{alex},x))))$
\z 

%As above, this reading is excluded by the constraint on clausal negation that bans a logical connective intervening between a clausal negation and the verb's internal content.
%the \CCB{} from \refer{SemATB}: in the bi-propositional analysis, whatever is contributed within one conjunct constituent needs to be part of the semantic representation of this constituent.

\bigskip%
We have shown that the split readings can be derived for both the mono-propositional and the bi-propositional analysis of CNNPs. For the first case, we made use of the semantic ATB-exception incorporated into the \CCB{} in \refer{SemATB}. For the second case, we saw the effect of the equality up-to constraint and the non-applicability of the semantic ATB exception.


\subsubsection{Disjunction effect}
\label{Sec-AnalysisDisjunctionEffect}

Before we close the presentation of the analysis, we should have another look at the \emph{disjunction effect}.
We saw that distributive readings only emerge under a bi-propositional analysis. 
 The bi-propositional formula from \refer{alex-cnnp-lf}, repeated in \refer{alex-cnnp-lf3bi}, is logically equivalent to the one given in \refer{alex-cnnp-lf3}, in which there is a disjunction in the restrictor of the existential quantifier, i.e., the quantification takes any assignment for $x$ into consideration that is a letter or an e-mail message.

\ea 
\begin{xlist}
\ex \label{alex-cnnp-lf3bi}
$\lnot \exists x (\co{letter}(x) : \co{write}(\co{alex},x))$%\\
%\quad 
$\land \lnot \exists x (\co{mess}(x) : \co{write}(\co{alex},x))$
\ex \label{alex-cnnp-lf3}
$\lnot \exists x ((\co{letter}(x) \lor \co{mess}(x)) : \co{write}(\co{alex},x))$
\end{xlist}
\z 

Our analysis has a number of attractive features: 
We can assume a surface-oriented syntactic analysis, i.e., an analysis in terms of noun phrase coordination, and the conjunction particle \bspT{und}{and} is translated as ordinary boolean conjunction. 
Nonetheless, we derive a bi-propositional semantic representation which is equivalent to a disjunctive mono-propositional representation.


% \kommentar{Das ist nicht der Punkt:
% This is not equivalent to an interpretation that combines the two conjuncts with a logical conjunction, i.e., the formula in \refer{alex-cnnp-lf4}. In this reading, $x$ would be required to be both a letter and an e-mail message.

% \ea \label{alex-cnnp-lf4}
% $\lnot \exists x ((\co{letter}(x) \land \co{e-mail-mess}(x)) \land \co{write}(\co{alex},x))
% $
% \z 
% }


% \kommentar{MS raus, 19.5.19; stimmt nicht mehr
% The formula in \refer{alex-cnnp-lf3} shows that 
% our mono-propositional analysis in  \refer{alex-cnnp-lf} captures the disjunction effect in an elegant way: while the conjoined noun phrases are combined logically with a conjunction, the overall interpretation is rather like a disjunction. 

% Similarly, for the bi-propositional analysis, we only get the reading in \refer{alex-cnnp-lf3bi}.
% }

% \kommentar{
% The basic observation is that we get a contrast between CNNP and a negation taking scope over two conjoined indefinite noun phrases. CNNP only allows for a ``neither'' reading, whereas negated conjoined indefinites allow for both a ``neither'' and a ``not both'' reading. 
% We illustrated this above with example \refer{ex-tie-glaub-both}, repeated in \refer{ex-glaub-bi}.

% \ea \label{ex-glaub-bi}% = \refer{ex-tie-glaub-both}\\
% \gll Ich glaube nicht, dass Monika [einen Vortrag und ein Seminar] zu halten \npi{braucht}.\\
% I believe not that Monika a lecture and a seminar to teach need\\
% \glt \mbox{I don't think that Monika need teach a lecture and a seminar.}
% \z 
% }

We can now turn to the contrast between CNNP and the negation of conjoined indefinite noun phrases, illustrated in examples \refer{ex-tie-glaub-both} and \refer{ex-tie-glaub-neither} above. 
The contrast only arises in cases in which a bi-propositional reading is possible. 
We observed that CNNP does not allow for a ``not both'' reading, while this reading is readily available for negated conjoined indefinites. 
We exclude the ``not both'' reading for CNNP as a consequence of deriving the disjunction effect. 

We will show how we derive the ``not both'' reading for conjoined indefinite noun phrases. 
We repeat the relevant sentence in \refer{ex-glaub-notEX2}. 
Again, we use the NPI-verb \bspT{brauchen}{need} to guarantee that the negation is interpreted in the embedded clause.
For the purpose of this subsection, we are only interested in narrow scope readings of the indefinite noun phrases.



% \kommentar{%raus: stimmt jetzt nicht mehr
% Since we assume that the indefinite article can be translated either by $\existsE$ or by $\existsN$ lead to a total of four different readings ($\existsE \existsE$, $\existsN \existsN$, $\existsE \existsN$, and $\existsN \existsE$). 
% The readings in which the same translation is chosen for the two conjuncts are truth-conditionally equivalent and correspond to the ``neither'' reading. If, however, we use a $\existsE$ for one conjunct and $\existsN$ for the other, we get the ``not both'' reading. In \refer{ex-glaub-notEX-nb} this is illustrated for the option $\existsE \existsN$.
% }


\ea \label{ex-glaub-notEX2}
\gll Alex glaubt nicht, dass Monika eine Vorlesung und ein Seminar zu halten braucht.\\
Alex believes not that Monika a lecture and a seminar to teach need\\
\glt \mytrans{Alex doesn't think that Monika need teach a lecture and a seminar.}
\z 

Before considering the example in \refer{ex-glaub-notEX2}, we will start with the simpler sentence in \refer{notEX}. This example has neither a modal nor an attitude predicate, but will still allow us to describe the relevant readings.
%To be able to talk about the various readings in a more straightforward way, however, we will work with a sentence that has the relevant constellation but no modal and intensional operators.

\ea \label{notEX}
\gll Es stimmt nicht, dass Monika eine Vorlesung und ein Seminar hält.\\
it {is true} not that Monika a lecture and a seminar teaches\\
\glt \mytrans{It is not true that Monika teaches a lecture and a seminar.}
\z 

% \kommentar{%stimmt nicht, gibt nur die "not both"-derivation
% We will show that this sentence has two possible translations, given in \refer{notEX-bi-lf}.

% \ea \label{notEX-bi-lf}
% \begin{itemize}

% \item ``not both'' reading:\\
% $\lnot (\exists x (\co{lecture}(x) : \co{teach}(\co{monika},x))$\\
% \qquad $\land \exists x (\co{seminar}(x) : \co{teach}(\co{monika},x)))$\\
% $\equiv$
% $(\lnot \exists x (\co{lecture}(x) : \co{teach}(\co{monika},x)))$\\
% \qquad $\lor (\lnot \exists x (\co{seminar}(x) : \co{teach}(\co{monika},x)))$
% \item ``neither'' reading:\\
% $(\lnot \exists x (\co{lecture}(x) : \co{teach}(\co{monika},x)))$\\
% \qquad $\land (\lnot \exists x (\co{seminar}(x) : \co{teach}(\co{monika},x)))$
% \end{itemize}

% \z 
% }

If we interpret the embedded sentence first and then add a negation through the main clause, we arrive at the semantic representation in \refer{notEX-bi-lf} with a wide-scope negation over the conjunction. 
By de Morgan's laws, this is logically equivalent to
 a disjunction of two negated formul\ae.

\ea \label{notEX-bi-lf}
$\lnot (\exists x (\co{lecture}(x) : \co{teach}(\co{monika},x))$\\
\hspace*{\fill}
$\land \exists x (\co{seminar}(x) : \co{teach}(\co{monika},x)))$\\
$\equiv$
$(\lnot \exists x (\co{lecture}(x) : \co{teach}(\co{monika},x)))$\\
\hspace*{\fill} $\lor (\lnot \exists x (\co{seminar}(x) : \co{teach}(\co{monika},x)))$
\z 
 
The formul\ae{} in \refer{notEX-bi-lf} are true as long as Monika does not teach both a lecture and a seminar. This covers the ``neither'' case, but is weaker in that it is also compatible with a situation in which Monika teaches a lecture but not a seminar, or the other way around.

We can now turn to the more complex example in \refer{ex-glaub-notEX2}. 
This example includes a modal verb to ensure a neg-raising reading. 
In the following, we will, however, ignore the semantic contribution of the modal verb.
In an LRS analysis of neg-raising, \citet{Sailer:05cssp} assumes that the negation that is syntactically part of the matrix clause is interpreted inside the embedded clause. This leads to the semantic representation in \refer{ex-glaub-notEX2-bi}.%
%\footnote{Sailer argues that the negation }

\ea \label{ex-glaub-notEX2-bi}
$\co{believe}(\co{alex},\lnot (\exists x (\co{lecture}(x) : \co{teach}(\co{monika},x))$\\
\hspace*{\fill}
$\land \exists x (\co{seminar}(x) : \co{teach}(\co{monika},x))))$
\z 

This formula expresses the ``not both'' reading. This shows that we correctly derive the difference between CNNP and coordinated non-negative indefinites in the scope of negation.



% \kommentar{
% This sentence allows, in fact, for any of the following four continuations:

% \ea \label{notEX-cont}
% \begin{itemize}
% \item \ldots Sie unterrichtet nämlich gar nicht. [``neither'']
% \item \ldots Sie hält nämlich nur eines davon. [``not both'']
% \item \ldots Sie hält nämlich nur eine Vorlesung. [``not the second'']
% \item \ldots Sie hält nämlich nur ein Seminar. [``not the first'']
% \end{itemize}
% \z 
% }

% \kommentar{
% Let us look at the translation of this sentence under the bi-propositional translation of the coordination particle.

% \ea 
% \bspT{eine Vorlesung}{a lecture}: 
% $\exists x (B[\co{lecture}(x)]: B'[x])$\\
% \bspT{ein Seminar}{a seminar}: 
% $\exists x (E[\co{lecture}(x)]: E'[x])$\
% \z 

% \ea
% \bspT{und}{and}: $(F[x] \land G[x])$, $F/U \sim G/V$
% \z 
% }



% \kommentar{
% To see that this predication is correct, we can go back to the examples \refer{ex-eltern} and \refer{ex-eltern2}. 
% We argued that in some cases only a bi-propositional analysis is available. 
% In the relevant examples, we had a negated conjuncts in the Vorfeld and we did not get a reading in which the predicate expressed by the verb could relate elements of members of the two conjuncts. With singular conjuncts and a collective predicate, this led to an uninterpretability; with plural conjuncts and a collective predicate, this led to readings in which the predicate only applies within the conjunct sets.

% \ea
% Alex sucht kein Einhorn.
% \z 

% \ea 
% \gll Alex sucht kein Einhorn und keinen Wolpertinger.\\
% Alex seeks no unicorn and no wolpertinger\\
% \z

% \ea
% \gll {}[\NE{Kein} Kind und \NE{kein} Erwachsener] \npi{brauchen} zu streiten.\\
% no child and no adult need to quarrel\\
% \z 

% \ea 
% \gll {}[\NE{Keine} Kinder und \NE{keine} Erwachsenen] \npi{brauchen} zu streiten.\\
% no children and no adults need to quarrel\\
% \z 


% \ea 
% Alex braucht kein Gedicht und kein Drama miteinander zu vergleichen.
% \z 

% \subsubsection{Disjunction effect}


% \ea
% \gll {}[\underline{\NE{Kein} Student} und \uwave{\NE{kein} Dozent}] hat gelacht.\\
% {no student} and {no lecturer} has laughed\\
% \z 

% \ea Fully bi-propositional reading (``disjunction effect''):\\
% $\underline{\lnot \exists x (\co{student}(x): \co{laugh}(x))}
% \land 
% \uwave{\lnot \exists x (\co{lecturer}(x): \co{laugh}(x))}
% $\\
% $\equiv
% \lnot \exists x ((\co{student}(x) \lor \co{lecturer}(x)): 
% \co{laugh}(x))
% $
% \z 

% \ea \label{split-ccb} Readings expected under a split analysis:\\
% \# $\uwave{\underline{\lnot}} (\underline{\exists x (\co{student}(x): \co{laugh}(x))}
% \land 
% \uwave{\exists x (\co{lecturer}(x): \co{laugh}(x))})
% $
% \z 
% \begin{itemize}
% \item Negation is contributed by each conjunct.
% \item Bi-propositional conjunction, i.e.\@ \textsc{dr}-identity.
% \item Conjunct Confusion Blocking enforces negation to be conjunct internal.
% \item Reading \refer{split-ccb} ok if negation is not in the conjuncts!
% \end{itemize}

% }

\section{Consequences of the analysis}
\label{Sec-Analysis-Consequences}

In this section, we will put our analysis of CNNP in StG in the context of related data: First, we will look at CNNP in languages with negative concord, Section \ref{Sec-AnalysisOhterLang}. Second, we will compare CNNP to coordination with the negative coordination particles \bsp{neither \ldots nor} in Section \ref{Sec-NeitherNor}.

\subsection{Application to NC languages}
\label{Sec-AnalysisOhterLang}

%\marginpar{MS: Hedde meint, wir sollen Polnisch statt Franz.\@ nehmen. Habe Ewa Trutkowski kontaktiert, aber die Daten kann ich auf die Schnelle nicht so einbauen.}

%VIELLEICHT KÖNNEN WIR DAS HIER ERSTMAL OFFEN LASSEN. ICH DENKE ABER, DIE DATEN SIND ÄHNLICH WIE IM DT, WIE WIR DAS AUCH VORHERSAGEN. DAS IST DANN MOTIVATION, DASS DER NC/nicht-NC-UNTERSCHIED NICHT SO WIRKLICH GROß IST.

A basic assumption of the LRS approach to negation is that there is no difference in the lexical specifications of n-words in NC and non-NC languages. 
The differences lie in the interpretational strategies and in the inventory of words associated with negation. 
Since coordination structures are exempt from the Negation Faithfulness Constraint in StG, a semantic representation can be derived that is based on the same mechanism that we use for negative concord, namely the identity of semantic contributions.

This leads to the prediction that NC-languages should behave just like StG with respect to the interpretation of CNNP. 
In this paper, we cannot fully explore this prediction. 
We will briefly consider French, an optional NC language, but have to postpone the application to an obligatory NC language such as Polish.

A French CNNP-sentence is given in \refer{Meaux}. The sentence has the same truth conditions as the corresponding StG example sentences.
In particular, we get the disjunction effect, i.e., a ``neither'' reading.


\ea \label{Meaux}
\gll 
{}[\NE{Aucun} train et \NE{aucun} [bus ou car]] ne partait de la gare de Meaux \ldots\\
no train and no bus or coach NM left from the station of Meaux\\
\glt \mytrans{Neither trains nor buses or coaches left from Meaux station.}\\
(\url{www.leparisien.fr/seine-et-marne-77/meaux-bloques-a-la-gare-les-voyageurs-pas-en-colere-mais-resignes-07-02-2018-7546974.php}, 2018-04-28)
\z 

French also allows negated conjuncts to act as 
complements of a collective verb.
%antecedent for a reciprocal or a collective expression. 
This is shown in \refer{roman-poeme}.%
\footnote{The availability of a mono-propositional reading seems to be as restricted as in StG, i.e., many speakers may reject this reading.}
This points to a mono-propositional analysis for French CNNP. 

\ea \label{roman-poeme}
\gll Léo n' a comparé [\NE{aucuns} romans et \NE{aucuns} poèmes].\\
Léo NM has compared \hphantom{[}no novels and no poems\\
\glt \mytrans{There are no novel-poem pairs such that Léo compared the novel and the poem.}
\z 

To complete the similarity between French and StG, we find split readings in French as well, see \refer{fr-Monique}.

\ea \label{fr-Monique}
\gll Monique n' est obligée de diriger [\NE{aucune} communication et \NE{aucun} séminaire].\\
Monique NM is obliged to give \hphantom{[}no lecture and no seminar\\
\glt \mytrans{It is not  the case that Monique is obliged to teach a lecture and a seminar.}
\z 



We take this parallel behavior as support for our approach: The lexical encoding of n-words is the same in NC and non-NC languages, but they show different interpretational strategies in headed structures. 
In coordinated structures, however, there are no differences in the interpretation strategies, consequently, the same readings obtain, independently of a language's NC status.

Given the repertoire of negation-related expressions in French and the interpretation strategies of French, the negative determiner \bspT{aucun}{no} is not as common as its StG counterpart \bsp{kein-}. 
Again, this is independent of the different NC statuses of French and StG. 
Standard English is a non-NC language like StG, but, just as French, uses verbal negation 
more frequently than StG. 
Therefore, negative determiners are much less common in English than they are in StG. 

\subsection{CNNP vs.\@ \bsp{neither nor}}
\label{Sec-NeitherNor}


We have characterized CNNP as giving rise to a ``neither'' reading in many places in this paper.
Whereas CNNP has not received systematic attention in the literature, negative conjunctions of the \bsp{neither nor}-type have been explored \citep{deSwart:01ni,Doetjes:05,Gajic:16}.%
%\footnote{Jovana Gaji\'c (Göttingen) explores \bsp{neither nor} from a cross-linguistic perspective. This is particularly interesting for us here, as she assumes a difference in meaning between NC and non-NC languages. However, since her analysis has not yet appeared, we cannot compare it explicitly.}

Sticking to StG data, we see that \bsp{neither nor} conjuncts as in \refer{weder-reci} cannot serve as a collective antecedent for a reciprocal pronoun.

\ea \label{weder-reci}
\gll ?* Alex hat [weder einen Roman noch ein Gedicht] \reci{miteinander} verglichen.\\
{} Alex has \hphantom{[}neither a novel nor a poem {with each other} compared\\
\glt \# \mytrans{There is no novel-poem pair such that Alex compared the novel with the poem.}
\z 

The same can be shown with an inherently reflexive collective predicate 
in \refer{weder-treffen}.

\ea \label{weder-treffen}
\gll ?* [Weder ein Kind noch ein Erwachsener] haben sich getroffen.\\
{} \hphantom{[}neither a child nor an adult have \textsc{refl} met\\
\glt \# \mytrans{There had not been a meeting between a child and an adult.}
\z 

Based on these observations, we conjecture that \bspT{weder noch}{neither nor} always gives rise to a bi-propositional semantic representation. 
This is in line with \cite[33]{Winter:01}, who argues that all coordination  particles except for \bsp{and} and its cognates in other languages trigger a bi-propositional analysis.

The example in \refer{brauch-weder} shows that we can find split readings with \bspT{weder noch}{neither nor}, which is parallel
to what we found with the bi-propositional readings in CNNP-sentences.

\ea \label{brauch-weder}
\gll Monika \npi{braucht} [weder einen Vortrag noch ein Seminar] zu halten.\\
Monika need \hphantom{[}neither a lecture nor a seminar to teach\\
\glt %\mytrans{Monika need teach neither a lecture nor a seminar.}
\mytrans{Monika need not teach a lecture and Monika need not teach a seminar.}
\z 

We will not give an analysis of \bspT{weder noch}{neither nor}, especially since we do not want to commit ourselves to a particular syntactic analysis for the conjunction particles. The data discussed in this subsection, however, suggest that an LRS analysis would include lexical entries for the conjunction particles that are like the lexical entry for bi-propositional \bspT{und}{and} in \refer{le-und1-Pinkal}, but include a negation. This is sketched in \refer{le-weder}.

\ea \label{le-weder}
\bspT{weder noch}{neither nor}: $\{(\lnot F[\dr{x}] \land \lnot G[\dr{x}])\}$, $F/U \sim G/V$
\z 


In this section, we briefly explored the consequences and predictions of our analysis of CNNP in StG to two related phenomena -- CNNP in an optional NC language and \bsp{neither nor} coordination. 

\section{Anaphoric relations among the conjuncts}
\label{Sec-Anaphor}

%\marginpar{MS: hier kommt jetzt der kritische Abschnitt}

It is essential for our analysis of the bi-propositional reading that the two conjoined noun phrases have the same index. 
While this is an example of a one-to-many relation -- the same index being used in two conjuncts -- it might have undesired consequences.
A potentially problematic example is given in \refer{ex-some-actress}. 
In this example, the second conjunct contains a pronoun that is coindexed with the first conjunct, but the second conjunct refers to a different entity.%
\footnote{The classical example of this constellation is given in \refer{dog}, which is discussed in \cite[24]{Moltmann:92}, for example.

\ea 
Every man and his dog left.\label{dog}
\z 

}

\vbox{
\ea \label{ex-some-actress}
\begin{xlist}
\ex []{
%My son hates [a teacher]$_i$ and [her$_i$ subject]$_j$.}
\label{ex-some-actressA}
Alex adores [a French actress]$_i$ and [(some of) her$_i$ films]$_j$.}
\ex 
%[$=$]{My son hates a teacher and  my son hates her subject.}
[$=$]{Alex adores [a French actress]$_i$ and Alex adores [(some of ) her$_i$ films]$_j$.}
\label{ex-some-actressB}
\end{xlist}
\z
}

In this section, we will first look at such data independently of negation, then we will discuss corresponding constellations for CNNP cases.


\subsection{Anaphoric relations in non-negated conjuncts}
\label{Sec-Anaphor-Pos}

To get a better idea of the correct analysis of sentences such as \refer{ex-some-actress}, it is worth looking at analogous examples with other determiners.
In \refer{ex-every-ana}, a universal determiner is used. 
As above, the anaphorical relation between the conjoined NPs is possible. 
Nonetheless, a bi-clausal paraphrase as in \refer{ex-every-ana-french} is not possible, as the universal quantifier does not easily allow for cross-sentential anaphora \citep{Kamp:81}.
This shows that these data cannot be captured in a straightforward way in an analysis that uses \refer{ex-every-ana-french} as the syntactic basis for the surface noun phrase conjunction in \refer{ex-french-actress}.



\ea \label{ex-every-ana}
\begin{xlist}
%\ex []{Alex adores every French actress and her films.}
%\ex Alex interviewed an actress and her manager.\label{student-supervisor}
%\ex Alex interviewed every actress and her manager.
%Alex interviewed every actress and Alex interviewed her manager.
\ex []{Alex adores [every French actress]$_i$ and [(some of) her$_i$ films]$_j$.}\label{ex-french-actress}\label{ex-every-anaA}
\ex [*]{Alex adores [every French actress]$_i$ and Alex adores [(some of) her$_i$ films]$_j$.}\label{ex-every-ana-french}
\label{ex-every-anaB}
\end{xlist}
\z 

% \kommentar{
% \ea \label{ex-most-ana}
% \begin{xlist}
% %\ex []{Alex adores every French actress and her films.}
% %\ex Alex interviewed an actress and her manager.\label{student-supervisor}
% %\ex Alex interviewed every actress and her manager.
% %Alex interviewed every actress and Alex interviewed her manager.
% \ex []{Alex knows [many/most German sinologists]$_i$ and [(some of) their$_i$ papers]$_j$.}\label{ex-most-french-actress}\label{ex-most-every-anaA}
% \ex [*]{Alex knows [many/most German sinologists]$_i$ and Alex knows [(some of) their$_i$ papers]$_j$.}\label{ex-most-ana-french}
% \label{ex-most-anaB}
% \end{xlist}
% \z 
% }


The data are equally problematic for both our mono-propositional and our bi-propositional approach. 
A mono-propositional analysis will be confronted with the same problem as \refer{ex-every-anaB}, i.e., the universal quantifier contributed in the first conjunct only has scope within the first conjunct and cannot bind a variable in the second conjunct. 
It is furthermore doubtful that we can pursue a mono-propositional analysis for the sentences \refer{ex-some-actressA} or \refer{ex-every-anaA}. The sentences use a distributive predicate and, as we saw above, mono-propositional readings are dispreferred. The anaphoric relation in the given sentences is, however, unproblematic.

\kommentar{
\ea
$\exists Z ((\forall x (\co{actress}(x) : x \in \pI z) \land(\forall y (\co{film}(y,x) :$
\z 
}
\begin{sloppypar}
  The problem for the bi-propositional approach is different.  Our
  bi-propositional semantic representations rely on using the same
  discourse referent marker for both conjuncts. This does not seem
  possible in examples like \refer{ex-some-actressA} and
  \refer{ex-every-anaA}.
\end{sloppypar}


We will show how the present approach can be extended to capture the data with anaphoric relations across the conjuncts. 
The basic idea of our analysis will be that the quantifier in the first conjunct in \refer{ex-some-actressA} and \refer{ex-french-actress} takes wide scope over both conjuncts. 
To achieve this, we will apply an \emph{existential split}, i.e., we will introduce an additional existential quantifier in the scope of the overt quantifier. 
Let us introduce the necessary tools step by step.

\refer{exist-splitA} shows a simple quantified formula in which the variable $x$ is bound. In \refer{exist-splitB}, the scope of the determiner is enhanced by an existential quantifier binding the variable $x$. 
The original quantifier, $\mathcal{Q}$ binds a new variable, $y$, and we need to replace all free occurrences of $x$ in the restrictor of $\mathcal{Q}$ with $y$. In the scope of the quantifier, the restrictor of the existential quantifier is the formula $x=y$. 
As indicated, the two expressions in \refer{exist-splitA}  and \refer{exist-splitB} are logically equivalent. 


\ea \label{exist-split}
For each variable $x, y$, each formula $\phi, \psi$ that has no free occurrence of $y$, and for each determiner $\mathcal{Q}$:
\begin{xlist}
\ex []{\label{exist-splitA}
$\mathcal{Q} x (\phi : \psi)$
}
\ex []{\label{exist-splitB}
%$\equiv \mathcal{Q} y (\exists x (x=y : \phi) : \exists x (x=y : \psi))$
$\equiv \mathcal{Q} y (\phi \replace{x}{y} : \exists x (x=y : \psi))$\\
where $\phi \replace{x}{y}$ is a formula that is identical to $\phi$ but with every free occurrence of $x$ replaced with $y$.%
\footnote{We use this idionsyncratic notation for the replacement of subexpressions instead of the more common $[x/y]$ to avoid ambiguous use of the square brackets.}
}
\end{xlist}
\z 


Existential split has no truth-conditional effect, but it allows us to introduce a new variable in the scope of the determiner.
So far, the lexical contribution of a logical determiner always had the form in \refer{exist-splitA}. We propose that it can, alternatively, have the form in \refer{exist-splitB}. 
The corresponding lexical specifications for \bsp{every} are given in \refer{le-every-split}.

\ea \label{le-every-split}
\begin{xlist}
\ex Simple specification: $\exc{\forall \dr{x} (A[x] : A'[x])}$
\ex Split specification: \label{def-every-split}
$\exc{\forall y (A[x]\replace{x}{y}) : B[\exists \dr{x} (x = y : A'[x]]))}$
%$\exc{\forall \dr{y} (\exists x (x=y: A[x]) : B[\exists x (x = y : A'[x]]))}$
%$\exc{\forall \dr{y} (\exists x (x=y: A[x]) : B[\exists x (x = y : A'[x]]))}$
\end{xlist}
\z 

The split specification makes it necessary to change our variable management. 
Now there are two variables associated with the noun phrase, $x$ and $y$. The variable $y$ will be used internal to the noun phrase, i.e., as the discourse referent marker of the noun and the determiner. The variable $x$ is used outside the noun phrase, for argument-identification, i.e.\@ linking. Therefore, this variable is used as the discourse referent marker of the quantified NP. As the variable $x$ will be related to the verb's argument structure, it is this variable that will be used for anaphoric binding. 
The variable $y$, on the other hand, is essential for all other cases of binding and coreference.
This includes binding into another conjunct, as in example \refer{ex-every-anaA}. 
This is shown in \refer{split-actress-dr}, where we indicate the discourse referent marker for the bi-propositional reading on each noun,  each determiner, and each noun phrase in the conjunction. 
In addition we mark the variable bound by the determiner in the noun phrases with an exclamation mark.
In the second conjunct, the determiner \bsp{some} and the head noun \bsp{films}
have the same discourse referent marker $x$ and this is also the variable bound by the quantifier. 
In the first conjunct the head noun has the discourse referent marker $y$, which is bound by the universal quantifier, marked as $y!$. However, the discourse referent marker of the determiner and the first conjunct is $x$.


\ea \label{split-actress-dr}
Alex adores [every$_{x,y!}$ French actress$_y$]$_x$ and [some$_{x!}$ of her$_y$ films$_x$]$_x$
\z 

This example shows that, as before, the discourse referent of each nominal head is bound by its quantificational determiner, and the discourse referent mark\-er of the quantificational determiner is the same as that of the noun phrase.  However, these two relations are now split over two variables: $y$ for the nominal head and $x$ for the noun phrase \bsp{every actress} in our example.

%has the same discourse referent marker -- $y$ for \bsp{every French actress}, and $x$ for \bsp{some of her films}. Also, both conjuncts can have the same marker, $x$.


The corresponding constraints on the discourse referent markers are given in \refer{dr-princ}. 
In all previous LRS publications, the discourse referent marker was shared between a mother node and its head daughter. We have to change this in such a way that it percolates from the non-head daughter in cases in which the non-head daughter is a logical determiner.


\ea \label{dr-princ}
\begin{xlist}
\ex []{In a head-specifier phrase with a non-head with a quantificational external content, the discourse referent marker of the phrase is identical with that of the specifier.}
\ex []{In all other cases, the discourse referent marker of the head and the mother are identical.}
\end{xlist}
\z 

The new percolation mechanism is illustrated for our example in \refer{ex-dr-princ}.


\ea \label{ex-dr-princ}
\begin{xlist}
\ex []{\bsp{actress}: $\co{actress}(\noc{\dr{y}})$}
\ex []{\bsp{every}: $\exc{\forall y (A[y]:B[\exists \dr{x} (x=y : A'[x]])}$}
\ex []{\bsp{every actress}: 
$\exc{\forall y (A[\co{actress}(y)]:B[\exists \dr{x} (x=y : A'[x]])}$
}
\end{xlist}
\z 


The new, split, encoding of the quantifiers opens up the possibility to insert operators that are in the scope of the quantifier but have scope over the embedded existential. This option is indicated in \refer{def-every-split} by the metavariable $B$.

% \kommentar{%MS: andere Verwendung von x, auch innerhalb der NP
% The split specification makes it necessary to change our variable management slightly. Now there are two variables associated with the noun phrase, $x$ and $y$. The variable $x$ will be used as the discourse referent marker of the head noun and, consequently, of the overall noun phrase. The variable $y$, which is bound by the determiner contributed by the specifier of the noun phrase, is required for binding. 


% \ea
% every actress: 
% $\exc{\forall y (\exists \dr{x} (x=y: A[\inc{\co{actress}(x)}, x]) : B[\exists x (x = y : A'[x]]))}$
% \z 

% \ea
% Alex adores [every$_y$ French actress$_x$] and [some$_x$ of her$_y$ films$_x$]
% \z 
% }

We can now capture the examples with anaphoric binding into the second conjunct using the existentially split version of the determiner. 
In these cases, the semantics of the second conjunct will take scope below the external content of the first conjunct. 
In \refer{bi-every-ana2} this is shown for the bi-propositional analysis of sentence \refer{ex-every-anaA}. 

\ea \label{bi-every-ana2}
Bi-propositional representation of example \refer{ex-every-anaA}:\\[1ex]
$
\forall y (\co{actress}(y) : $\\
\hspace*{3em}$(\exists x (x = y : \co{adore}(\co{alex},x))
\land
\exists x (\co{film-of}(x,y) : \co{adore}(\co{alex},x))
))$
\z 



As in the simple cases discussed in Section \ref{Sec-Conjunction}, the two conjuncts have identical discourse referent markers, $x$, and they have the same expression in  their scope, $\co{adore}(\co{alex},x)$.
What is new is that 
the universal quantifier, $\forall y$, contributed in the first conjunct constituent takes scope over both conjuncts in the semantic representation.
There, it binds the variable $y$ and can, now, bind an occurrence of this variable in the second conjunct as well.
%Its discourse referent marker is the variable $y$.
%The conjoined propositions are given in the second line in \refer{bi-every-ana2}, 
%The referential variable in each conjunct is $x$. In the first conjunct, we need to set $x$ to the discourse referent marker of the first conjunct. 
%The second conjunct is interpreted as usual.
%For this reason, we need to introduce the additional semantic material $\exists x (x= Z : A)$.


% \kommentar{
% The hypothetical non-split formula would look as in \refer{lf-nonsplit}.
% This is, of course, due to the variable $y$ that occurs freely in the scope of the highest existential quantifier.

% \ea \label{lf-nonsplit}
% $\forall x (\co{actress}(x) : (\co{adore}(\co{alex}, x) \land \exists x (\co{film}(x,y) : \co{adore}(\co{alex}, x))$
% \z 
% }

%In fact, we only see evidence for the splitting in those cases in which we have to express something that cannot be expressed without the split, i.e., we can only observe the application of splitting in those cases in which the split and the non-split respresentations are not equivalent.

\medskip
The cases with anaphoric relations from the first conjunct into the second conjunct are not licensed by the CIC as stated in \refer{SemATB}.
The universal quantifier in \refer{bi-every-ana2} is contributed inside the first conjunct daughter only, yet it has scope over the entire conjunction. 
We think that this type of wide scope is, nonetheless, an instance of the ATB exception, as it is only possible if the quantifier binds a variable in the second conjunct. 
Seen this way, the second conjunct does contribute some part of the operator that takes wide scope, namely the variable that it binds. 
To have a general term for this, we will define the notion of \emph{anchoring} as in \refer{def-anchor}.

\ea \label{def-anchor}
A semantic expression $A$ is \emph{anchored} in a constituent $c$ iff it is contributed by $c$ or it binds a variable that is contributed in $c$.
\z 

The more tolerant version of the ATB exception that we are going to pursue in this section has been put forward in  \citet{Fox:95} and \citet{Sauerland:03}. 
They assume that raising a quantifier is possible out of the first conjunct when it binds a trace in the first conjunct and a variable in the second conjunct.%
\footnote{Both of these publications mention \citet{Ruys:93} as the original source of this generalization. Unfortunately, we were not able to get hold of a copy of that work.} 
The following can be considered an LRS adaptation of their proposal.


Before we can state the final version of the ATB exception, we need to address a technical issue: 
in existentially split readings, the overall quantifier is only contributed by one of the conjuncts, and so is everything in its restrictor -- $\co{actress}(y)$ in our example. 
This can be seen as an instance of \emph{semantic pied-piping}, i.e., the expression $\co{actress}(y)$ may occur outside the representation of the conjunct daughter in which it is contributed because it is the restrictor of the quantifier that takes wide scope.

To allow for semantic pied-piping and to restrict it at the same time, we introduce the notion of \emph{contributionally closedness up-to}, defined in \refer{def-closed-upto}. 
This notion allows us to refer to a set of semantic contributions that form a contingent expression with a potential hole in it. 


%\marginpar{MS: brauche großes Phi! Warum gehen die griechischen Großbuchstaben nicht?}
\vbox{
\ea Contributional closedness up-to:\label{def-closed-upto}\\
For each set of expressions $\Phi$ and each expression $\psi$,
$\Phi$ is \emph{contributionally closed up-to} $\psi$, ${\Phi} / \psi$, iff 
there exists an expression $\phi$ such that 
every subexpression of $\phi$ is an element of $\Phi$ or a subexpression of $\psi$, and $\psi$ and every element of $\Phi$ is a subexpression of $\phi$. 
\z 

}


The usefulness of this definition for our analysis is clear when we look at our analysis in \refer{bi-every-ana2}, repeated in \refer{ex-phi}.
Contributional closedness up-to allows us to separate the semantic representation into two parts: 
the representation of the conjunction, which is the expression $\psi$ from the definition, given in \refer{ex-psi}, and the contributions for the first conjunct that occur outside of the conjunction, i.e.\@ the set $\Phi$ from the definition, which is stated as a meta-expression in \refer{ex-PHI}.


\ea
\begin{xlist}
\ex 
$\forall y (\co{actress}(y) :$\\
\hspace*{\fill}
$(\exists x (x = y : \co{adore}(\co{alex},x)) \land \exists x (\co{film}(x,y) : \co{adore}(\co{alex},x)))$ \label{ex-phi}
%
\ex $(\exists x (x = y : \co{adore}(\co{alex},x)) \land \exists x (\co{film}(x,y) : \co{adore}(\co{alex},x)))$
\label{ex-psi}
\ex $\forall y (\co{actress}(y) : A)$
\label{ex-PHI}
\end{xlist}
\z 

In this example, the set $\Phi$ contains the following expressions: the variable $y$, the constant \co{actress}, the formula $\co{actress}(y)$, and the quantified expression $\forall y (\co{actress}(y): A)$. 
%Let us illustrate contributional closedness up-to for one of them, the formula $\co{actress}(y)$.
The overall formula in \refer{ex-phi} is the expression $\phi$ from the definition. 
All its subexpressions are either in \refer{ex-PHI} or in \refer{ex-psi}.
Consequently, the expressions that are outside the conjunction constitute a set that is contributionally closed up-to the conjunction.

The universally quantified expression $\forall y (\co{actress}(y) : A)$ is not only contributionally closed up-to the conjunction in \refer{ex-phi}, it is also anchored in the second conjunct, because \bsp{her} in the second conjunct daughter also contributes the variable, $y$, which is bound by the universal quantifier.
%
% \kommentar{
% The universal quantifier binds a variable in the second conjunct, $\exists x (\co{film}(x,y) : \co{admire}(\co{alex},x))$. This means that the quantified expression is anchored in the second conjunct. 
%  %
%  This shows that the expressions that end up outside the conjunction are not arbitrary but that they are contributionally closed up-to the conjunction and that they contain at least one contribution that is anchored in both conjuncts.
% }
We think that these are exactly the two constraints determining when a semantic ATB exception is possible.

We can use the notions of anchoring and contributionally closedness up-to in our final formulation of 
the CIC in \refer{SemATB2}. 

\vbox{
\ea 
Conjunct Integrity Constraint with Semantic ATB exception (CIC, final version)\label{SemATB2}\\
%In a conjunction phrase with an internal content of the form $F \land G$,\\
%there exist expressions $U$ and $V$ contributed by the first and the second conjunct respectively, such that $F$ is just like $G$ with $U$ instead of $V$.
%\begin{itemize}
%\item the external content of the first conjunct is in $F$, and the external content of the second conjunct in $G$,
%
%\item %there exist expressions $U$ and $V$ contributed by the first and the second conjunct respectively, such that $F$ is just like $G$ with $U$ instead of $V$,
%For each expression $H$ which occurs only in one conjunct, $H$ is contributed by the conjunct daughter linked to that conjunct, or 
%there is an expression $H'$, where either $H$ or $H'$ is contributed by the conjunction particle, such that for some $J, J'$,  $H/J \sim H'/J'$.
%\item 
For each $H$ contributed by one conjunct daughter, 
\begin{itemize}
    \item  $H$ must not occur in a conjunct in which it is not contributed,
    \item $H$ may only have scope over the overall conjunction if it is anchored in the other conjunct daughter as well, and
    \item $H$ may only occur outside the conjunction if it is part of some subset of the contributions of its conjunct that is contributionally closed up-to some formula that contains the conjunction.
\end{itemize}
%$H$ occurs inside this conjunct,
%TO BE ADDED LATER FOR ATB: 
%alt: or $H$ is contributed by the other conjunct as well and $H[F\land G]$. 
%\end{itemize}
\z 

}

In this reformulation, we no longer require the wide-scope element to be contributed in both conjuncts. It is enough if it is anchored in the sense defined in \refer{def-anchor}.
The semantic representations with existential split discussed in this section satisfy this final version of the \CCB.

\bigskip%
\begin{sloppypar}
  Existential split also applies to the mono-propositional analysis.
  We can change our running example to enforce a mono-propositional
  reading.
\end{sloppypar}
\ea \label{ex-every-ana2}
Every actress$_i$ and one of her$_i$ fans met right after the premiere.
\z 

The semantic representation of this sentence is given in \refer{mono-every-ana2}.
The universal quantifier takes intermediate scope between the existential quantifier over the discourse referent of the conjunction, $\exists z(\ldots)$, and the conjunction.%
\footnote{To allow for this additional universal quantifier, we need to allow that the conjunction, $F \land G$, is not an immediate subterm of the restrictor of $\exists z$.
The necessary lexical specification is given in \refer{le-mono-and-split}, where $F'$ is a new metavariable that indicates the possibility of additional material taking scope over the conjunction.

\ea \label{le-mono-and-split}
\bsp{und}: $\exc{\exists \drr{z} (F'[\inc{(F[\noc{x} \in \pI z]
\land G[\noc{y} \in \pII z])}]
: H[z])}$
\z 

}

\vbox{
\ea \label{mono-every-ana2}
Mono-propositional representation of example \refer{ex-every-ana2}:\\[1ex]
$\exists z (
\forall y (\co{actress}(y) : $\\
\hspace*{\fill}$(\exists x (x = y : x \in \pI z)
\land
\exists v (\co{fan-of}(v,y) : v \in \pII z)
)) : \qquad\qquad$\\
\hspace*{\fill}$\co{meet}(z))$
\z 
}

% \kommentar{%raus, altes Bsp; kann keine Monoprop haben, weil distributiv
% A hypothetical mono-propositional analysis of \refer{ex-every-anaA} is given in \refer{mono-every-ana}. The problem of this representation is that the variable $x$ in the second conjunct cannot be bound by the universal quantifier from the first conjunct.

% \ea \label{mono-every-ana}
% Hypothetical mono-propositional representation of example \refer{ex-every-anaA}:\\[1ex]
% $\exists z ((
% \forall x (\co{actress}(x) : x \in \pI z)
% \land
% \exists y (\co{film-of}(y,x) : y \in \pII z)
% ) : $\\
% \hspace*{\fill}$\co{adore}(\co{alex},z))$
% \z 
% }

The pronoun \bsp{her} can now be bound by \bsp{every French actress} as the universal quantifier has wide scope over the conjunction. % 
%This solves the problem of the representation in \refer{mono-every-ana}.
%
The mentioned intermediate scope of the universal quantifier in  \refer{mono-every-ana2} seems to be obligatory. 
In particular, it cannot take scope over $\exists z (\ldots)$. 
This cannot follow from the lexical specification of the mono-propositional conjunction particle, as our analysis of the standard CNNP cases relies heavily on the possibility that material from inside conjuncts can take wide sope over the group/pair individual $z$. 
Consequently, it must be a constraint on the existential split, i.e., there needs to be a constraint on how close the added wide-scope quantifier and the embedded existential quantifier $\exists x (x=y: \ldots)$ must be. 







%SAG WAS ZUM CPC MIT ATB
\medskip%
The analysis outlined above predicts the availability of anaphoric relations between the two conjuncts. 
At the same time, we also predict the contrast between noun phrase conjunction and clausal conjunction in \refer{ex-every-ana}.
In a mono-clausal syntactic analysis, we expect that a universal quantifier can have scope over the rest of the clause. 
In a bi-clausal syntactic structure, no such wide scope is possible, and cross-sentential dynamic effects are excluded by the non-dynamicity of the universal quantifier.
To achieve this, we 
adjusted the \CCB{} in such a way that we allow for semantic ATB exceptions in the case of binding.

% \kommentar{
% The existential split of the allowed us to maintain our original hypothesis that the two conjunct daughters have the same discourse referent marker in a bi-propositional analysis.
% We showed that a syntactically bi-clausal analysis would not be able to capture the critical examples with a bound variable in the second conjunct as we find binding into the second conjunct also for quantifiers that do not allow for dynamic, cross-sentential binding such as the universal quantifier. 
% The existential split also gives us the right readings for the mono-propositional case.
% }


\subsection{Anaphoric relations in CNNP}
\label{Sec-Anaphor-Neg}

%\bigskip%
\begin{sloppypar}
  Some speakers reject all anaphoric links between the conjuncts in
  CNNP constructions, others accept them when a bi-propositional
  reading is available. Those speakers who have difficulties getting
  mono-propositional readings in the first place, find such readings
  even less acceptable if there is an anaphoric relation between the
  two conjuncts.  Finally, some speakers have no problem with
  anaphoric relations under any of the readings.  These three distinct
  judgement patterns are shown in \refer{ex-no}, where the first
  sentence illustrates a bi-propositionally interpretable structure,
  the second sentence an only mono-propositionally interpretable case.
\end{sloppypar}


\vbox{
\ea \label{ex-no}
\begin{xlist}
%\ex Alex interviewed an actress and her manager.\label{student-supervisor}
%\ex Alex interviewed every actress and her manager.
%Alex interviewed every actress and Alex interviewed her manager.
\ex [*/ \footnotesize{ok}/ \footnotesize{ok}]{
\gll Alex mag [keine französische Schauspielerin]$_i$ und 
[keinen ihrer$_i$ Filme]$_j$.\\
Alex likes \hphantom{[}no French actress and \hphantom{[}none {of her} films. 
\label{ex-no-ana}\\
}
\label{ex-no-actress}
%\ex [*]{Alex likes [no French actress]$_i$ and Alex likes [none of her$_i$ films]$_j$.}\label{ex-no-ana-french}
%\end{xlist}
\ex[*/ */ \footnotesize{ok}]{\label{ex-no-ana-compare}
%No French actress and none of her fans met.
\gll Alex vergleicht [kein Buch]$_i$ und [keine seiner$_i$ Verfilmungen]$_j$ (\reci{miteinander}).\\
Alex compared \hphantom{[}no book and \hphantom{[}none {of its} {movie renderings} {(with each other)} \\
% Alex hat keinen Roman und keine seiner Verfilmungen miteinander verglichen.
}
\end{xlist}
\z 

}

We will discuss the two readings separately. 
%As will become clear in the discussion, the availability of the two readings is independent of one another. 
The fact that anaphoric relations appear to be less available for the mono-propositional reading meshes well with the overall tendency that the mono-propositional reading is less easily accessible than the bi-propositional one.

%We will not discuss the intermediate type of speaker here. They may simply lack the mono-propositional conjunction particle or disallow it in the scope of negation. As for the other two judgment patterns, we will first describe the most restrictive judements and, then the most permissive ones.


Let us first consider hypothetical bi-propositional analyses of sentence \refer{ex-no-ana}, given in \refer{bi-no-ana2} and \refer{bi-no-ana}. 
If the existential quantifier of the first conjunct takes wide scope over the conjunction, so must its negation. 
For this to be possible, the negation contributed inside the second conjunct daughter has to take wide scope over the conjunction as well, to be a semantic ATB exception.

In \refer{bi-no-ana2}, however, the negation contributed by the second conjunct daughter is part of the second conjunct in the semantic representation. 
Consequently, there is a violation of the \CCB.

\vbox{
\ea 
Hypothetical bi-propositional analysis of \refer{ex-no-actress}, first option\label{bi-no-ana2}\\[1ex]
\# $\lnot \exists y (\co{actress}(x) : $\\
\hspace*{\fill} $(\exists x (x=y : \co{like}(\co{alex},x)) \land
\lnot \exists x (\co{film}(x,y) : \co{like}(\co{alex},x))))$
\z 

}

The semantic representation in \refer{bi-no-ana} respects the \CCB. Here, there is only one negation, which takes scope over the entire conjunction. 

\vbox{
\ea Hypothetical bi-propositional analysis of \refer{ex-no-actress}, second option\label{bi-no-ana}\\[1ex]
\# $\lnot \exists y (\co{actress}(x) : $\\
\hspace*{\fill} $(\exists x (x=y : \co{like}(\co{alex},x)) \land
\exists x (\co{film}(x,y) : \co{like}(\co{alex},x))))$
\z 

}


This formula represents a reading in which there is no actress such that Alex likes both her and some of her films. 
This would leave the option that Alex likes some French actress, but just not her films. 
This is, however, not a possible reading of the sentence, and our principles correctly exclude it: 
as in \refer{ex-brief-mail-biWide}, this formula cannot express a sentential negation because there is a logical connective intervening between the internal content of the verb, $\co{like}(\co{alex},x)$, and the negation.
Consequently, there is no well-formed bi-propositional analysis of example \refer{ex-no-ana}.

This raises the question what interpretation those speakers have who accept sentence \refer{ex-no-actress}. 
We will argue that the pronoun in the second conjunct in \refer{ex-no-ana} is not bound by the quantifier from the first conjunct. 
This argument is parallel to the argumentation for e-type pronouns in \citet{Evans:77,Evans:80}.
First, if the pronoun in the second conjunct were bound, we would get a reading like \refer{bi-no-ana}. But such a reading is not available for the sentence.
%i.e., where Alex could like an actress or some of her films, but just not both. 
%However, the sentence does not have such a reading.
%
% \kommentar{%MS: raus, 4.7.19, Grund: irrelevant
% This can be shown with negative determiners that allow for some exceptions, as in \refer{ex-ana-sogutwie}. 

% \ea \label{ex-ana-sogutwie}
% \gll Alex mag {so gut wie} keinen französischen Schauspieler und keinen einzigen seiner Filme.\\
% Alex likes almost no French actor and no {single one} {of his} films\\
% \glt \mytrans{Alex likes almost no French actor and not a single one of his films.}
% \z 
% %
% }
%
Second, we require that for all disliked actresses, all their films are also disliked, not just some.
The pronoun in the second conjunct in \refer{ex-no-ana} is interpreted with respect to the set of disliked actresses, i.e.\@ to what is called the RefSet in the literature on cross-sentential anaphora such as \citet{Nouwen:03} or \citet{Luecking:Ginzburg:19}.
%\footnote{Sentence \refer{ex-ana-sogutwie} may also have a reading in which the possessive pronoun refers to all French actors, even those liked by Alex. This would be reference to the so-called maximal set (or MaxSet). Such a reference to elements from the MaxSet is also expected.}
Consequently, we can give a paraphrase for the sentence in which the possessive pronoun is replaced with a definite noun phrase, see \refer{para-ana-sogutwie}.

\ea \label{para-ana-sogutwie}
Alex mag keine französische Schauspielerin und 
keinen Film [von den französischen Schauspielerinnen, die Alex nicht mag].
\glt \mytrans{Alex likes no French actress and no movie [of the French actresses that Alex doesn't like].}
\z 

We cannot propose a treatment of this type of pronouns in this paper, but the resulting semantic representation of a sentence like \refer{ex-no-ana} could look as in \refer{lf-etype}.

\ea \label{lf-etype}
Sketch of an e-type analysis of \refer{ex-no-ana}:\\
$\lnot \exists x (\co{actress}(x) : \co{like}(\co{alex},x)) \land
\lnot \exists x (\co{film}(x,Y) : \co{like}(\co{alex},x))$, 
\\where $Y$ is the set
$\lambda x. (\co{actress}(x) \land \lnot \co{like}(\co{alex},x))$.
\z 

The important aspect of this representation is that the variable management for our analysis of bi-propositional readings is not problematic. The discourse referent marker in both conjuncts is the same variable, $x$. 
The overall sentence is negated as its internal content, $\co{like}(\co{alex},x)$ is in the scope of negation with no intervening connective. 
Finally, the possessive pronoun is interpreted as referring to the RefSet, $Y$, i.e., to the set of all actresses that Alex does not like.

We showed that there is a difference between the cases with real binding into the second conjunct and the cases of more discourse-like pronouns in CNNP. 
We have to leave for future research the reasons for why many speakers do not easily get the last kind of reading.

% \kommentar{
% \noindent%
% In \refer{ex-ana-sogutwie}, there could be individual actresses that Alex likes, but it is clear that Alex just does not like any of the films of the actresses Alex does not like.


% \ea
% Alex mag keine französische Schauspielerin und keinen ihrer Filme.
% Alex mag keine französiche Schauspielerin und Alex mag keinen ihrer Filme.
% \z 
% }

%In examples of this type, the first noun phrase always presupposes a non-empty set, i.e., there is a set of French actresses in our example. 



\medskip%
Next, we turn to \refer{ex-no-ana-compare}, a sentence in which the conditions for a mono-pro\-po\-si\-tional reading are met.
The mono-propositional analysis of the sentence is shown in \refer{mono-no-ana}. 

\vbox{
\ea Mono-propositional analysis of \refer{ex-no-ana-compare}\label{mono-no-ana}\\[1ex]
$\lnot \exists z (\exists y (\co{book}(y) : $\\
\hspace*{3em}$(\exists x (x = y : x \in \pI z) 
\land \exists v (\co{film-rendering}(v,y) : v \in \pII z)
) $\\
\hspace*{\fill}$: \co{compare}(\co{alex},z)))
)$
\z 

}

\begin{sloppypar}
  This semantic representation meets the \CCB{}.  Each of the
  conjoined noun phrases contributes a negation, so the negation can
  take wide scope over the overall conjunction as a semantic ATB
  exception.
\end{sloppypar}

Similarly, the quantifier which we get by the existential split, $\exists y (\co{book}(y) : \ldots)$ in \refer{mono-no-ana} is anchored in both conjuncts: it is contributed in the first and binds a variable, $y$, in the second.
Given this constellation, there should be no problem with the reading in \refer{mono-no-ana}, i.e., our constraints are formulated in such a way that binding into the second conjunct should be possible in a mono-propositional reading.


%At present, we can only stipulate about the reasons for the unacceptability of the indicated reading. 
%
Since not all speakers accept this constellation, we will show how it can be excluded.
One difference between this reading and the earlier, well-formed, examples of split readings is that the two elements that take scope over the entire conjunction are separated from each another. 
In the present example, the existential quantifier $\exists z (\ldots)$ intervenes between the negation and $\exists y (\ldots)$.
If this reasoning is on the right track, all elements from inside individual conjuncts that take scope over the entire conjunction need to form a \emph{contributionally closed up-to} constellation. 
In other words, the speakers who do not accept \refer{mono-no-ana} have a stricter version of the last clause of the \CCB{} from \refer{SemATB2} in which all conjunct-internal contributions that take wide scope need to 
to be part of a single set which is contributionally closed up-to the conjunction.
Such a formulation is given in \refer{cic-last-strict}.

\vbox{
\ea Strict version of the last clause of the \CCB{}:\label{cic-last-strict}
\begin{itemize}
    \item 
all $H$ that occur outside the conjunction are part of some
subset of the contributions of their conjunct that is contributionally
closed up to some formula that contains the conjunction.
\end{itemize}
\z 
}

With this formulation, the semantic representation in \refer{mono-no-ana} is excluded.
There are two hypothetical representations that would not violate this constraint: one in which $\exists y (\ldots)$ takes scope over $\exists z (\ldots)$, and one in which the negation takes narrow scope inside the restrictor of $\exists z (\ldots)$. These two constellations are sketched in \refer{NotExEx} and \refer{ExNotEx}, respectively.
We show that they violate other constraints.


\ea
\begin{xlist}
\ex 
$\lnot \exists y (\co{book}(y) : \exists z (\exists x (x=y: x \in \pI z) \land \exists v (\co{film}(v,y) : v \in \pII z)$\\ 
\hspace*{\fill}$ : 
\co{compare}(\co{alex}, z)))$\label{NotExEx}
\ex
$\exists z (\lnot \exists y (\co{book}(y) : 
(\exists x (x=y : x \in \pI z)
\land \exists v (\co{film}(v,y) : v \in  \pII z) 
))$\\ 
\hspace*{\fill}$: \co{compare}(\co{alex},z))$\label{ExNotEx}
\end{xlist}
\z

In  \refer{NotExEx}, the quantifier contributed in the first conjunct takes wide scope over the group/pair object $z$. 
In our discussion below \refer{mono-every-ana2}, we argued that such a constellation should be excluded on independent grounds by a -- yet to be defined -- constraint on what material may intervene between the two quantifiers contributed by  an existentially split determiner.

In the representation in \refer{ExNotEx}, the negation is in the restrictor of the quantifier over the pair individuals.
Consequently, the internal content of the verb is not in the scope of negation, the sentence is not negated. 

\medskip%
We have discussed possible bi- and mono-propositional analyses of CNNP with anaphoric relations between the two conjuncts. 
We showed that there cannot be proper binding in the bi-propositional analysis. 
To the extent that some speakers can interpret such sentences, the pronoun in the second conjunct is not bound by the negative indefinite in the first conjunct but refers to the RefSet established in the first conjunct. 
For the mono-propositional analysis, the situation is different: there are two versions of the \CCB, the weaker version in \refer{SemATB2}, and the stronger version in \refer{cic-last-strict}. 
Speakers with the weak version accept CNNPs with an anaphoric relation, speakers with the strong version do not.

Since the mechanisms for the bi- and the mono-propositional readings are independent of each other in our analysis, it is possible that some speakers accept the mono-propositional case but not the bi-propositional one. 
However, mono-propositional readings are less easily available even in the absence of anaphoric relations. 
Therefore, we are not surprised that we have not yet found a speaker accepting an anaphoric relation in the mono-propositional case but not in the bi-propositional case.


\bigskip%
LRS allows for using the same semantic material in different contexts. 
We showed that assuming identical discourse referent markers for conjoined noun phrases in a bi-propositional reading is compatible with syntactic configurations in which a pronoun in the second conjunct is anaphorically related to the first conjunct. 


% \kommentar{%MS: raus, 4.7.19. Grund: geht jetzt einfacher
% \ea 
% Scopal chain\\
% Expressions $\alpha_1, \ldots \alpha_n$ form a \emph{scopal chain} iff
% \begin{enumerate}
%     \item there exists an expression $\beta$ that is a component of each $\alpha_i$ in the chain, 
%     \item for each $\alpha_i, \alpha_j$ in the chain, if either $\alpha_i$ is in the scope of $\alpha_j$ or the other way around,
%     \item for each $\alpha_i, \alpha_j$ such that $\alpha_i$ is in the scope of $\alpha_j$, any expression that is in the scope of $\alpha_j$ but takes scope over $\alpha_i$ is an element of the chain as well.
% \end{enumerate}
% \z 


% \ea Semantic compaction constraint\\
% Scopal expressions $\alpha_1, \ldots, \alpha_n$ which are contributed by a conjunct daughter $A$ and which take scope over the overall conjunction form a scopal chain.


% \z 
% }

% \kommentar{%MS: alte Version, raus 25.6.19
% This is in contrast with the \refer{ex-no-act-film-cnnp}, an ordinary case of CNNP, with no anaphoric relation between the conjuncts, for which a bi-clausal paraphrase can be given, as in \refer{ex-no-act-film-bi}

% \ea \label{ex-no-act-film}
% \begin{xlist}
% \ex[]{
% \label{ex-no-act-film-cnnp}
% Alex likes [\NE{no} French actress] and [\NE{no} French film].}
% \ex[$=$]{Alex likes \NE{no} French actress and Alex likes \NE{no} French film.\label{ex-no-act-film-bi}}
% \end{xlist}
% \z 

% Deutsche Daten mit anderem Szenario zum Vergleich:
% \ea \label{ex-de-schule}
% \begin{xlist}
% \ex Ich habe die Schule immer gemocht: Ich habe keinen Lehrer und kein Fach jemals wirklich gehasst.\\
% \mytrans{I always liked shool: I never really disliked any teacher and  her subjets;  }
% %
% \ex \%Ich habe die Schule immer gemocht: Ich habe keinen Lehrer und keines seiner Fächer, jemals wirklich gehasst.
% \end{xlist}
% \z 



% The data in \refer{ex-some-actress}--\refer{ex-no-ana} seem to be challenging for our assumption of having identical discourse referent markers for the conjuncts, which we are pursuing here to allow for a surface-oriented syntactic analysis. 
% We showed that they are equally problematic for an analysis that tries to derive noun phrase conjunction from an underlying clausal conjunction.

% }


%\marginpar{DAS ALLES VERSCHIEBEN, VERSTÖßT AUCH GEGEN CIC!!!}
%\marginpar{MS: Hedde: ist Dr-Identität nicht ein zu hoher Preis, nur um eine Oberflächensyntax zu haben?}


% \kommentar{%MS raus, 17.5.19
% % Grund: versucht, das noch positiver zu formulieren
% allows us, however, to maintain a surface-oriented syntactic analysis, i.e., we can analyze sentences with conjoined noun phrases as simple sentence and still derive a bi-propositional reading. Providing all conjuncts with the same discourse referent makes it possible to  use the ordinary semantic combinatorics and linking when the conjoined NPs combine with the verb. 
% }




\section{Conclusion}
\label{Sec-Conclusion}

\begin{sloppypar}
  We presented a first systematic analysis of conjoined negative noun
  phrases (CNNP), a phenomenon whose discussion has previously been
  restricted to side remarks or footnotes.  Our analysis combines an
  existing analysis of negation with a negation-independently
  developed analysis of coordination.  In other words, we did not need
  any CNNP-specific assumptions.
\end{sloppypar}
It is an important property of CNNP that its readings do not seem to differ between NC and non-NC languages.
We attributed this to the fact that coordination is subject to an across-the-board constraint, which is a cross-linguistically robust property of coordination. 
The NC/non-NC distinction is argued to be based on interpretation constraints that are not at work in coordination.



% \kommentar{
% \begin{itemize}
% \item New data on negative conjuncts
% \item Follows from independent analysis of negation and conjunction.
% \item SemATB is part of a general ATB principle for coordination.
% \item SemATB is cross-linguistically more robust than the interpretation strategy of n-constituents in non-coordinated structures.
% \item CNNP is not equivalent to negation with conjoined indefinites\\
% $\Rightarrow$ Negation is part of the conjunct's semantic contribution.
% \end{itemize}
% }


A constraint-based system of semantic combinatorics such as LRS proved to be apt for modeling the data. LRS is inherently one-to-many friendly. The constraint-based perspective allows a fresh view on the semantic contributions of lexical items and on interpretation strategies at the phrasal level: by using a particular lexical item, a speaker constrains the semantic representation to contain some constants, variables, etc; by using a particular syntactic construction, the speaker constrains the way in which these pieces of our semantic representation language are combined. 
The first property makes it very natural to assume that several lexical items require the same semantic constant or operator to occur in the semantic representation. 
The second property shows that LRS treats ambiguity as the norm rather than the exception and, at the same time, emphasizes the role of syntax and of general interpretation strategies to reduce the amount of ambiguity.

We motivated the Semantic Across-the-Board Constraint by its analogy to ATB constraints in syntax. Its syntactic analogue has been shown to be reducible to an independent semantic effect in \citet{Chaves:12}. The same will, hopefully, be true for the version presented in this paper.
In its current version, it provides a good starting point for further research in this direction.
%We hope that this  contributes to our understanding of coordination and that, eventually, it will be shown that either all ATB constraints follow from a more general underlying property of coordination or that they need to be parametrized as cross-linguistically varying constraints on the autonomy of individual conjuncts. 
As it stands, it seems to us that the \SemATB{} is valid in both NC and non-NC languages.


In our analysis, we treat CNNP as a residual syntactic construction in StG that requires a negative-concord style interpretation. The relatively clear reading judgements on CNNP help us make this point here. 
In their work on French, \citet{Burnett:al:15} show that while French is an optional NC language, there are preferences for particular readings depending on the syntactic constellation, the context, but also on some extra-linguistic properties. A constraint-based system like  LRS will allow us to derive all possible readings and, at the same time, to formulate empirically motivated constraints to exclude readings or to impose strong contextual conditions on readings.


% \kommentar{
% \begin{itemize}
% \item LRS is inherently one-to-many friendly.
% %\item Bi-propositional CNNP readings are one-to-many: multiple use of the verb's semantics.
% \item Mono-propositional CNNP readings are one-to-many: multiple contribution of a single negation.

% \item Even non-NC languages allow for a many-to-one interpretation of negated constituents.
% \item Variation in other syntactic constructions \citep{Burnett:al:15}, even within one language.
% \item Constraining one-to-many:
% %\begin{itemize}
% %\item 
% Validity of (Sem)ATB reduces cross-linguistic variation in SN/DN ambiguities.
% %\item Conjunction Confusion Blocking 
% %\end{itemize}
% \end{itemize}

% }



\section*{Abbreviations}

  \begin{tabular}{ll}
    \lsptoprule
    Abbreviation & Full form\\
    \midrule
    ATB & Across-the-Board\\
    CIC & Conjunct Integrity Constraint\\
    CNNP & conjunction of negative noun phrases\\
    CPS & Coordination Parallelism Constraint\\
    CSC & Coordinate Structure Constraint\\
    DN & double negation (reading)\\
    HPSG & Head-driven Phrase Structure Grammar\\
    LF & Logical Form\\
    LRS & Lexical Resource Semantics\\
    % MRS & Minimal Recursion Semantics\\
    NC & negative concord\\
    NM & negative marker (French \bsp{ne}, Polish \bsp{nie})\\
    n-word & negative indefinite determiner or pronoun\\
    SN & single negation (reading)\\
    StG & Standard German\\
    TLF & Transparent Logical Form\\
    \lspbottomrule
  \end{tabular}

%\begin{table}[h]
%\caption{Abbreviations}
%\end{table}


\section*{Acknowledgements}

The research for the paper was initiated as part of the networking grant \emph{One-to-many relations in morphology, syntax, and semantics} of the \textit{Deu\-tscher Akade\-mi\-scher
  Austauschdienst} (DAAD).
We gratefully acknowledge the funding received through this grant. We received important feedback from the audience of the DGfS workshop  (\textit{Arbeitsgruppe} 4) One-to-Many Relations in Morphology, Syntax, and Semantics (Stuttgart, March 2018). 
We would also like to thank the reviewers for their remarks and comments.
We are grateful for additional input from Berthold Crysmann, Nicolas Lamoure and Ewa Trutkowski.
All errors are ours.


\printbibliography[heading=subbibliography,notkeyword=this]

\end{document}

%%% Local Variables:
%%% mode: latex
%%% TeX-master: "../main"
%%% End:
