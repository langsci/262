\documentclass[output=paper]{langsci/langscibook}
\ChapterDOI{10.5281/zenodo.4729808}

\title{Modification of literal meanings in semantically non-decomposable idioms}

\author{Sascha Bargmann\affiliation{Goethe-Universität Frankfurt a.M.} and Berit Gehrke\affiliation{Humboldt-Universität zu Berlin} and Frank Richter\affiliation{Goethe-Universität Frankfurt a.M.}}

\abstract{In the literature on idioms, conjunction modification is understood as involving a modifier that does not lexically belong to the idiom at hand, modifying the literal meaning of a noun in that idiom while the idiomatic meaning of the expression as a whole is preserved. The construction relies on the hearer perceiving the idiomatic meaning of the whole and the literal meaning of a part of it simultaneously and in conjunction. We investigate instances of naturally occurring examples of four semantically non-decomposable verb-phrase idioms (two English, two German) whose complements contain such a modifier. We examine the possible interpretations and the contextual conditions of these idiom-modifier combinations. They are particularly interesting instances of one-to-many relations between form and meaning.}

\begin{document}
\maketitle

\section{Introduction}
In any comprehensive investigation of one-to-many relations between
form and meaning, there is no way around idioms. In nearly all cases,
the string that can be interpreted as an idiom (e.g.\@ \textit{pull
  $x$'s leg} $\rightsquigarrow_{id}$ `playfully deceive $x$') can also
be interpreted literally (\textit{pull $x$'s leg} $\rightarrow_{lit}$
`pull $x$'s leg'), so that one and the same string provides several
meanings. This becomes especially obvious in so-called conjunction
modification \citep{ernst81}, in which a modifier inserted into the
nominal complement of a verb-phrase idiom modifies the literal meaning
of the noun, while the idiom as a whole is still understood in its
idiomatic meaning (\textit{pull $x$'s} \underline{\smash{tattooed}}
\textit{leg} $\rightsquigarrow_{id}$ `playfully deceive $x$' and
$\rightarrow_{lit}$ `$x$ has a tattooed leg').\footnote{Here and in
  the following, we italicize those words that belong to the idiom,
  underline the modifier(s), and put single quotation marks around the
  meaning representations, which we state informally by means of
  natural language (English) expressions.}
The perceived interpretation of the resulting expression requires both the idiomatic meaning of the idiom and the literal meaning of the idiom's noun.

Overall, \citet{ernst81} distinguishes three types of modification in what he calls ``extraneous" modifiers in idioms (i.e.\@ modifiers that are not part of the idiom itself): internal modification, external modification, and conjunction modification.\footnote{As far as we know and as \citet[83]{stathi07} states as well, \citet{ernst81} is the first to systematically look into modification in idioms. Since our purpose is mainly to study naturally occurring data, rather than to provide a complete account, we will not discuss other, more recent papers on modification \citep[see, for instance,][]{stathi07, cserep10, mcclure11, sailer17}.}
The aim of this paper is to explain this tripartite division of idiom modification and then to focus on conjunction modification and corpus examples that fall into this category. As our discussion will show (and as \citealt{ernst81} already emphasizes as well), it is not always uncontroversial %, however, 
which one(s) of the three categories of idiom modification a specific example falls into. Such complications might ultimately lead to a revision of Ernst's characterizations of the three classes or to a different theory of idiom modification altogether. With our present discussion, we want to contribute to a better understanding of the empirical situation as a necessary foundation to such a revised theory.

The paper is structured as follows. First, we will give a short introduction to Ernst's tripartite division of idiom modification (\mbox{Section \ref{Ernst 81}}). We will then zoom in on conjunction modification and present corpus data on two English and two German semantically non-decomposable verb phrase idioms with the meaning `die' (\textit{kick the bucket}, \textit{bite the dust}, \textit{den Löffel abgeben} `(lit.) pass on the spoon', and \textit{ins Gras beißen} `(lit.) bite into the grass') that include an extra modifier. %Their analysis will not always be unanimous
We did not always agree on how these idiom-modifier combinations are to be analyzed (\mbox{Section \ref{Zoom ConjMod}}). Before we conclude our paper (\mbox{Section \ref{Conclusion}}), we will point to some idiom examples beyond modification that nonetheless seem to be analyzable in a similar way %as 
to conjunction modification (\mbox{Section \ref{Beyond Mod}}).

Our discussion of semantic interpretation will remain mostly nontechnical, although we have a suitably expressive logical language in mind for semantic representations when we explicate the meaning of our examples in English paraphrases. How these representations are to be built from the representations of words, or how the representations of larger semantically non-decomposable idioms enter the semantic composition mechanism, is an important question, but it is not the focus of the present discussion. Only with an explicit system that answers these questions and governs a precise semantic composition mechanism could we begin a serious investigation of issues concerning compositionality, which are regularly and naturally raised in connection with the analysis of idioms.

When we use the term \textit{compositionality} here, it is meant as a broad reference to a semantic composition operation that starts from simple or phrasal lexical units (the latter being possibly necessary for semantically non-decomposable idioms) and constructs the representations of larger units from them, conditional on syntactic structure. When we say for some examples, following common parlance, that we do not know how to analyze them compositionally, this means that we are unsure how to spell out a composition operation in this sense in full detail. It is not to be understood as a technical statement about the relationship between the syntax and semantic composition mechanism(s) of the grammar framework of choice in which the operation would have to be expressed.\footnote{Two authors of the present paper have a preference for a constraint-based semantics in HPSG for which compositionality in the traditional sense does not hold, although it formulates a precise \textit{systematic} relationship between syntactic structure and semantic interpretation.}



\section{Ernst's tripartite division of idiom modification} \label{Ernst 81}

According to \citet{ernst81}, modification in idioms is -- at least in principle -- three-way ambiguous between external modification, internal modification, and conjunction modification. Context and world knowledge narrow down the interpretative options that the semantics provides on the basis of the combination of the meaning of the modifier and the meaning of the idiom. 
 
If an idiom has \textit{internal semantic structure} in the sense that its ``particular words [...] correspond to specific independent elements in the idiom's semantic representation'' \citep[67]{ernst81}, as in \textit{pull strings} \mbox{($\rightsquigarrow_{id}$ `use connections')} or \textit{jump on the bandwagon} ($\rightsquigarrow_{id}$ `join a movement'), the idiom allows for all three modification options. Following \citet{Nunberg:al:94}, we call such idioms \textit{semantically decomposable}. If, by contrast, the idiom has no internal semantic structure, as in \textit{kick the bucket} ($\rightsquigarrow_{id}$ `die') or \textit{tighten one's belt} \mbox{($\rightsquigarrow_{id}$ `economize')}, internal modification is impossible. These idioms we call \textit{semantically non-decomposable}.\footnote{It is important to note at this point that the semantic decomposability of an idiom cannot be proven by simply finding a paraphrase for the idiom in which each word corresponds to exactly one of the words of the idiom. In order to show that an idiom is semantically decomposable, i.e. that the idiom's meaning disseminates over its words in such a way that each of these words receives a meaning component of the overall meaning of the idiom, it must pass tests like semantic modification of the idiomatic meaning of its nominal part (= Ernst's internal modification), quantifier variation in the idiomatic meaning of its nominal part, and/or anaphoric references to the idiomatic meaning of its nominal part; see \citet{Nunberg:al:94}.}



\subsection{Internal modification} \label{IntMod}

In internal modification, the literal or figurative meaning of the modifier applies to the idiomatic meaning of the idiom's noun, see \REF{Reagan bandwagon}, Ernst's (8).

\ea \label{Reagan bandwagon}
In spite of its conservatism, many people were eager to \textit{jump on the} \underline{\smash{horse-drawn}} 
\underline{\smash{Reagan}} \textit{bandwagon}.
\z

\noindent If you jump on the bandwagon in the idiomatic sense, you join a growing movement (in an opportunistic way or simply for the excitement) once that movement is perceived to be successful.\footnote{Variations of this idiom are \textit{hop on the bandwagon} and \textit{climb on the bandwagon}. All of them allude to literally jumping/hopping/climbing on the wagon that used to carry (and sometimes still does) the band and the candidate during a political campaign.}
This is directly reflected in Ernst's decomposition of the idiom into two parts and his assumption that the literal and the idiomatic meaning of each part are linked:  `jump on' is linked to `join', and `bandwagon' is linked to `movement'.

\begin{sloppypar}
  In the sentence in \REF{Reagan bandwagon}, there are two modifiers
  within \textit{jump on the bandwagon}: \underline{\smash{Reagan}}
  and \underline{\smash{horse-drawn}}.\footnote{Note, however, that
    \citet{ernst81} focuses on the modifier
    \underline{\smash{horse-drawn}} only.}  Together with these
  modifiers, Ernst argues, the idiom expresses something like `join
  the old-fashioned Reagan campaign', i.e.\@
  \underline{\smash{Reagan}} and \underline{\smash{horse-drawn}}
  modify the noun \textit{bandwagon} on its idiomatic reading, not
  only syntactically but also semantically. More precisely, the
  figurative meaning of the modifier \underline{\smash{horse-drawn}}
  ($\rightsquigarrow_{\text{\itshape inf}}$ `old-fashioned' or `behind
  the times', at least in relation to \textit{bandwagon}) modifies the
  meaning of the nominal \textit{Reagan bandwagon}, in which the
  literal meaning of the modifier \underline{\smash{Reagan}}
  ($\rightarrow_{lit}$ `Reagan') modifies the idiomatic meaning of the
  noun \textit{bandwagon} ($\rightsquigarrow_{id}$ `movement').
\end{sloppypar}

To conclude, in internal modification, modifiers %do 
not only have the form and position (= morphosyntactic characteristics) of prenominal modifiers but also behave like them semantically, as they characterize the meaning of the following nominal. While the noun itself is interpreted in its idiomatic meaning, the interpretation of the modifiers can be literal (as with \underline{\smash{Reagan}}) or figurative (as with \underline{\smash{horse-drawn}}).



\subsection{External modification} \label{ExtMod}

In external modification, the literal or figurative meaning of the modifier applies to the idiomatic meaning of the idiom as a whole and functions like a domain adverb, see \REF{social bucket}, taken from \citet[51]{ernst81}.

\ea \label{social bucket}
With that dumb remark at the party last night, I really \textit{kicked the} \underline{\smash{social}} \textit{bucket}.
\z

\noindent If you kick the bucket in the idiomatic sense, you die. Nothing is said about a bucket or kicking. \mbox{In \REF{social bucket}}, we again have a modifier in the idiom: \underline{\smash{social}}. In contrast to the situation in \REF{Reagan bandwagon}, however, it is not the case that the modifier modifies the idiomatic meaning of the idiom's noun. Instead, \textit{I kicked the social bucket} means that the speaker did the ``bucket-kicking" in the social domain, i.e.\@ she did not die physiologically (if she had, she would not have been able to report that) but only socially. It is not the meaning of the idiom's noun but the meaning of the entire idiom that is modified. Truth-conditionally, the meaning of the sentence in \REF{social bucket} seems to be indistinguishable from the meaning of the sentence in \REF{socially ktb}:

\ea \label{socially ktb}
\underline{\smash{Socially}}, I really \textit{kicked the bucket} with that dumb remark at the party last night.
\z

\noindent As the modifier in external modification specifies the domain within which the meaning of the idiom applies, Ernst calls external modifiers \textit{domain delimiters}. Typical domain delimiters are adjectives belonging to professional or academic domains, like \textit{political}, \textit{economic}, \textit{musical}, etc. However, there are also non-typical domain-delimiting modifiers that can nonetheless function as domain delimiters in certain contexts, see \REF{celluloid revenge}, Ernst's (24).

\ea \label{celluloid revenge}
He denied that the Saudis, angry over [the movie] \textit{Death of a Princess}, were \textit{seeking} some \underline{\smash{celluloid}} \textit{revenge} with a movie of their own.
\z

\noindent In this example, ``\underline{\smash{celluloid}} is being used figuratively, and is more or less equivalent to the literal \underline{\smash{cinematic}}'' \citep[55]{ernst81}. From examples like these Ernst concludes that external modification is not restricted to one particular lexical class of adjectives.



\subsection{Conjunction modification} \label{ConjMod}

In conjunction modification, the last of Ernst's three types of idiom modification and our central topic in this paper, the meaning of the modifier applies to the meaning of the idiom's noun, just like in internal modification. However, unlike in internal modification, Ernst argues, the modifier does not apply to the idiomatic meaning of the noun but to its literal meaning, and this happens in an additional proposition that is independent of the proposition that expresses the meaning of the idiom. Conjunction modification is exemplified in \REF{cross-gartered leg}, Ernst's (10), taken from a review of a production of the Shakespearean play \textit{Twelfth Night}:

\ea \label{cross-gartered leg}
Malvolio deserves almost everything he gets, but ...\@ there is that little stab of shame we feel at the end for having had such fun \textit{pulling} his \underline{\smash{cross-}}\underline{\smash{gartered}} \textit{leg} for so long.
\z

\noindent If you pull someone's leg in the idiomatic sense, you playfully deceive that person. It need not, and usually does not, have anything to do with that person's leg(s). However, the insertion of the modifier \underline{\smash{cross-gartered}}, as in \REF{cross-gartered leg}, suddenly leads to an interpretation that includes the proposition that Malvolio has a cross-gartered leg, a proposition that is entirely independent of the meaning of the idiom. For reasons of clarity, let us look at a simplified version of \REF{cross-gartered leg}, namely \REF{cross-gartered leg -- simple}: 

\ea \label{cross-gartered leg -- simple}
We \textit{pulled} Malvolio's \underline{\smash{cross-gartered}} \textit{leg}.
\z

\noindent According to Ernst, this sentence expresses the conjunction of two independent propositions. Here and in the following, we will spell his analysis out in detail and use the representation format shown in \REF{analysis cross-gartered leg -- simple} to do so.\footnote{In our representations and explanations of the conjunction modification analyses, in contrast to our representations and explanations of the natural language examples, we italicize not just the words that belong to the idiom but all words, including the modifier. Moreover, and more importantly, we strike out those words that are not semantically interpreted at a particular instance (this is different from the Minimalist notation, in which strikeout usually represents the deletion of phonological material while keeping that material's meaning). It is important to note here that $s_{1}$ and $s_{2}$ are, in fact, one and the same string with different parts of that same string being semantically interpreted in $s_{1}$ and $s_{2}$. For reasons of simplicity, however, we will talk about them as if they were two different strings.\label{strings}}

\ea \label{analysis cross-gartered leg -- simple} 
Conjunction modification analysis of \REF{cross-gartered leg -- simple}: \smallskip\\
\begin{tabularx}{\linewidth}[t]{llQ}
 & 	$s_{1}$: & \textit{We pulled Malvolio's \sout{cross-gartered} leg.} \\
$\rightsquigarrow_{id}$ & $p_{1}$: & `We playfully deceived Malvolio.' \medskip\\
						& $s_{2}$: & \textit{\sout{We pulled} Malvolio's cross-gartered leg.} \\
$\rightarrow_{lit}$		& $p_{2}$: & `Malvolio has a cross-gartered leg.' \medskip\\
						& $p_{1}$ \& $p_{2}$: & `We playfully deceived Malvolio, who has a cross-gartered leg.' \\
\end{tabularx}
\z

\noindent The analysis in \REF{analysis cross-gartered leg -- simple} expresses that the proposition $p_{1}$ (`We playfully deceived Malvolio.') represents the idiomatic meaning ($\rightsquigarrow_{id}$) of the string $s_{1}$ (\textit{We pulled Malvolio's leg.}), which is the sentence in \REF{cross-gartered leg -- simple} without the modifier \textit{cross-gartered}. Without that modifier, $s_{1}$ says nothing about Malvolio's leg. The proposition $p_{2}$ (`Malvolio has a cross-gartered leg.'), in contrast, is the non-idiomatic and non-figurative (hence $\rightarrow_{lit}$) meaning of the string $s_{2}$ (\textit{Malvolio's cross-gartered leg} -- the NP-complement of the verb in \REF{cross-gartered leg -- simple}) and hence does say something about Malvolio's leg, namely that it is cross-gartered. The two independent propositions $p_{1}$ and $p_{2}$ are then conjoined into \mbox{$p_{1}$ \& $p_{2}$}: `We playfully deceived Malvolio, and Malvolio has a cross-gartered leg.' Alternatively, and expressed more naturally: `We playfully deceived Malvolio, who has a cross-gartered leg.' 

On top of cases like the one we have just dealt with, Ernst also points to cases in which $p_{2}$ is figuratively reinterpreted, see \REF{Gucci belts}, Ernst's (40).

\ea \label{Gucci belts}
With the recession, oil companies are having to \textit{tighten} their \underline{\smash{Gucci}} \textit{belts}.
\z

\noindent If you have to tighten your belt in the idiomatic sense, you have to economize. Let us once again simplify the example:

\ea \label{Gucci belts -- simple}
Oil companies have to \textit{tighten} their \underline{\smash{Gucci}} \textit{belts}.
\z

\noindent Just like ``We \textit{pulled} Malvolio's \underline{\smash{cross-gartered}} \textit{leg}.'' in \REF{cross-gartered leg -- simple}, the sentence in \REF{Gucci belts -- simple} expresses the conjunction of two propositions of which the first is idiomatic, whereas the second is non-idiomatic and independent of the first. In contrast to \REF{cross-gartered leg -- simple}, however, the second proposition expressed by \REF{Gucci belts -- simple} is the result of a figurative reinterpretation (subsumed under $\rightsquigarrow_{\text{\itshape inf}}$ in this paper):\footnote{Here and in the following, we will use the arrow $\rightsquigarrow_{\text{\itshape inf}}$ whenever a figurative reinterpretation is at play or any other kind of inference needs to be drawn from the literal meaning by taking into account the overall context and/or world knowledge. Note that in a non-figurative inference, the literal meaning that the inference is based on continues to hold, whereas in a figurative reinterpretation, it does not.}

\ea \label{analysis Gucci belts -- simple} 
Conjunction modification analysis of \REF{Gucci belts -- simple}:\smallskip\\
\begin{tabularx}{\linewidth}[t]{@{}llQ@{}}
						 & $s_{1}$: & \textit{Oil companies$_{i}$ have to tighten their$_{i}$ \sout{Gucci} belts.} \\
$\rightsquigarrow_{id}$	 & $p_{1}$: & `Oil companies have to economize.' \medskip\\
                         & $s_{2}$: & \textit{\sout{Oil companies$_{i}$ have to tighten} their$_{i}$ Gucci belts.} \\
$\rightarrow_{lit}$      & $p_{2}$: & `Oil companies have Gucci belts.' \\
$\rightsquigarrow_{\text{\itshape inf}}$ &	$p_{2'}$: &  `Oil companies are rich.' \medskip\\
                        &	$p_{1}$ \& $p_{2'}$: & `Oil companies have to economize, and they are rich.'
\end{tabularx}
\z

\noindent The proposition $p_{1}$ (`Oil companies have to economize.') is the idiomatic meaning ($\rightsquigarrow_{id}$) of the string $s_{1}$ (\textit{Oil companies$_{i}$ have to tighten their$_{i}$ belts.}), which is the sentence in \REF{Gucci belts -- simple} without the modifier \textit{Gucci}. The proposition $p_{2'}$ (`Oil companies are rich.'), in contrast, is a figurative reinterpretation of the intermediate proposition $p_{2}$ (`Oil companies have Gucci belts.'), which expresses a possessive relation between oil companies \mbox{(= the possessors)} and belts by the luxury brand Gucci \mbox{(= the possessions)}, which are symbols of great wealth. This intermediate proposition represents the non-idiomatic and non-figurative (hence $\rightarrow_{lit}$) meaning of $s_{2}$ (\textit{their$_{i}$ Gucci belts}), which is the NP-complement of the verb in \REF{Gucci belts -- simple}, in which the reference of the possessive determiner \textit{their$_{i}$} has already been resolved, so that \textit{their$_{i}$ Gucci belts} is identical in meaning to \textit{oil companies' Gucci belts}. The two independent propositions $p_{1}$ and $p_{2'}$ are then conjoined into `Oil companies have to economize, and oil companies are rich.' More naturally: `Oil companies have to economize, and they are rich.' So, neither $p_{1}$ nor $p_{2'}$ nor their conjunction says anything about belts or Gucci or Gucci belts, and there is no literal possession of such belts by oil companies.

\begin{sloppypar}
  However, whereas the meaning components of a literal or idiomatic
  meaning can simply be retrieved from the lexicon, i.e.\@ accessed
  directly, a figurative interpretation (\mbox{in \ref{analysis Gucci
      belts -- simple}:} `Oil companies are rich.') is always based
  on, and hence a reinterpretation of, a literal meaning (\mbox{in
    \ref{analysis Gucci belts -- simple}:} `Oil companies have Gucci
  belts.'). Consequently, at one point within the analysis of
  \REF{Gucci belts -- simple}, the literal meaning of the idiom's noun
  \textit{belts} and the literal meaning of the modifier
  \textit{Gucci} actually do play a role, just like the literal
  meaning of the idiom's noun \textit{leg} and the literal meaning of
  the modifier \textit{cross-gartered} do in the analysis of
  \REF{cross-gartered leg -- simple}, whose interpretation process
  does not contain any figurative steps. One of the reasons why a
  propo\-sition is reinterpreted figuratively can be that its literal
  meaning does not make much sense, which is the case \mbox{in
    \REF{analysis Gucci belts -- simple}}, as oil companies do not
  usually have belts.\footnote{However, even if we were talking about
    people instead of companies, it would not be necessary that those
    people have (literally possess) Gucci belts, and a figurative
    reinterpretation would still be possible.}
\end{sloppypar}


\section{Zooming in on conjunction modification} \label{Zoom ConjMod}

Before we turn to our corpus examples and their analysis in the spirit of \citeauthor{ernst81}'s (\citeyear{ernst81}) conjunction modification (see Section \ref{clear cases} to Section \ref{controversial cases}), let us delineate our general take on conjunction modification (see Section \ref{our take}) and present the four semantically non-decomposable idioms to be studied (see Section \ref{our four idioms}).



\subsection{Our take on conjunction modification} \label{our take}

First, we perceive conjunction modification and the modification of literal and idiomatic meanings within idioms in general to be well within the scope of a grammatical theory of idioms. Sometimes these phenomena have been denied this status, being discarded as ``word play''.\footnote{See, for instance, \citet{schenk95} or \citet{nicolas95}, who claim that any modification of idioms is either (i) external modification or (ii) statistically negligible and outside the scope of a grammatical theory of idioms, which for them are always semantically non-decomposable units.}
Even if conjunction modification were to fall within ``word play'' (however we define it), it would still involve language and thus should be analyzable.

\begin{sloppypar}
  Second, if conjunction modification, as Ernst claims, adds an
  independent prop\-osition, it should be a non-restrictive kind of
  noun modification. Restrictive modification, e.g. in the combination
  of adjective (A) and noun (N), involves intersecting the set of
  entities with the property N with the set of entities with the
  property A, or with subsective As, narrowing the set down to the set
  of entities that have both the A and the N properties
  (e.g. \textit{black elephants} have both the black property and the
  elephant property, or are a subset of \textit{elephants}) and
  therefore the A denotes a property \citep[see,
  e.g.,][]{kamppartee}. Non-restrictive modification, on the other
  hand, adds a secondary proposition that does not narrow down the
  nominal property and the role it plays in the primary proposition;
  therefore the content of the secondary proposition is often analyzed
  as being outside the main assertion of the first
  proposition %\citep[see, e.g.,][and literature cited therein]{morzyckibook, mcnally16}
  (see, e.g., \citealt{morzyckibook}; \citealt{mcnally16}; and
  literature cited therein). Propositions, in contrast to properties
  (predicates) expressed by adjectives or restrictive relative
  clauses, cannot modify an N restrictively.
\end{sloppypar}

Third, we would like to emphasize, just like Ernst does, that semantically non-decomposable idioms only allow for conjunction modification and external modification, as internal modification requires access to an idiomatic meaning of the idiom's noun, which semantically non-decomposable idioms cannot provide. Therefore, if Ernst's hypothesis is correct that modifiers in idioms are in principle three-way ambiguous, focusing on semantically non-decomposable idioms in the empirical investigation removes one level of ambiguity. In the following we therefore restrict our attention to semantically non-decomposable idioms. 


\subsection{Our four idioms} \label{our four idioms}

We chose two English and two German semantically non-decomposable idioms with the meaning `die', see \REF{ktb+btd} for the English and \REF{dLa+iGb} for the German idioms. 

\ea \label{ktb+btd}
\ea
kick the bucket
\ex
bite the dust
\z
\ex \label{dLa+iGb}
\ea
\gll	den Löffel abgeben \\
	the.\textsc{acc} spoon on.pass \\
\glt	`(lit.) pass on the spoon'
\ex 
\gll 	ins Gras beißen \\
	in.the.\textsc{acc} grass bite \\
\glt	`(lit.) bite into the grass'
\z
\z

\noindent We searched for occurrences of these four idioms in combination with modifiers that seemed likely to be of the conjunction modification kind using the corpora ENCOW16A (World Englishes) and DECOW16A (German, Austrian and Swiss German) at webcorpora.org.

In \REF{ktb+btd} and \REF{dLa+iGb}, our four idioms are paired up by language. However, there are good reasons to pair them up instead as in \REF{ktb+pots} and \REF{btd+bits}. In order to make those reasons more obvious, \REF{ktb+pots} and \REF{btd+bits} do not contain the original German idioms but their literal 
translations (as if they existed in English that way).\largerpage[-1]

\ea \label{ktb+pots}
\ea
kick the bucket
\ex
pass on the spoon
\z
\ex \label{btd+bits}
\ea
bite the dust
\ex
bite into the grass
\z
\z

\noindent Whereas buckets and spoons, just like belts, are typical personal possessions, dust and grass can be interpreted as types of ground. Personal possessions and their traits, like their brand and/or their material, invite inferences about their possessors \citep[see, e.g.\@,][]{Belk88}, while grounds and their traits, like their surface and/or what you find on it, invite pars pro toto inferences about the locations that they are a part of \citep[for a somewhat similar reasoning based on conceptual contiguity, see][92]{stathi07}. Building on this and on \citeauthor{ernst81}'s (\citeyear{ernst81}) definition of conjunction modification, see Section \ref{ConjMod}, we expected that the analyses of our corpus examples would contain a proposition including $\textbf{die}(x)$ and a proposition of the form `$x$ has a \textsc{modifier} bucket/spoon' or `the dust/grass is \textsc{modifier}'\footnote{As \citet{ernst81} expresses at the top and bottom of page 60, in (47), and in the middle of page 64, the second conjunct in conjunction modification is not limited to `$x$ has a \textsc{modifier} $y$' but can take on different forms. Given that this second proposition is anchored in the first proposition, we adjust its tense/aspect/mood accordingly.} and that it would be necessary at times to reinterpret the latter proposition figuratively, as in the analysis of the \underline{\smash{Gucci}} \textit{belts} example in \REF{analysis Gucci belts -- simple}, or to draw non-figurative inferences from it.

To make the possessive relation in our first pair of idioms explicit also in cases where there is no possessor (as there is in \ref{cross-gartered leg -- simple}) or no possessive determiner (as there is in \ref{Gucci belts -- simple}), we will also co-index the definite expressions \textit{the bucket, the spoon} with the subjects, in analogy to \REF{Gucci belts -- simple} (e.g. \textit{the$_i$ bucket}). We treat the definites in these cases as weak possessive definites \citep[in the sense of][]{poesio94, barker05}, of the sort we find in \REF{weakdef} \citep[from][]{lebruyn14}.

\ea\label{weakdef}
\ea
I hit him on the hand.
\ex	
He raised the hand.
\z
\z

\noindent Le Bruyn's analysis of the definite in these examples (at some step of the analysis) involves a relation to a \textsc{pro} that is co-indexed with an (intrinsic) possessor, as in \REF{thehand} \citep[adapted from][324]{lebruyn14}.

\ea	the \textsc{pro}$_i$ hand\quad $\stackrel{trans}{\rightsquigarrow}$\quad $\iota z (\textbf{hand}(z) \wedge \textbf{intrinsically\_belong\_to}(i)(z))$\label{thehand}
\z

\noindent In the following, when we use co-indexation on the definites in our idioms (e.g. \textit{the$_i$ bucket}), we will do this as a short-cut for an analysis of the sort in \REF{thehand}, although we are not committed to a particular account of weak (possessive) definites at this point. With these observations in mind, let us turn to our corpus examples.



\subsection{Corpus examples of conjunction modification} \label{clear cases}\largerpage[-1]

For each of our four idioms, we will now discuss a corpus example that we think fits Ernst's conjunction modification category. The first example in this line-up is about the death of Hugo Ch\'avez, the former President of Venezuela, see \REF{golden bucket}.

\ea \label{golden bucket}
Venezuela's Friend of the Working Class, Hugo Ch\'avez, \textit{kicked the} \underline{\smash{golden}} \textit{bucket} with an estimated net worth of 2 billion dollars.\footnote{\url{https://canadafreepress.com/article/a-socialism-spill-on-aisle-9} (last accessed on 5 April 2018)}
%Unique COW16 ID 985a4269b5c68d85fa3193ca7cd9b71d0818
\z

\noindent A conjunction modification analysis of this example in our representation format looks as in \REF{analysis golden bucket}.

\ea \label{analysis golden bucket} 
Conjunction modification analysis of \REF{golden bucket}:\smallskip\\
\begin{tabularx}{\linewidth}[t]{@{}llQ@{}}      
                         & 	$s_{1}$: & \textit{Hugo Ch\'avez$_{i}$ kicked the$_{i}$ \sout{golden} bucket.} \\
$\rightsquigarrow_{id}$ &	$p_{1}$: & `Hugo Ch\'avez died.' \medskip\\
  & 	$s_{2}$: & \textit{\sout{Hugo Ch\'avez$_{i}$ kicked} the$_{i}$ golden bucket.} \\
$\rightarrow_{lit}$	 &	$p_{2}$: & `Hugo Ch\'avez had a golden bucket.' \\
$\rightsquigarrow_{\text{\itshape inf}}$	&	$p_{2'}$: &  `Hugo Ch\'avez was rich.' \medskip\\
								&	$p_{1}$ \& $p_{2'}$: & `Hugo Ch\'avez died, who was rich.'
\end{tabularx}
\z

\noindent As mentioned underneath \REF{btd+bits}, the material of a personal possession like a bucket invites inferences about its possessor. And since the material gold is a well-known symbol for wealth, stating that the late Hugo Ch\'avez had a golden bucket ($p_{2}$) invites the inference that he was rich ($p_{2'}$). If you take that inference to be a figurative reinterpretation of $p_{2}$, which seems to be the most plausible variant here, then nothing is said about Hugo Ch\'avez having a golden bucket. All that you obtain in the end is that he was rich (cf.\@ the analysis of Ernst's \underline{\smash{Gucci}} \textit{belts} example in \REF{analysis Gucci belts -- simple}). In conjunction, $p_{1}$ and $p_{2'}$ then result in `Hugo Ch\'avez died, who was rich.'\footnote{An anonymous reviewer correctly observed that sentences such as \textit{Hugo Ch\'{a}vez kicked the} \underline{\smash{drunk}}/\underline{\smash{poor}}/\underline{\smash{70-year-old}} \textit{bucket} cannot (easily) express `Hugo Ch\'{a}vez died drunk\slash poor\slash at the age of 70' and wondered why this should be the case. Following the conjunction modification analysis, the answer would go as follows: Neither literal \underline{\smash{drunk}} nor literal \underline{\smash{poor}} makes any sense as a modifier of literal \textit{bucket} (a bucket can neither be drunk nor poor). This is different with literal \underline{\smash{70-year-old}}, which does make sense as a modifier of literal \textit{bucket} (a bucket can certainly be 70 years old), but maybe having a 70-year-old bucket (in contrast to having a rusty bucket, for example) is simply not graphic enough to be easily interpreted in a figurative manner.

The above does not mean, of course, that \underline{\smash{golden}} is the only possible modifier that can occur within a conjunction modification of \textit{kick the bucket}. Consider the following example: \textit{To her detractors, the ``iron lady" has finally kicked the tin bucket -- may she rust in peace.} (\url{https://dinmerican.wordpress.com/2013/04/08/53476}). Just like literal \underline{\smash{golden}}, literal \underline{\smash{tin}} does make sense as a modifier of literal \textit{bucket}, as a tin bucket is a steel bucket coated with zinc oxide, which makes the steel more rigid and rugged, and there is an obvious figurative interpretation of the Iron Lady having such a steel bucket, namely that she was tough and uncompromising, as the name \textit{Iron Lady} already indicates.}\largerpage[-2]

Our second corpus example is about the mentalist Vincent Raven, who, just like Uri Geller, claims to be able to bend spoons by sheer mental power and who almost died from a stroke that he had after falling on his head. See \REF{bent spoon} for the example and \REF{analysis bent spoon} for the analysis.

\ea \label{bent spoon}
Oder Vincent Raven aus Uri Gellers ProSieben-Sendung, der einen Unfall hatte und beinahe \textit{den} \underline{\smash{verbogenen}} \textit{Löffel abgegeben} hätte.\footnote{\url{https://carolin-neumann.de/2009/02/fuehlt-euch-bravo} (last accessed on 5 April 2018)}
%Unique COW16 ID 743d99c6f280fa0cbad074e7ee04e1572940
\glt `Or Vincent Raven from Uri Geller's show on ProSieben [German TV channel], who had an accident and almost \textit{passed on the} \underline{\smash{bent}} \textit{spoon}.'
\ex \label{analysis bent spoon} 
Conjunction modification analysis of \REF{bent spoon}:\smallskip\\
\begin{tabularx}{\linewidth}[t]{@{}llQ@{}}
 & 	$s_{1}$: &  \textit{Vincent Raven$_{i}$ almost passed on the$_{i}$ \sout{bent} spoon.} \\
$\rightsquigarrow_{id}$	&	$p_{1}$: &  `Vincent Raven almost died.' \medskip\\
& 	$s_{2}$: &  \textit{\sout{Vincent Raven$_{i}$ almost passed on} the$_{i}$ bent spoon.} \\
$\rightarrow_{lit}$ &	$p_{2}$: & `Vincent Raven has a bent spoon.' \\
$\rightsquigarrow_{\text{\itshape inf}}$	&	$p_{2'}$: & `Vincent Raven bends spoons.' \medskip\\
&	$p_{1}$ \& $p_{2'}$: & `Vincent Raven, who bends spoons, almost died.'
\end{tabularx}
\z

\noindent Just as idiomatic \textit{kick the bucket} in English, idiomatic \textit{pass on the spoon} in German means `die' ($p_{1}$). And just as \underline{\smash{golden}} in \REF{golden bucket} nonetheless applies to the literal meaning of the noun \textit{bucket}, \underline{\smash{bent}} in \REF{bent spoon} nonetheless applies to the literal meaning of the noun \textit{spoon}, and, here too, this happens in an additional proposition ($p_{2}$) that is independent of the proposition that expresses the meaning of the idiom. However, learning that someone has a bent spoon is far less telling than learning that someone has a Gucci belt or a golden bucket. In order for readers/listeners to be able to interpret this, they need some knowledge about Vincent Raven or Uri Geller's show ``The next Uri Geller'' or a telling linguistic or non-linguistic context, so that they get the inference $p_{2'}$ that Vincent Raven bends spoons. And if they take that inference to be a figurative reinterpretation of $p_{2}$, then the content of $p_{2}$ plays no role in the final interpretation of \REF{bent spoon}, so that there is no claim that Vincent Raven actually has a bent spoon.

Our third corpus example is about the three ideals of the French Revolution and the lives that were taken in the attempt to achieve these ideals, see \REF{blood-spattered dust}.

\ea \label{blood-spattered dust}
It was the great Trinity of the French Revolution, and you can still see it carved in stone over town halls and elsewhere in France: `Liberty, Equality, Fraternity'. But the greatest of these, it turns out, is `Equality'. `Liberty' soon \textit{bit the} \underline{\smash{blood-spattered}} \textit{dust} along with `Fraternity' as the drive to the unattainable goal of `Equality' took over as it was bound to do.\footnote{\url{http://thebritishresistance.co.uk/tim-haydon/1637-the-destructive-lie-of-equality} 
(could no longer be accessed on 5 April 2018)}
%Unique COW16 ID a02a2f58a35106448ce4ed3a4f8f4b33814a
\z\largerpage

\noindent For a conjunction modification analysis of this example, see \REF{analysis blood-spattered dust}. 

\ea \label{analysis blood-spattered dust} 
Conjunction modification analysis of \REF{blood-spattered dust}:\smallskip\\
\begin{tabularx}{\linewidth}[t]{@{}llQ@{}}
& 	$s_{1}$: & \textit{Liberty bit the \sout{blood-spattered} dust.} \\
$\rightsquigarrow_{id}$				&	$p_{1}$: &  `Liberty died.' \\
$\rightsquigarrow_{\text{\itshape inf}}$	&	$p_{1'}$: & `Liberty was no longer pursued.'\medskip\\
& 	$s_{2}$: & \textit{\sout{Liberty bit} the blood-spattered dust.} \\
$\rightarrow_{lit}$	&	$p_{2}$: & `The dust was blood-spattered.' \\
$\rightsquigarrow_{\text{\itshape inf}}$	&	$p_{2'}$: & `The location was blood-spattered.' \\
$\rightsquigarrow_{\text{\itshape inf}}$	&	$p_{2''}$: & `People lost their lives.'\medskip\\
&	$p_{1'}$ \& $p_{2''}$: & `Liberty was no longer pursued, and people lost their lives.'
\end{tabularx}
\z

\noindent If you state that an ideal, like liberty, bit the dust ($s_{1}$), you state that it died ($p_{1}$). Since an ideal cannot literally die, however, this is to be reinterpreted figuratively, which, in our case, results in something like: `Liberty was no longer pursued.' ($p_{1'}$). 

The inference from `The dust was blood-spattered.' ($p_{2}$) to `The location was blood-spattered.' ($p_{2'}$) is not something that Ernst assumes. However, as mentioned underneath \REF{btd+bits}, dust can be interpreted as a type of ground, whose surface and/or what you find on it (like spattered blood) invite pars pro toto inferences about the location that the ground is a part of. In an additional inferential step, we take this location to be the location of the event expressed by the idiom.\footnote{In all the examples that follow, we assume that the steps from `dust/grass' to `a location that contains the dust/grass' to `the location of the event in question' are fairly natural inferences that are drawn in discourse, and we will not specify these steps any further.\label{inferential steps}}
From `The location was blood-spattered.' ($p_{2'}$), it can then be inferred that people lost their lives ($p_{2''}$), especially in the context of the French Revolution. Combined, $p_{1'}$ and $p_{2''}$ result in `Liberty was no longer pursued, and people lost their lives.'

Our fourth example is about the 1925 peasant court in the high-lying Renchtal of the Black Forest in Germany, at which the peasant who hosted it during the last week of that year offered his guests a dish that, among others, had cost the lives of several little bunnies, see \REF{snow-covered grass} for the example and \REF{analysis snow-covered grass} for the analysis.\largerpage

\ea \label{snow-covered grass} 
Der vorbedachte Hauswirt hat für die Bedürfnisse seiner Gäste bestens gesorgt. Mehrere Häslein mussten fürs Bauerngericht \textit{ins} \underline{\smash{schneeige}} \textit{Gras beißen} und ein Schwein und Kalb das Leben lassen.\footnote{\url{http://www.museum-durbach.de/heiteres-und-geschichtliches/die-bottenauer-und-ihr-bauerngericht.html} (last accessed on 5 April 2018)}
%Unique COW16 ID bf1a284e2ef112e366e3162c694c1f6bc783
\glt `The thoughtful landlord took perfect care of his guests' needs. For the peasant court, several little bunnies had to \textit{bite into the} \underline{\smash{snow-covered}} \textit{grass}, and a pig and a calf had to give their lives as well.'
\ex \label{analysis snow-covered grass} 
Conjunction modification analysis of \REF{snow-covered grass}:\smallskip\\
\begin{tabularx}{\linewidth}[t]{@{}llQ@{}}
& 	$s_{1}$: & \textit{Several little bunnies had to bite into the \sout{snow-covered} grass.} \\
$\rightsquigarrow_{id}$	&	$p_{1}$: & `Several little bunnies had to die.' \medskip\\
& 	$s_{2}$: & \textit{\sout{Several little bunnies had to bite into} the snow-covered grass.} \\
$\rightarrow_{lit}$ & $p_{2}$: & `The grass was snow-covered.' \\
$\rightsquigarrow_{\text{\itshape inf}}$	&	$p_{2'}$: &`The location was snow-covered.' \medskip\\
&	$p_{1}$ \& $p_{2'}$: & `Several little bunnies had to die, and the location was snow-covered.' \\
\end{tabularx}
\z

\noindent Whereas in English you bite the dust, in German you bite into the grass. As a type of ground, grass, just like dust, invites pars pro toto inferences about the location that it is a part of, so that we easily get from the grass being snow-covered ($p_{2}$) to the location being snow-covered ($p_{2'}$). Apart from the two additional inferences in \REF{analysis blood-spattered dust} (from `Liberty died.' to `Liberty was no longer pursued.' and from `The location was blood-spattered.' to `People lost their lives.'), \REF{analysis snow-covered grass} and \REF{analysis blood-spattered dust} work the exact same way.

Conjunction modification is not restricted to prenominal modification, though. In example \REF{hardly visible grass}, the modifier is neither an attributive adjective nor a noun but a non-restrictive relative clause. The example is taken from Ludwig Ganghofer's 1914 novel \textit{Der Ochsenkrieg} (English title: \textit{The war of the oxen}).

\ea \label{hardly visible grass}
Und während die ausgesperrten siebenunddreißig Reiter ein zorniges Ge\-schrei erhoben, kam es innerhalb des Tores zwischen der Besatzung des Grenzwalles und den drei Abgeschnittenen zu einem Scharmützel, in dem der heilige Zeno Sieger blieb; aber zwei von seinen Soldknechten mußten \textit{ins Gras beißen}, \underline{\smash{das bei dieser mitternächtigen Finsternis kaum zu sehen war}}.\footnote{\url{http://freilesen.de/werk_Ludwig_Ganghofer,Der-Ochsenkrieg,1106,8.html}
(last accessed on 5 April 2018)}\\
%Unique COW16 ID 9ebe788a1b156f134ecf8b1a7531d7f5e2f3
\glt `And while the locked out thirty-seven horsemen clamored furiously, there was a skirmish within the gateway between the garrison of the boundary wall and the three horsemen that had been cut off, in which Saint Zeno was victorious; but two of his mercenaries had to \textit{bite into the grass}, \underline{\smash{which was hardly visible in this midnight darkness}}.'
\z

\noindent A conjunction modification analysis of this example looks as in \REF{analysis hardly visible grass}.

\ea \label{analysis hardly visible grass} 
Conjunction modification analysis of \REF{hardly visible grass}:\smallskip\\
\begin{tabularx}{\linewidth}[t]{@{}llQ@{}}
 & 	$s_{1}$: &  \textit{Two of his mercenaries had to bite into the grass, \sout{which was hardly visible in this midnight darkness}.} \\
$\rightsquigarrow_{id}$		&	$p_{1}$: & `Two of his mercenaries had to die.' \medskip\\
							& 	$s_{2}$: &  \textit{\sout{Two of his mercenaries had to bite into} the grass, which was hardly visible in this midnight darkness.} \\
$\rightarrow_{lit}$			&	$p_{2}$: & `The grass was hardly visible in this midnight darkness.' \\
$\rightsquigarrow_{\text{\itshape inf}}$	&	$p_{2'}$: & `The location was hardly visible in this midnight darkness.' \medskip\\
                            &	$p_{1}$ \& $p_{2'}$: & `Two of his mercenaries had to die, and the location was hardly visible in this midnight darkness.'\\ 
\end{tabularx}
\z

\noindent As in \REF{snow-covered grass}, \textit{ins Gras beißen} means `die' here ($p_{1}$) -- independently of any literal grass -- but still the modifier \underline{\smash{which was hardly visible in this midnight darkness}}, just like \underline{\smash{snow-covered}} in \REF{snow-covered grass}, applies to the literal meaning of the noun \textit{grass}, which happens in an additional proposition ($p_{2}$) that is independent of $p_{1}$. And as in \REF{snow-covered grass}, the modification of \textit{grass} is interpreted as a modification of the location of the dying event, just like the modification of \textit{dust} in \REF{blood-spattered dust}. The additional proposition $p_{2}$, which in this case is explicitly given by the non-restrictive relative clause (and therefore is easier to ``unpack'' than conjunction modification by an adjective or a noun, for which one always has to add a suitable relation to create a proposition), is then interpreted as `The location was hardly visible in this midnight darkness.' ($p_{2'}$). Together, $p_{1}$ and $p_{2'}$ result in: `Two of his mercenaries had to die, and the location was hardly visible in this midnight darkness.' 

In the following section, we will address three examples that are more complex cases of conjunction modification, either because they require additional background knowledge or because they go beyond a simple analysis of conjunction modification involving two propositions, since they involve a third one. After these examples, we will discuss corpus examples for which an analysis in terms of conjunction modification might not be the only option.\largerpage[2]



\subsection{Complex conjunction modification examples} \label{complex cases}

The following example, \REF{long spoons}, is taken from a review of \textit{Enigma Rosso} (English title: \textit{Red rings of fear}), a 1978 Italian-German-Spanish giallo film. In the example, the idiom \textit{den Löffel abgeben} `to pass on the spoon' is slightly altered, as it contains \textit{Löffel} `spoon' in the plural (which might reflect that more than one person died) and, more importantly for our purposes, the modifier \textit{langen}, which is an inflected form of the adjective \textit{lang} `long'.

\ea \label{long spoons}
Die Geschichte um die Umtriebe in einem Mädcheninternat, das in Teen\-agerprostitution verstrickt ist und dessen bezaubernde Zöglinge nach und nach \textit{die} \underline{\smash{langen}} \textit{Löffel abgeben}, gibt einen nett anzuschauenden Thriller ab -- leider nicht mehr.\footnote{\url{http://www.christiankessler.de/enigmarosso.html} (last accessed on 5 April 2018)}
%Unique COW16 ID b6718d495110d33fa4f829164e774c5a0707
\glt `The story of the activities at a girls' boarding school that is entangled in teenage prostitution and whose enchanting pupils, one by one, \textit{pass on the} \underline{\smash{long}} \textit{spoons}, makes for a thriller that is nice to watch -- unfortunately, that is as far as it goes.'
\ex \label{analysis long spoons -- incomplete} 
Incomplete conjunction modification analysis of \REF{long spoons}:\smallskip\\
\begin{tabularx}{\linewidth}[t]{@{}llQ@{}}
							& 	$s_{1}$: & \textit{The enchanting pupils$_{i}$ pass on the$_{i}$ \sout{long} spoons.} \\
$\rightsquigarrow_{id}$		&	$p_{1}$: & `The enchanting pupils die.'\medskip\\
							& 	$s_{2}$: & \textit{\sout{The enchanting pupils$_{i}$ pass on} the$_{i}$ long spoons.} \\
$\rightarrow_{lit}$			&	$p_{2}$: & `The enchanting pupils have long spoons.' \\
$\rightsquigarrow_{\text{\itshape inf}}$	&	$p_{2'}$: & `The enchanting pupils are ???'\medskip\\
                            &	$p_{1}$ \& $p_{2'}$: & `The enchanting pupils die, who are ???' \\
\end{tabularx}
\z

\noindent Since the proposition `The enchanting pupils have long spoons.' does not make any sense as the second conjunct of this example (not even considering the larger context of the example and/or the movie itself), that proposition must be figuratively reinterpreted. But how? One remote possibility to make sense of `The enchanting pupils have long spoons.' would be to evoke yet another idiom, \textit{jemandem die Löffel lang ziehen} `(lit.) pull someone.\textsc{dat} the spoons long', with a figurative use of \textit{spoons} for \textit{ears},\footnote{This figurative meaning of \textit{spoons} also appears in expressions like \textit{jemandem ein paar hinter die Löffel geben} `(lit.) to give someone.\textsc{dat} a few behind the spoons' (fig.\@ `to slap someone'), which might also be the idiom evoked here, and also in \textit{sich etwas hinter die Löffel schreiben} `(lit.) to write oneself.\textsc{dat} sth. behind the spoons' (fig. `to make sure to remember sth.').}
which is commonly used to refer to a teacher or a parent scolding or punishing a pupil or a child. Under this interpretation, you might infer from $p_{2}$ that the pupils have been punished before, or are being punished by being killed, as in \REF{analysis long spoons 1}.\largerpage[2]

\ea \label{analysis long spoons 1} 
First conjunction modification analysis of \REF{long spoons}:\smallskip\\
\begin{tabularx}{\linewidth}[t]{@{}llQ@{}}
                        & 	$s_{1}$: & \textit{The enchanting pupils$_{i}$ pass on the$_{i}$ \sout{long} spoons.} \\
$\rightsquigarrow_{id}$	&	$p_{1}$: & `The enchanting pupils die.' \medskip\\
                        & 	$s_{2}$: & \textit{\sout{The enchanting pupils$_{i}$ pass on} the$_{i}$ long spoons.} \\
$\rightarrow_{lit}$		&	$p_{2}$: & `The enchanting pupils have long spoons.' \\
$\rightsquigarrow_{\text{\itshape inf}}$	&	$p_{2'}$: & `The enchanting pupils are being\slash have been punished.' \medskip\\
                        &	$p_{1}$ \& $p_{2'}$: & `The enchanting pupils die, who are being\slash have been punished.'\\
\end{tabularx}
\z

\noindent The figurative interpretation of $p_{2}$ on the basis of \textit{jemandem die Löffel lang ziehen} `pull someone the spoons long', which results in $p_{2'}$ in \REF{analysis long spoons 1}, might be facilitated by the fact that in this idiom the noun \textit{Löffel} `spoon' occurs in the plural, just as in \REF{long spoons}. 

The following example, \REF{quoted long spoon}, points to a more plausible option of reinterpreting `The enchanting pupils have long spoons.' It is about Bertolt Brecht's play \textit{Mutter Courage und ihre Kinder} (English title: \textit{Mother Courage and her children}).

\ea \label{quoted long spoon}
Im Nordbayerischen Kurier schrieb Gero v.\@ Billerbeck über ``Eine Moritat gegen den Krieg'': ``Wer mit dem Teufel frühstückt, muss einen langen Löffel haben. Der Feldprediger kennt sich aus und weiß auch, dass dieser Dreißigjährige Krieg ein gottgefälliger Glaubenskrieg ist. Und weil er selbst nicht mitmischt, sondern nur davon profitiert, wie seine Wegge\-nos\-sin Anna Fierling, wird er \textit{den} \underline{\smash{zitierten langen}} \textit{Löffel} ebenso wenig \textit{abgeben} müssen [...]''\footnote{\url{http://www.luisenburg-aktuell.de/id-2009/articles/bertolt-brecht-mutter-courage-und-ihre-kinder.html} (could no longer be accessed on 5 April 2018)}
%Unique COW16 ID 04c3e4144c21f6d1edb54d88436c8f9c3978
\glt `In the N.K. [German newspaper] Gero v.\@ Billerbeck wrote about ``\mbox{A Ballad} Against the War'': ``He who sups with the devil must have a long spoon. The field preacher knows his way around and is also aware of the fact that this Thirty Years War is a God-pleasing religious war. And because he does not get involved but only benefits from it, like his companion Anna Fierling, he will not have to \textit{pass on the} \underline{\smash{quoted long}} \textit{spoon} [...]'''
\z

\noindent A conjunction modification analysis of the example in \REF{quoted long spoon} looks just like the conjunction modification analysis of the example in \REF{long spoons}, but now we can make sense of someone having a long spoon, because the beginning of the example in \REF{quoted long spoon} indicates what that is supposed to mean by making reference to the proverb \textit{He who sups with the devil must have a long spoon}. This proverb expresses a conditional (you sup with the devil $\Rightarrow$ you have a long spoon) from which we can infer by pragmatic strengthening or conditional perfection \citep[][]{geiszwicky71}, i.e.\@ by turning the conditional into a biconditional (you sup with the devil $\Leftrightarrow$ you have a long spoon), that people with a long spoon sup with the devil and hence, just like the devil himself, must be deceitful. On that account, we get the analysis in \REF{analysis quoted long spoon}.\largerpage[-2]

\ea \label{analysis quoted long spoon} 
Second conjunction modification analysis of \REF{quoted long spoon}:\smallskip\\
\begin{tabularx}{\linewidth}[t]{@{}llQ@{}}
    & 	$s_{1}$: & \textit{The field preacher$_{i}$ will not have to pass on the$_{i}$ \sout{long} spoon.} \\
$\rightsquigarrow_{id}$	&	$p_{1}$: & `The field preacher will not have to die.'\medskip\\
    & 	$s_{2}$: & \textit{\sout{The field preacher$_{i}$ will not have to pass on} the$_{i}$ long spoon.} \\
$\rightarrow_{lit}$	&	$p_{2}$: & `The field preacher has a long spoon.' \\
$\rightsquigarrow_{\text{\itshape inf}}$	&	$p_{2'}$: & `The field preacher is deceitful.' \medskip\\
                    &	$p_{1}$ \& $p_{2'}$: & `The field preacher, who is deceitful, will not have to die.' \\
\end{tabularx}
\z

\noindent Analogously, we could now infer from $p_{2}$ in \REF{analysis long spoons -- incomplete} (`The enchanting pupils have long spoons.') that the enchanting pupils are deceitful and, on the basis of that inference, complete the analysis of \REF{long spoons} as shown in \REF{analysis long spoons -- complete}.

\ea \label{analysis long spoons -- complete} 
Complete conjunction modification analysis of \REF{long spoons}:\smallskip\\
\begin{tabularx}{\linewidth}{@{}llQ@{}}
    & 	$s_{1}$: & \textit{The enchanting pupils$_{i}$ pass on the$_{i}$ \sout{long} spoons.} \\
$\rightsquigarrow_{id}$	&	$p_{1}$: & `The enchanting pupils die.'\medskip\\
            & 	$s_{2}$: & \textit{\sout{The enchanting pupils$_{i}$ pass on} the$_{i}$ long spoons.} \\
$\rightarrow_{lit}$	 &	$p_{2}$: & `The enchanting pupils have long spoons.' \\
$\rightsquigarrow_{\text{\itshape inf}}$	&	$p_{2'}$: & `The enchanting pupils are deceitful.'\medskip\\
                            &	$p_{1}$ \& $p_{2'}$: & `The enchanting pupils die, who are deceitful.' \\
\end{tabularx}
\z

\begin{sloppypar}
What these examples show is that we sometimes need considerable background knowledge (e.g.\@ of the proverb \textit{He who sups with the devil must have a long spoon.}) to make sense of the idiom-modifier combination and find an appropriate overall interpretation.
\end{sloppypar}

Our next example is complex for a different reason than the necessity of considerable background knowledge. It is complex because there is more going on than just conjunction modification. The example is from a German review of \textit{Journey to the Center of Time}, a 1967 U.S.\@ science fiction film, see \REF{silver spoon} for the example and \REF{analysis silver spoon ++} for its analysis.\largerpage[-4]

\ea \label{silver spoon}
Stanton Sr. war ein gutherziger Millionär, der viel Geld in außergewöhnliche Forschung steckte und leider kürzlich \textit{den} \underline{\smash{silbernen}} \textit{Löffel} an Stanton Jr. \textit{abgab}, welcher nix von Friede, Freude, Wissenschaft wissen, sondern Geld machen will und zwar pronto.\footnote{\url{http://www.filmflausen.de/Seiten/centeroftime.htm} (last accessed on 5 April 2018)}
%Unique COW16 ID 234ba04cd8ce44a7d65711c18c8393cb3761
\glt `Stanton Sr. was a kind-hearted millionaire who invested a lot of money in extraordinary research and, unfortunately, recently \textit{passed on the} \underline{\smash{silver}} \textit{spoon} to Stanton Jr., who does not want to know about peace, joy, science, but wants to make money, pronto.'
\ex \label{analysis silver spoon ++}
Analysis of \REF{silver spoon}:\footnote{Here, it is not just $s_{1}$ and $s_{2}$ but $s_{1}$, $s_{2}$, and $s_{3}$ that are one and the same string with different parts of that same string being semantically interpreted in $s_{1}$, $s_{2}$, and $s_{3}$ (cf.\@ footnote \ref{strings}).}\smallskip\\
\begin{tabularx}{\linewidth}[t]{@{}l@{\hspace{.5\tabcolsep}}l@{\hspace{.5\tabcolsep}}Q@{}}
& 	$s_{1}$: & \textit{Stanton Sr.$_{i}$ passed on the$_{i}$ \sout{silver} spoon \sout{to Stanton Jr}.} \\
$\rightsquigarrow_{id}$	&	$p_{1}$: & `Stanton Sr. died.'\medskip\\
                    & 	$s_{2}$: & \textit{\sout{Stanton Sr.$_{i}$ passed on} the$_{i}$ silver spoon \sout{to Stanton Jr}.}\\
$\rightarrow_{lit}$ &	$p_{2}$: & `Stanton Sr. had a silver spoon.' \\
$\rightsquigarrow_{\text{\itshape inf}}$	&	$p_{2'}$: & `Stanton Sr. was rich.'\medskip\\
                    & 	$s_{3}$: & \textit{Stanton Sr.$_{i}$ passed on the$_{i}$ silver spoon to Stanton Jr.} \\
$\rightarrow_{lit}$	&	$p_{3}$: & `Stanton Sr. passed on his silver spoon to Stanton Jr.' \\
$\rightsquigarrow_{\text{\itshape inf}}$	&	$p_{3'}$: & `Stanton Sr. passed on his wealth to Stanton Jr.'\medskip\\
 &	$p_{1}$ \& $p_{2'}$ \& $p_{3'}$: & `Stanton Sr. died, who was rich, and he passed on his wealth to Stanton Jr.'\\
\end{tabularx}
\z

\noindent Just like in the analyses of all the previous conjunction modification examples, we have one proposition that includes the idiomatic meaning of the idiom, namely that Stanton Sr.\@ died ($p_{1}$), and one proposition in which the literal meaning of the modifier is applied to the literal meaning of the idiom's noun, namely that Stanton Sr.\@ had a silver spoon ($p_{2}$), from which we infer that he was rich ($p_{2'}$),\footnote{The reinterpretation of `Stanton Sr. had a silver spoon.' as `Stanton Sr. was rich.' is additionally facilitated by the existence of the German idiom \textit{mit einem silbernen Löffel im Mund geboren sein} `to be born with a silver spoon in the mouth' (with its English equivalent \textit{to be born with a silver spoon in one's mouth}), which means that one is wealthy by birth.} 
as in the \underline{\smash{Gucci}} \textit{belts} example in \REF{Gucci belts} and the \underline{\smash{golden}} \textit{bucket} example in \REF{golden bucket}.

What sets this example apart from all the previous conjunction modification examples, however, is that its analysis does not result in the conjunction of two but three propositions. This is due to the addition of the literal goal argument \textit{to Stanton Jr.}, which, as soon as it is interpreted ($s_{3}$), enforces \textit{pass on the spoon} to be literally interpreted as well ($p_{3}$) because there is no idiom \textit{pass on the spoon to sb.} In parallel to the figurative interpretation of `having a silver spoon' ($p_{2}$) as `being rich' ($p_{2'}$), `passing on your silver spoon to sb' ($p_{3}$) is figuratively reinterpreted as `passing on your wealth to sb' ($p_{3'}$).

In the end, we %do 
not only have different interpretations of the idiom's noun \textit{spoon} but also different interpretations of the idiom's verb \textit{pass on}. Whereas $p_{1}$ includes the idiomatic meaning of \textit{pass on}, $p_{3'}$ includes its literal meaning in the sense of `hand down' or `bequeath', i.e.\@ a change of possession, and the goal phrase specifies the beneficiary of the inheritance.

In the next section, we will discuss a number of examples for which it is less clear that they involve conjunction modification. Those examples caused intense debates among the three authors of this paper, as at least one of the authors preferred to analyze them in terms of what we will call extended external modification, a broader construal of Ernst's external modification not limited to domain delimitation \citep[cf.\@][Section 4.2, in which she argues for a similar approach whilst retaining Ernst's original term]{stathi07}. In the following section, we will provide reasons why such an extended external modification analysis might be a valid alternative for the examples.



\subsection{Controversial cases} \label{controversial cases}

We have shown that our four idioms can be divided into two groups, \textit{kick the bucket} and \textit{pass on the spoon} vs.\@ \textit{bite the dust} and \textit{bite into the grass}: buckets and spoons are typical personal possessions, whose properties invite inferences about their possessors, whereas dust and grass can be interpreted as different types of ground, whose properties invite inferences about the event location. When we modify an event location, however, the event is modified as a whole, which opens up the option to analyze such a modification as a type of external modification, not in the sense of Ernst, i.e.\@ as domain delimitation, but in a more general or extended sense. There are two factors that point in this direction. 

First, as we noted, Ernst observed that external modifiers often allow an adverbial paraphrase. Given that adverbs, however, are not always domain delimiters (frame-setting sentence adverbials) but can be of various kinds, depending on where they attach and what they modify, we expect external modification in idioms not to be restricted to domain delimiters either. For example, one prominent kind is event-related modification, which, however, still relates to the idiom as a whole and could, for that reason, also be analyzed as a type of external modification.\newpage

Second, the data that Ernst uses to illustrate external modification either involve relational adjectives (e.g.\@ \textit{social} in \ref{social bucket}) or prenominal noun modifiers (of the \textit{stone lion} type). These are both types of modifiers that express an underspecified relation between modifier and modifiee \citep[see, e.g.\@,][]{mcnallyboleda04}, and a hypothesis one could pursue in future research is that this additional relation facilitates external modification.\footnote{This is not Ernst's observation, who, as we pointed out above, assumes that external modification is not restricted to a particular lexical class of adjectives.} In this section, we discuss examples that could be analyzed in terms of conjunction modification, but which also all contain relational adjectives and therefore could also be analyzed as extended external modification. While we will not offer the details of a compositional analysis of these cases -- which we have not done for any of the examples in Section \ref{clear cases} and Section \ref{complex cases}, either -- the intuitive idea should be clear.\footnote{For further discussion and a possible analysis of external modification in this broader, extended sense, see \citet{gehrkemcnally18}.}

With these considerations in mind, let us see why the following examples caused controversies among the authors of this paper. Our first example is about a South Tyrolean writer, Norbert Conrad Kaser, who apparently did not find the literature of his fellow writers very compelling, see \REF{home grass}.

\ea \label{home grass}
Erstes Aufsehen erregte der junge Kaser an einer Studientagung der Süd\-tiroler Hochschulschaft, die in Brixen von Gerhard Mumelter organisiert wurde. Hier meinte er, dass 99\% der Südtiroler Literaten am besten nie geboren wären, seinetwegen könnten sie noch heute \textit{ins} \underline{\smash{heimatliche}} \textit{Gras beißen}, um nicht weiteres Unheil anzurichten.\footnote{\url{http://www.selected4you.de/dolomiten/thema/norbert-c-kaser} (last accessed on 5 April 2018); see \citet[][91]{stathi07} for a variant of this example in which the statement of the young Kaser is reported in direct speech -- and not in indirect speech, as in \REF{home grass}.}
%Unique COW16 ID bb837f99215887316224459b19c62260648b
\glt The young Kaser caused a first stir at a South Tyrolean study conference, which was organized in Brixen by Gerhard Mumelter. There he said that it would have been better if 99\% of South Tyrolean writers had never been born and that they have his blessing to \textit{bite into the} \underline{\smash{home}} \textit{grass} by today, so as not to do any more mischief.
\z

\noindent If we take this to be conjunction modification, the analysis looks as in \REF{analysis1 home grass}.\largerpage[-2]\pagebreak

\ea \label{analysis1 home grass} 
Conjunction modification analysis of \REF{home grass}:\footnote{As \textit{heimatlich} `of one's home, native, local' (a relational adjective consisting of \textit{Heimat} `homeland' + the adjectival suffix \mbox{\textit{-lich}}) and \textit{home} are relational (any home must be the home of someone or something), the definite determiner of the verb's internal argument is co-indexed with the verb's external argument, just like in the \textit{kick the bucket} and \textit{pass on the spoon} examples.}\smallskip\\
\begin{tabularx}{\linewidth}[t]{@{}llQ@{}}
& 	$s_{1}$: & \textit{They$_{i}$ have his blessing to bite into the$_{i}$ \sout{home} grass by today.} \\
$\rightsquigarrow_{id}$				&	$p_{1}$: & `They have his blessing to die by today.'\medskip\\
& 	$s_{2}$: & \textit{\sout{They$_{i}$ have his blessing to bite into} the$_{i}$ home grass \sout{by today}.} \\
$\rightarrow_{lit}$	&	$p_{2}$: & `The grass would be their home grass.' \\
$\rightsquigarrow_{\text{\itshape inf}}$	&	$p_{2'}$: & `The location would be their homeland.' \medskip\\
                            &	$p_{1}$ \& $p_{2'}$: &`They have his blessing to die by today, and the location would be their homeland.' 
\end{tabularx}
\z

\noindent While $p_{1}$ (`They have his blessing to die by today.') is the idiomatic meaning of $s_{1}$ (\textit{They$_{i}$ have his blessing to bite into the$_{i}$ grass by today.}), $p_{2'}$ (`The location would be their homeland.') is an inference from $p_{2}$ (`The grass would be their home grass.'), which again is the non-idiomatic and non-figurative (hence $\rightarrow_{lit}$) meaning of $s_{2}$ (\textit{the$_{i}$ home grass} -- the definite NP that is (part of) the verb's internal argument in \REF{home grass}). The two independent propositions $p_{1}$ and $p_{2'}$ are then conjoined into `They have his blessing to die by today, and the location would be their homeland.' We perceive $p_{2'}$ as some kind of side information (since it is non-restrictive modification) that conveys the idea that the South Tyrolean writers would make sure to die in/on their homeland.

Given the broader understanding of external modification outlined above, \linebreak where the modifier contributes something external to the idiom (or modifies the idiom as a whole), we might also interpret  \REF{home grass} as in \REF{analysis2 home grass}:\largerpage 

\ea \label{analysis2 home grass} 
Extended external modification analysis of \REF{home grass}:\footnote{\textsc{pro} is meant as a convenient notation for indicating an implicit subject argument which plays a role in the analysis. Grammar frameworks without \textsc{pro} will usually have appropriate counterparts in their structural analyses of our examples.}\smallskip\\
\begin{tabularx}{\linewidth}[t]{@{}llQ@{}}
& 	$s_{1}$: & \textit{They$_{i}$ have his blessing to} \textsc{pro}$_{i}$ \textit{bite into the \sout{home} grass by today.} \\
$\rightsquigarrow_{id}$ &	$p_{1}$: & `They have his blessing to die by today.'\medskip\\
& 	$s_{2}$: & \textit{\sout{They$_{i}$ have his blessing to}} \textsc{pro}$_{i}$ \textit{bite into the home grass \sout{by today}.} \\
%$\rightarrow_{lit}$					&	$e_{1}$: \hspace{4pt} `their home grass' \\
\end{tabularx}\\\begin{tabularx}{\linewidth}[t]{@{}llQ@{}}
$\rightsquigarrow_{id}$	&	$p_{2}$: & `They would die in their homeland.'\medskip\\
%$\rightsquigarrow_{\text{\itshape inf}}$	&	$p_{2}$:  `The dying event would take place in their homeland.' \\
&	$p_{1}$ \& $p_{2}$: & `They have his blessing to die by today, and the dying event would take place in their homeland.' 
\end{tabularx}
\z


%VORHER(bis April 2019)
%\begin{tabular}{ll}
%$p_{1}$:				& 	They have his blessing to bite into the grass by today. \\
%					&	$\rightsquigarrow_{id}$ They have his blessing to die by today. \\
%$p_{2}$:				&	The dying event would be home. \\
%					&	$\rightsquigarrow_{\text{\itshape inf}}$ The dying event would take place in their homeland. \\
%$\sum$:				&	They have his blessing to die by today, \\ 
%					&	and the dying event would take place in their homeland.


%They have his blessing to bite into the grass by today. \\
%$\rightsquigarrow_{id+in f}$ They have his blessing to die by today, and the dying event would take place in their homeland. \textbf{***CHANGE TO TWO STEPS MAYBE***}
%\z [OLDER VERSION]

\noindent The analysis of $p_{1}$ (`They have his blessing to die by today.') is more or less the same as before: the idiomatic meaning of $s_{1}$ (\textit{They$_{i}$ have his blessing to \textsc{pro}$_i$ bite into the grass by today.}). The difference lies in $p_{2}$ (`They would die in their homeland.'), which comes about by taking the relational adjective \textit{heimatlich} `of one's home, native, local' as specifying the location for the dying event associated with the idiom as a whole and by resolving the relation of home to the subjects of this dying event (to keep things a bit more simple we did not represent this here). This looks more like an analysis in terms of external modification, just not in Ernst's more restricted sense, because the modifier is not a domain delimiter. It is still a non-restrictive kind of modification, but external modification should in principle be possible restrictively and non-restrictively. The two independent propositions $p_{1}$ and $p_{2}$ are then conjoined into `They have his blessing to die by today, and the dying event would take place in their homeland.' Again, we perceive $p_{2}$ as some kind of side information (since it is non-restrictive modification) that conveys the idea that the South Tyrolean writers might as well die in South Tyrol, where they happen to be.

The example in \REF{German grass} is similar at first sight.

\ea \label{German grass}
Auch die deutsche Geschichte mag im Gesamten alles Andere als rosig sein, doch ich lebe in diesem Staate und somit \textsc{mit} seiner Vergangenheit, seiner Gegenwart und höchstwahrscheinlich auch zukünftig, was da heissen wird, dass ich eines Tages \textit{in} \underline{\smash{deutsches}} \textit{Gras beissen} werde.\footnote{\url{http://www.chat24.de/archive/index.php?t-256.html} 
(could no longer be accessed on 5 April 2018)}
%Unique COW16 ID c9861f9c7789bcf501799940241292bc67f7
\glt German history as a whole may be anything but rosy as well, but I live in this country and thus \textsc{with} its past, its present and most likely also in the future, which will mean that one day I will \textit{bite into} \underline{\smash{German}} \textit{grass}.
\z

\noindent An analysis in terms of conjunction modification looks like in \REF{analysis1 German grass}.

\ea \label{analysis1 German grass} 
Conjunction modification analysis of \REF{German grass}:\smallskip\\
\begin{tabularx}{\linewidth}[t]{@{}llQ@{}}
 & 	$s_{1}$: & \textit{One day, I will bite into \sout{German} grass.} \\
$\rightsquigarrow_{id}$ &	$p_{1}$: & `One day, I will die.' \medskip\\
& 	$s_{2}$: & \textit{\sout{One day, I will bite into} German grass.} \\
\end{tabularx}\\\begin{tabularx}{\linewidth}[t]{@{}llQ@{}}
$\rightarrow_{lit}$	&	$p_{2}$: & `The grass will be German.' \\
$\rightsquigarrow_{\text{\itshape inf}}$	&	$p_{2'}$: & `The location will be Germany.' \medskip\\
&	$p_{1}$ \& $p_{2'}$: & `One day, I will die, and the location will be Germany.' 
\end{tabularx}
\z

\noindent Again, we infer from the second proposition (`The grass will be German.') that the location of the dying event will be Germany. However, this kind of analysis faces the problem that the modifier in this case does not seem to be adding mere side information, as non-restrictive modification would, but it rather functions as a restrictive modifier. In particular, if we left out the modifier entirely, we would lose the main information of the sentence and it would not make much sense anymore in this context (unlike in our previous example in \ref{home grass}). So, adding the modifier via conjunction modification wrongly places the meaning of the modifier in the secondary proposition rather than the primary proposition. 

Understanding the term \textit{external modification} in a broader, extended sense could be a way out of this dilemma, and we could interpret the whole sentence as one proposition, as in \REF{analysis2 German grass}.

\ea \label{analysis2 German grass} 
Extended external modification analysis of \REF{German grass}:\smallskip\\
%One day, I will bite into German grass. \\
%$\rightsquigarrow_{id+in f}$ One day, I will die (my dying will take place) in Germany.
%\z
\begin{tabularx}{\linewidth}[t]{@{}llQ@{}}
& 	$s$: & \textit{One day, I will bite into German grass.} \\
$\rightsquigarrow_{id%+in f
}$ &	$p$: & `One day, I will die (my dying will take place) in Germany.' \\
\end{tabularx}
\z

\noindent This interpretation is further facilitated by the fact that \textit{German}, like all ethnic adjectives, is a relational adjective. 

Let us now move on to controversial cases in which the referent of the literal meaning of the idiom's noun is a typical personal possession, and let us remind ourselves that personal possessions and their features can invite inferences about their possessors. The example in \REF{celestial bucket} is about Gid, a hypothetical God-like creature that is postulated and used in a proof of the existence of God in which the author talks about Gid's mortality.

\ea \label{celestial bucket}
He is presumably mortal himself; at least, being a creature of this universe, when (if) it collapses back to a mathematical point again (called the ``Big Crunch"), Gid would die then, if he hasn't already \textit{kicked the} \underline{\smash{celestial}} \textit{bucket}.\footnote{\url{http://biglizards.net/blog/archives/2011/08} (last accessed on 5 April 2018)}
%Unique COW16 ID 4e70996834c49930f35355708651fef9426d
\z

\noindent If we analyze this example in terms of conjunction modification, we get \REF{analysis1 celestial bucket}.\largerpage[-1]

\ea \label{analysis1 celestial bucket} 
Conjunction modification analysis of \REF{celestial bucket}:\smallskip\\
\begin{tabularx}{\linewidth}[t]{@{}llQ@{}}
							& 	$s_{1}$: & \textit{... if Gid$_{i}$ hasn't already kicked the$_{i}$ \sout{celestial} bucket.} \\
$\rightsquigarrow_{id}$	    &	$p_{1}$: & `... if Gid hasn't already died.' \medskip\\
							& 	$s_{2}$: & \textit{... \sout{if Gid$_{i}$ hasn't already kicked} the$_{i}$ celestial bucket.} \\
$\rightarrow_{lit}$			&	$p_{2}$: & `Gid has a celestial bucket.' \\
$\rightsquigarrow_{\text{\itshape inf}}$	&	$p_{2'}$: & `Gid is a celestial being.' \medskip\\
							&	$p_{1}$ \& $p_{2'}$: & `... if Gid, who is a celestial being, hasn't already died.' 
\end{tabularx}
\z

\noindent Under this interpretation we assume the proposition $p_{2}$ that Gid has a celestial bucket, from which we infer that Gid is a celestial being ($p_{2'}$), metonymically, like a pars pro toto (if his bucket is celestial everything else might as well be, %also he as a being
including him). However, it is also clear that this involves an additional step. The simple proposition `Gid has a celestial bucket' does not provide all of that content by itself. 

An alternative analysis of \REF{celestial bucket} in terms of external modification -- this time along the lines of Ernst's original idea that external modifiers are domain delimiters -- is shown in \REF{analysis2 celestial bucket}, where the modification is, again, interpreted restrictively so that we only get one proposition.

\ea \label{analysis2 celestial bucket} External modification analysis (in Ernst's sense) of \REF{celestial bucket}:\smallskip\\
\begin{tabularx}{\linewidth}[t]{@{}llQ@{}}
& 	$s$: & \textit{... if Gid hasn't already kicked the celestial bucket.} \\
$\rightsquigarrow_{id}$ & $p$: & `... if Gid hasn't already died in the celestial domain.' \\
$\rightsquigarrow_{\text{\itshape inf}}$ & $p'$: & `... if Gid hasn't already ceased to exist as a celestial entity.' \\
\end{tabularx}
%.. if Gid hasn't already kicked the celestial bucket.\\ 
%$\rightsquigarrow_{\text{\itshape inf}}$	... if Gid hasn't already kicked the bucket in the celestial domain. \\
%$\rightsquigarrow_{id+in f}$	... if Gid hasn't already ceased to exist as a celestial entity. 
\z

\noindent This restrictive, external interpretation of the modifier leads to a completely different understanding though: Here, we assume that Gid might first cease to exist as a celestial entity (as expressed in $p'$) to then become a terrestrial being, a mortal, and die as such when the `Big Crunch' hits (as the remaining context in \REF{celestial bucket} suggests). Under the conjunction interpretation in \REF{analysis1 celestial bucket}, on the other hand, which takes the modification to be non-restrictive, Gid dies only once and happens to be a celestial creature. The question, then, is how the text is actually supposed to be understood. 

Yet another interpretation of \REF{celestial bucket} is provided in \REF{analysis3 celestial bucket}.

\ea \label{analysis3 celestial bucket} Extended external modification analysis of \REF{celestial bucket}:\smallskip\\
\begin{tabularx}{\linewidth}[t]{@{}llQ@{}}
& 	$s$: & \textit{... if Gid hasn't already kicked the celestial bucket.} \\
$\rightsquigarrow_{id+in f}$	&	$p$:  & `... if Gid hasn't already died a celestial death (which is much more spectacular than an earthly death).'\\
\end{tabularx}
%... if Gid hasn't already kicked the celestial bucket.\\
%$\rightsquigarrow_{id+in f}$ ... if Gid hasn't already died a celestial death. (which is much more spectacular than an earthly death) 
\z

\noindent This is clearly not a conjunction modification interpretation, since we do not add a second proposition (it is again a restrictive kind of modification), but it rather feels like a manner modifier of the event (the idiom as a whole) and should then be taken as yet another instance of extended external modification. This kind of interpretation might lead to an additional inferential step (provided in brackets in $p$), and it opens up the possibility to analyze an idiom like \textit{kick the \textsc{mod} bucket} on a par with cognate object constructions of the sort \textit{die a \textsc{mod} death}, in which the modifiers in question in turn have been taken to be event modifiers \citep[see, e.g.,][]{mittwoch98, sailer10}.

Finally, example \REF{pear-shaped spoon} is about giardia, which are microscopic pear-shaped parasites that live in the intestines and cause Giardiasis, a diarrheal disease.

\ea \label{pear-shaped spoon}
Hi, die Giardien sollen doch bei 60--70°C \textit{ihren} \underline{\smash{birnenförmigen}} \textit{Löffel abgeben}. Warum muss ich dann meine Bettwäsche bei 90°C kochen?\footnote{\url{https://www.katzen-links.de/forum/darmparasiten-giardien/giardien-faq-allumfassende-infosammlung-t69985-p6.html} (last accessed on 5 April 2018)}
%Unique COW16 ID b0b5d0890328747a73b6ac793bb16ae2ff5d
\glt Hi, the giardia are supposed to \textit{pass on their} \underline{\smash{pear-shaped}} \textit{spoon} at 60--70°C. Why do I have to wash my sheets at 90°C then?
\z

\noindent An analysis of this example as conjunction modification would look like \REF{analysis1 pear-shaped spoon}.

\ea \label{analysis1 pear-shaped spoon} 
Conjunction modification analysis of \REF{pear-shaped spoon}:\smallskip\\
\begin{tabularx}{\linewidth}[t]{@{}llQ@{}}
& 	$s_{1}$: & \textit{The giardia$_{i}$ are supposed to pass on their$_{i}$ \sout{pear-shaped} spoon at 60-70$^\circ$C.} \\
$\rightsquigarrow_{id}$ & $p_{1}$: & `The giardia are supposed to die at 60-70$^\circ$C.' \medskip\\
& 	$s_{2}$: & \textit{\sout{The giardia$_{i}$ are supposed to pass on} their$_{i}$ pear-shaped spoon \sout{at 60-70$^\circ$C}.}\\
$\rightarrow_{lit}$	&	$p_{2}$: & `The giardia have a pear-shaped spoon.' \\
$\rightsquigarrow_{\text{\itshape inf}}$	&	$p_{2'}$: & `The giardia are pear-shaped.' \\
 &	$p_{1}$ \& $p_{2'}$: & `The giardia, which are pear-shaped, are supposed to die at 60-70$^\circ$C.'\\
\end{tabularx}
\z

\noindent As in the conjunction modification analyses of all the previous examples with \textit{kick the bucket} and \textit{pass on the spoon}, we here have a $p_{2}$ that includes a possession relation: `The giardia have a pear-shaped spoon.' Unlike in the previous examples, %however, 
but just like in \textit{pull sb's leg} in \REF{cross-gartered leg} and \textit{tighten one's belt} in \REF{Gucci belts}, this possessive relation is explicitly expressed by a possessive determiner. We then again infer metonymically that if the giardia have a pear-shaped spoon, they themselves are pear-shaped.

However, at this point, the question arises whether we indeed get from the giardia (literally or metaphorically) having a pear-shaped spoon to them being pear-shaped; one author of this paper does not share the intuition that a pear-shaped spoon ever plays a role in this example. In that author's opinion, the modifier seems to be attributed to the possessor right away, without the intermediate step of attaching it to `spoon', even if syntactically this is where the modifier appears. This seems to indicate that if we explicitly add a possessor via a possessive determiner inside the nominal phrase, we can combine the modifier with that possessor rather than with the noun itself, as in \REF{analysis2 pear-shaped spoon}.

\ea \label{analysis2 pear-shaped spoon} 
Possessor modification analysis of \REF{pear-shaped spoon}:\smallskip\\
\begin{tabularx}{\linewidth}{@{}llQ@{}}
& 	$s_{1}$: & \textit{The giardia$_{i}$ are supposed to pass on their$_{i}$ \sout{pear-shaped} spoon at 60-70$^\circ$C.} \\
$\rightsquigarrow_{id}$	&	$p_{1}$: &  `The giardia are supposed to die at 60-70$^\circ$C.'\medskip\\
& 	$s_{2}$: & \textit{\sout{The giardia$_{i}$ are supposed to pass on their}$_{i}$ pear-shaped \sout{spoon} \sout{at 60-70$^\circ$C}.} \\
$\rightarrow_{lit}$	&	$p_{2}$:  & `The giardia are pear-shaped.'\medskip\\
&	$p_{1}$ \& $p_{2}$: & `The giardia, which are pear-shaped, are supposed to die at 60-70$^\circ$C.' \\
\end{tabularx}
\z

\noindent However, it is far from clear how this kind of analysis, which we dubbed possessor modification, would work in terms of a general semantic composition mechanism. Yet, the meaning we get is still: `And, by the way, the giardia are pear-shaped', which is non-restrictive (as represented by the conjunction of $p_1$ and $p_2$ in \ref{analysis2 pear-shaped spoon}). 

A problem similar to the one of how to analyze the composition of \REF{pear-shaped spoon} arises with what \citet[66]{ernst81} calls `displaced epithets':

\ea
I balanced a \underline{\smash{thoughtful}} lump of sugar on the teaspoon. \\
(P.G.\ Wodehouse, cited in \citealt{Hall73})
\z

\noindent From this example, we conclude that the speaker was thoughtful, not the lump of sugar. The giardia's pear-shaped spoon could then be of this kind, and the analysis would not involve conjunction modification at all. Again we do not have a semantic composition system to describe a displacement of epithets in a way that fits cases like these but does not over-generate and predict all kinds of interpretations to be possible when they are actually not. 

On the other hand, if we analyze both examples in terms of something like conjunction modification with a possessive relation, metonymical inferences would get us from the speaker having (as part of balancing) a thoughtful lump of sugar to the speaker being thoughtful, and from the giardia having a pear-shaped spoon to the giardia being pear-shaped. The question then is whether it is a fairly obvious metonymical inference: Is it common to infer from `I have a thoughtful lump of sugar.' that `I am thoughtful.'?

In sum, what our examples in this section have shown is that it is not always straightforward to obtain an interpretation for a given modifier that is added to an idiom, and furthermore that it is not always clear which of Ernst's three categories the kind of modification belongs to. Additionally, in most cases, even in our clear cases of conjunction modification, further inferences had to be drawn. They were not only based on the second proposition alone but also had to take context and world knowledge into account. In this section, we also saw that it might be possible to extend the notion of external modification beyond its original use to cover some other types of modifiers that we encountered. The broader, extended notion of external modification lumps together various types of modification that apply to the idiom as a whole, not just to the idiom's noun. The modifiers can thus be interpreted on a par with adverbials, which also form a heterogeneous group, and we obtain an alternative to an analysis in terms of conjunction modification. External modification could be facilitated or mediated by the use of relational adjectives, though this would be a topic for future research. Finally, we discussed challenges that some of these examples entail for a precise compositional analysis, which we have to leave for future research for all our examples, though.

In the following section, we will briefly show that challenges concerning additional inferences beyond literal, figurative or idiomatic meaning and concerning the adequate formulation of semantic composition principles arise in other idiom data that do not, however, involve the kind of modification discussed so far. These data demonstrate that the observed pattern extends beyond the presence of a modifier that might (or might not) be analyzed in terms of conjunction modification.



\section{Beyond modification} \label{Beyond Mod}
In this section, we study two corpus examples of \textit{ins Gras beißen} that do not contain a modifier in the linguistic sense but still contain an adjustment of the idiom's noun \textit{Gras}. As we have seen in \REF{blood-spattered dust}, \REF{snow-covered grass}, \REF{hardly visible grass}, \REF{home grass}, and \REF{German grass}, the nouns \textit{Gras} and \textit{dust} lend themselves to a location interpretation and in the context of the idioms invite inferences about the location of the dying event.

Example \REF{cave rock} is from a review of \textit{The Descent Part 2}, a 2009 British horror film.

\ea \label{cave rock}
Erneut werden billige Schockeffekte eingesetzt [... und] wieder ist es in der Höhle meist viel zu hell, und schon wieder mutieren die überlebenden Damen zu wahren Kampfmaschinen, nur um dann doch allesamt \textit{ins Gras} \underline{\smash{respektive ins Höhlengestein}} \textit{beißen} zu müssen.\footnote{\url{http://www.kreis-archiv.de/filme/descent2.html} (last accessed on 5 April 2018)}
%Unique COW16 ID 52deee793253c929b970ae5a77166979718a
\glt `Once again, there are cheap shock effects, and once again, it is way too bright inside the cave most of the time, and again, the surviving ladies mutate into true battle machines, but in the end they still have to \textit{bite into the grass}, \underline{\smash{or rather the cave rock}}.'
\z

\noindent Even though \textit{bite into the grass}, \underline{\smash{or rather the cave rock}} does not contain a modifier and hence is not an example of idiom modification in the linguistic sense, it still contains an adjustment of the idiom's noun, and this adjustment could be analyzed by dissociating two propositions, just like in conjunction modification, see \REF{analysis cave rock}.\footnote{Alternatively, we could also assume that this adjustment happens in the same proposition (e.g. for \REF{analysis cave rock} we would get something like \textit{The ladies have to bite into the cave rock instead of the grass}). However, no matter which route is ultimately the right one, we are still facing the same kind of compositionality issues outlined here.}

\ea \label{analysis cave rock} 
Analysis of \REF{cave rock}:\smallskip\\
\begin{tabularx}{\linewidth}[t]{@{}llQ@{}}
& 	$s_{1}$: & \textit{The ladies have to bite into the grass, \sout{or rather the cave rock}.} \\
$\rightsquigarrow_{id}$	&	$p_{1}$: & `The ladies have to die.' \medskip\\
& 	$s_{2}$: & \textit{\sout{The ladies have to bite into} the grass, or rather the cave rock.} \\
$\rightarrow_{lit}$&	$p_{2}$: & `The grass is cave rock.' \\
$\rightsquigarrow_{\text{\itshape inf}}$	&	$p_{2'}$: & `The location is cave rock.' \medskip\\
&	$p_{1}$ \& $p_{2'}$: & `The ladies have to die, and the location is cave rock.' 
\end{tabularx}
\z

\noindent As in our analyses of the conjunction modification examples, $p_{1}$ is concerned with the idiom (stating that the ladies have to die), whereas $p_{2}$ is all and only about the modification of the literal meaning of the idiom's noun, which in this case only applies in the non-linguistic sense, as the added material is neither an adjective, nor a noun, nor a relative clause but the part \textit{respektive ins Höhlengestein} `or rather into the cave rock', which is combined with \textit{beißen} `bite' in a parallel fashion as is \textit{ins Gras} `into the grass'. It is not clear how this interpretation can be obtained compositionally unless we impose a semantic decomposition on the idiom that is assumed to be absent from its conventional form.

A potentially even more problematic example is given in \REF{marble floor}.

\ea \label{marble floor}
Das soll er doch gesagt haben, der gute Caesar[,] bevor er \underline{\smash{statt}} \textit{ins Gras} \underline{\smash{in den Marmorboden vom Senat}} \textit{gebissen} hat.\footnote{\url{http://www.rom-fanclub.de/Episode-1-Folgen-1-12/3719-ReEP01-/-F12-Die-Kalenden-des-Februar/Page-7.html} (last accessed on 5 April 2018)}
%Unique COW16 ID f80b8bab07c2f87e1ba51c3d796da99a4ef6
\glt `He is supposed to have said that, our good old Caesar, before he \textit{bit into} \underline{\smash{the marble floor of the Senate instead of}} \textit{the grass}.' 
\z

\noindent In a parallel fashion to the previous example we might analyze this one along the lines of \REF{analysis marble floor}.

\ea \label{analysis marble floor} 
Analysis of \REF{marble floor}:\smallskip\\
\begin{tabularx}{\linewidth}[t]{@{}llQ@{}}
& 	$s_{1}$: & \textit{Caesar bit into \sout{the marble floor of the Senate instead of} the grass.} \\
$\rightsquigarrow_{id}$				&	$p_{1}$: & `Caesar died.' \medskip\\
& 	$s_{2}$: & \textit{\sout{Caesar bit into} the marble floor of the Senate instead of the grass.} \\
$\rightarrow_{lit}$	&	$p_{2}$: & `The grass was the marble floor of the Senate.' \\
$\rightsquigarrow_{\text{\itshape inf}}$	&	$p_{2'}$: & `The location was the marble floor of the Senate.' \medskip\\
&	$p_{1}$ \& $p_{2'}$: & `Caesar died, and the location was the marble floor of the Senate.' \\
\end{tabularx}
\z

\noindent This leads to the construction of the proposition $p_{2}$ above, and the following inference to the effect that Caesar died on the marble floor of the Senate. Again, we do not know how to get there via standard semantic composition principles. What is even worse %, however,
is that due to the negation that is part of the semantics of \textit{statt} `instead of', it is literally stated that Caesar did not bite into the grass. Therefore, our $p_{1}$ is not quite right; it should contain a negation. Nevertheless, we still get the interpretation that he died, only not on grass but on the marble floor of the Senate. So since the entire idiom is present, somehow its meaning is present as well. And substituting the literal \textit{marble floor of the Senate} for the idiomatic \textit{grass} has the effect that \textit{grass} is understood literally as well.



\section{Conclusion} \label{Conclusion} 
In this paper, we reviewed \citeauthor{ernst81}'s (\citeyear{ernst81}) classical three types of idiom modification (internal, external, and conjunction modification), followed by a close investigaton of conjunction modification in semantically non-decomposable idioms as a particularly challenging phenomenon for semantic theorizing. In order to get a deeper understanding of the scope of naturally occurring meaning effects in conjunction modification, we studied corpus data of two English and two German semantically non-decomposable idioms with the same idiomatic meaning but different formal structure. Some of our findings of the effects of idiom modification followed the general pattern of Ernst's observations, while others pointed to a possible relationship with external modification. Patterns of unexpected but apparently systematic inferences and contextual adjustments outside the core cases led us to investigate data beyond modification which demonstrated the need for assuming additional inferential mechanisms and pointed to effects that are clearly outside the range of regular semantic composition.

Many of the corpus examples with our two English and two German ``dying idioms'' which were originally collected as candidates for conjunction modification were accepted as such by all authors of the present study. In those cases there was agreement that their analysis comprises a main proposition $p_{1}$ including the predicate $\textbf{die}(x)$ and a secondary proposition $p_{2}$ of the form `$x$ has a \textsc{modifier} bucket/spoon' or `the dust/grass is \textsc{modifier}'. Often it was also necessary to interpret these forms figuratively or to draw additional inferences from their literal meaning in order to obtain a coherent interpretation in context. Some examples, however, turned out to be controversial, and the available analytical tools did not provide an easy resolution for conflicting intuitions: Whereas some authors analyzed them as conjunction modification in combination with additional inferences, the other(s) preferred (a version of) external modification, where the notion of external modification had to be broadened compared to Ernst's original proposal.

We think that our data show that the distinction between semantically decomposable and semantically non-decomposable idioms might not be as categorical as \citet{Nunberg:al:94} thought (see also \citealt{bargmannsailerta}). These idioms are certainly not a semantically monolithic lexical unit with complex syntactic structure. Not only are speakers aware of their internal structure, they also seem to be ready to fall back on alternative, literal meanings of smaller syntactic units, such as of the nominal head in a noun phrase complement, any time a consistent interpretation in context of all lexical material in a given structure requires their retrieval. The meaning of these smaller units, otherwise unavailable in the idiomatic reading of the complete idiomatic expression, even serves as a basis for further interpretive processes, which can and must be considered in parallel to the idiomatic reading of the idiom as a whole -- minus material whose interpretation it cannot integrate. To us it seems that this is a much more complex situation, and truly one-to-many, than most current semantic theories are ready to entertain. At the same time, corpus evidence suggests that the processes involved are far from unsystematic, and should definitely not be discarded into the realm of linguistically inexplicable creative word play.

Whichever way the open issues will ultimately be resolved, we have seen ample evidence that idioms are excellent instances of one-to-many relations between form and meaning, and that this becomes especially obvious in conjunction modification, where the idiomatic and the literal meaning of the idiom need to be present simultaneously.

\section*{Abbreviations}
\begin{tabularx}{\textwidth}{@{}lQ@{}}
$s_{1}$	& string including the idiom and everything else but not the modifier \\
$s_{2}$ & string consisting of nothing but the NP within the idiom's verb's complement, which includes the modifier \\
$p_{1}$	& main proposition \\
$p_{2}$	& secondary proposition \\
$\rightarrow_{lit}$	& literal meaning \\
$\rightsquigarrow_{id}$ & idiomatic meaning \\
$\rightsquigarrow_{\text{\itshape inf}}$ & figurative interpretation or additional inference within the context
\end{tabularx}



\section*{Acknowledgements}
\begin{sloppypar}
This paper profited from feedback at two workshops: the 19th Szklarska Poreba workshop (February 2018) and the DGfS workshop  (\textit{Arbeitsgruppe} 4) \textit{One-to-many relations in morphology, syntax, and semantics} (Stuttgart, March 2018). Special thanks go to Christopher Götze, Louise McNally, Manfred Sailer, and two anonymous reviewers for their valuable questions, comments, and suggestions. We are also very grateful for the work of Language Science Press community proofreaders. All remaining errors are, of course, ours.
\end{sloppypar}

{\sloppy\printbibliography[heading=subbibliography,notkeyword=this]}
\end{document}
