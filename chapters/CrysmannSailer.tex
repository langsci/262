\documentclass[output=paper]{langscibook}
\ChapterDOI{10.5281/zenodo.4729787}

\title{Introduction}

\author{Berthold Crysmann\affiliation{Université de Paris, Laboratoire de linguistique formelle, CNRS}  and Manfred Sailer\affiliation{Goethe-Universität Frankfurt a.M.}}

\abstract{}

\begin{document}
\maketitle


\noindent The standard view of the form-meaning interfaces, as embraced by the
great majority of contemporary grammatical frameworks, consists in the
assumption that meaning can be associated with grammatical form in a
one-to-one correspondence. Under this view, composition is quite
straightforward, involving concatenation of form, paired with
functional application in meaning. In this book, we shall discuss
linguistic phenomena across several grammatical sub-modules
(morphology, syntax, semantics) that apparently pose a problem to the
standard view, mapping out the potential for deviation from the ideal
of one-to-one correspondences, and develop formal accounts of the
range of phenomena. We shall argue that a constraint-based perspective
is particularly apt to accommodate deviations from one-to-many
correspondences, as it allows us to impose constraints on full
structures (such as a complete word or the interpretation of a full
sentence) instead of always deriving such structures step by step.


The book consists of a general introduction and seven topical contributions, ranging from
morphology to syntax and semantics. In the introductory chapter, we
shall give a general overview and typology of one-to-many
correspondences. A number of papers in this volume are formulated in a
particular constraint-based grammar framework, Head-driven Phrase
Structure Grammar \citep{Pollard94}. These contributions investigate
how the lexical and constructional aspects of this specific theory can
be combined to provide an answer to the issue of one-to-many relations
across different linguistic sub-theories.


\section{One-to-many relations across modules}

\subsection{Many-to-many nature of morphology}

Possibly the first module of grammar where the ideal of one-to-one
correspondence has been challenged is \textit{morphology}: classical
challenges \citep{Matthews72} include (i) cumulation, where several
morphosyntactic properties are jointly expressed by a single exponent,
(ii) extended (or multiple) exponence, where a morphosyntactic
property is jointly expressed by several exponents, and (iii)
overlapping exponence, i.e. the combination of cumulation and extended
exponence. These deviations from the canon of a one-to-one
correspondence pertain to the relation between form and function.

\emph{Cumulation}, or fusion, is indeed a highly common property of
inflectional systems, where one form $n=1$ corresponds to $m>1$
functions. In fact, fusion is considered as the property that
distinguishes the broad typological class of inflectional languages
from the agglutinative type. However, cumulation can even be attested
in agglutinative languages, such as Swahili \citep{Stump93} or Finnish
\citep{Spencer:03:Morphology}.  Taking German nominal inflection as an
example, marking of number and case is often fused, illustrated by the
paradigm of \textit{Rechner} in Table~\ref{tab:ParaSplit}.

\begin{table}  
  \subfigure[\textit{Rechner} `computer']{%
    \centering
    \begin{tabular}{rll}
      \lsptoprule
       & \textsc{singular} & \textsc{plural}\\
      \midrule
      \textsc{nom} & Rechner & Rechner\\
      \textsc{gen} & Rechner-s & Rechner\\
      \textsc{dat} & Rechner & Rechner-n\\
      \textsc{acc} & Rechner & Rechner\\
      \lspbottomrule
    \end{tabular}
  }
  \subfigure[\textit{Mensch} `human']{%
    \centering
    \begin{tabular}{rll}
      \lsptoprule
       & \textsc{singular} & \textsc{plural}\\
      \midrule
      \textsc{nom} & Mensch & Mensch-en\\
      \textsc{gen} & Mensch-en & Mensch-en\\
      \textsc{dat} & Mensch-en & Mensch-en\\
      \textsc{acc} & Mensch-en & Mensch-en\\
      \lspbottomrule
    \end{tabular}
  }
  
  \subfigure[\textit{Hals} `neck']{%
    \centering
    \begin{tabular}{rll}
      \lsptoprule
      & \textsc{singular} & \textsc{plural}\\
      \midrule
      \textsc{nom} & Hals & Häls-e\\
      \textsc{gen} & Hals-es & Häls-e\\
      \textsc{dat} & Hals & Häls-en\\
      \textsc{acc} & Hals & Häls-e\\
      \lspbottomrule
    \end{tabular}
  } 
  \subfigure[\textit{Arm} `arm']{%
    \centering
    \begin{tabular}{rll}
      \lsptoprule
       & \textsc{singular} & \textsc{plural}\\
      \midrule
        \textsc{nom} & Arm & Arm-e\\
        \textsc{gen} & Arm-s & Arm-e\\
        \textsc{dat} & Arm & Arm-en\\
        \textsc{acc} & Arm & Arm-e\\
      \lspbottomrule
    \end{tabular}
  }
  \caption{German nominal paradigms\label{tab:ParaSplit}}
\end{table}


The mirror image of cumulation is \emph{extended} or \emph{multiple exponence},
where a single function $m=1$ is expressed multiple times by $n>1$
exponents (see \citealp{caballero_g-harris_a12,Harris17} for a typological
overview). In German nominal plurals,
this is attested e.g. by the combination of affixation and
\textit{umlaut}, an instance of morphologically conditioned vowel
fronting. % Hals/Hälse vs. Arm/Arme
In this volume, the chapter by \textbf{Crysmann} explores a particularly
compelling case of extended exponence in Batsbi \citep{Harris09},
where identical class agreement markers may show up multiple times
within a verb. 



What is probably even more common than pure extended exponence is
\emph{overlapping exponence}, which can be pictured as a combination
of extended exponence and cumulation: e.g.\ in the dative plural
\textit{Arm-e-n}, plural is jointly expressed by the suffixes
\textit{-e} and \textit{-n} ($1:n>1$), while at the same time
\textit{-n} cumulates plural and dative marking ($m>1:1$).   


Perhaps the most common deviation from one-to-one correspondence is
\emph{zero exponence}, with $m>0$ functions being expressed by $n=0$ forms:
e.g.\ in the paradigm of German \textit{Rechner}, a substantial number
of case and number combinations are expressed by the absence of any
inflectional marker. What is peculiar about the zero-marked forms is
that they do not form any natural class here, neither in terms of
case, nor in terms of number, nor any combination of these two
dimensions. Rather, they are interpreted in terms of paradigmatic
opposition to overtly marked cells. A common way to capture this is in
terms of Pāṇini's principle or the elsewhere condition
\citep{kiparsky_p85}, a notion embraced by almost every theory of
inflection
\citep[cf.][]{Halle93,Prince93,Anderson92,Stump01,Crysmann:Bonami:2016}.
While zero exponence represents the default more often than not, zero
exponence may sometimes exceptionally constitute an override in an
otherwise overtly marked paradigm.  Consider the German paradigm of
\textit{Mensch} `human': here the only way to give a uniform
interpretation for the overt marker \textit{-en} is in terms of
opposition to the zero-marked nominative singular cell. Thus, within
this inflectional class, zero exponence constitutes the special case,
contrasting with non-zero default marking (\textit{-en}).

While inflectional morphology also witnesses one-to-one
correspondences between form and function, almost all possible
deviations are well attested: one-to-many (cumulation), many-to-one
(extended exponence), many-to-many (overlapping exponence), and
zero-to-one. The fact that these deviations from a one-to-one ideal can
be found in practically every inflectional system makes them qualify
as an indispensable property of this grammatical module.  

 One-to-many relations are not only pervasive in the
correspondence between morphosyntactic properties and the exponents
that express them, but they are also characteristic of paradigm
structure: a frequently attested phenomenon is \emph{syncretism}, the
systematic identity of forms in different cells of the paradigm. In a
sense, syncretism constitutes an instance of (local) ambiguity. The
nominal paradigms of German we cited above provide different patterns
of syncretism, illustrating identity of forms for different
cells in the paradigm of a single word, as well as identity of
patterns of exponence across different inflectional paradigms
(cf. e.g. the singular of \textit{Rechner} and \textit{Arm} in
Table~\ref{tab:ParaSplit}).

\begin{sloppypar}
  \emph{Heteroclisis} constitutes a particular case of cross-paradigm
  syncretism, where different parts of a lexeme's paradigm adhere to
  different inflection classes
  \citep{Stump2006}. Table~\ref{tab:cz:ov} illustrates the phenomenon
  with data from Czech: in the neuter, we find two basic declension
  classes (hard and soft), where corresponding cells are marked with
  different exponents. Mixed declension neuter nouns like
  \textit{kuře} `chicken', on the other hand, inflect like soft
  declension nouns in the singular, but, in the plural, the
  case/number exponents are identical to those found in the hard
  declension.
\end{sloppypar}\largerpage[-1]

\begin{table}
\fittable{\begin{tabular}{lll@{\hspace{.75\tabcolsep}}lllll}
\lsptoprule
& \multicolumn{4}{c}{\scshape masculine} & \multicolumn{3}{c}{\scshape neuter}\\\cmidrule(lr){2-5}\cmidrule(lr){6-8}
& hard & \multicolumn{2}{c}{mixed} & soft & hard & mixed & soft\\
\midrule
\multicolumn{6}{l}{\scshape sg}\\
\scshape nom & most & \multicolumn{2}{c}{pramen} & pokoj & měst-o & kuř-e & moř-e\\
\scshape gen & most-u & pramen-u&pramen-e & pokoj-e & měst-a & kuř-et-e & moř-e\\
\scshape dat & most-u & pramen-u&pramen-i & pokoj-i & měst-u & kuř-et-i & moř-i\\
\scshape acc & most & \multicolumn{2}{c}{pramen} & pokoj & měst-o & kuř-e & moř-e\\
\scshape voc & most-e & pramen-e & pramen-i & pokoj-i & měst-o & kuř-e & moř-e\\
\scshape loc & most-ě & pramen-u & pramen-i & pokoj-i & měst-ě & kuř-et-i & moř-i\\
\scshape ins & most-em & \multicolumn{2}{c}{pramen-em} & pokoj-em & měst-em & kuř-et-em & moř-em\\
\midrule
\multicolumn{6}{l}{\scshape pl}\\
\scshape nom & most-y & \multicolumn{2}{c}{pramen-y} & pokoj-e & měst-a & kuř-at-a & moř-e\\
\scshape gen & most-ů & \multicolumn{2}{c}{pramen-ů} & pokoj-ů & měst & kuř-at & moř-í\\
\scshape dat & most-ům & \multicolumn{2}{c}{pramen-ům} & pokoj-ům & měst-ům & kuř-at-ům & moř-ím\\
\scshape acc & most-y & \multicolumn{2}{c}{pramen-y} & pokoj-e & měst-a & kuř-at-a & moř-e\\
\scshape voc & most-y & \multicolumn{2}{c}{pramen-y} & pokoj-e & měst-a & kuř-at-a & moř-e\\
\scshape loc & most-ech & \multicolumn{2}{c}{pramen-ech} & pokoj-ích & měst-ech & kuř-at-ech & moř-ích\\
\scshape ins & most-y & \multicolumn{2}{c}{pramen-y} & pokoji & měst-y & kuř-at-y & moř-i\\
\midrule
& ‘bridge’ & \multicolumn{2}{c}{‘spring’} & ‘room’ & ‘town’ & ‘chicken’ & ‘sea’\\
\lspbottomrule
\end{tabular}}
\caption{Overabundance and heteroclisis in Czech declension \citep{Bonami17b}\label{tab:cz:ov}}
\end{table}

Syncretism, however, differs from most other cases of lexical
ambiguity in being systematic, rather than accidental. While
systematic attachment ambiguities in syntax are rooted in the
geometrical properties of tree structure (Catalan numbers), the
systematicity of syncretism patterns is of a different nature,
combining underspecification in the case of natural splits with a
specific type of default logic, in the case of Pāṇinian splits. By
studying patterns of syncretism, morphologists try to understand inter
alia how a small number of exponents are deployed to distinguish a
much greater number of cells.\largerpage

\begin{sloppypar}
  The opposite of syncretism is \emph{overabundance}
  \citep{Thornton2011,Thornton12,Thornton19}, which has been accepted
  only fairly recently in morphology. Overabundance is the
  inflectional equivalent of paraphrase, so its very existence should
  not come as too much of a surprise. However, with Pāṇinian
  competition as an organising principle of lexical and morphological
  knowledge, we should expect overabundance to be the exception rather
  than the rule in inflectional systems.
\end{sloppypar}

While heteroclisis, i.e. multiple inflection class membership can just
give rise to mixed paradigms, where one set of cells adheres to one
class and another set to a different class, multiple membership may
even give rise to overabundance \citep{Thornton2011}, as witnessed
e.g. by English \textit{dreamed/dreamt} where a function has more than
one possible realisation. 

The way in which heteroclisis and overabundance can interact is
illustrated by the Czech masculine mixed declension given in
Table~\ref{tab:cz:ov}: in the plural, \textit{pramen} `spring' uses
the case/number exponents of the hard declension, entirely parallel to
what we saw in the neuter mixed declension, whereas in the singular,
we find the exponents of both hard and soft declensions. In essence,
heteroclisis appears to be one of the contributing factors to
overabundance.

Syncretism and overabundance can be thought of as the inflectional
manifestations of two very general properties of language, namely
ambiguity and paraphrases. However, within morphological theory, the
situation where one form is identical across different functions is
recognised to the extent that formal theories are optimised to
describe syncretic patterns with minimal description length, typically
using preemptive devices such as extrinsic rule ordering
\citep{Anderson92} or Pāṇinian competition
\citep{Kiparsky05,Stump01,Prince93,Embick07}. The resulting functional, as
opposed to relational, perspective on the correspondence between
inflectional meaning and form poses some challenge towards the
integration of overabundance.

In his contribution to this volume, \textbf{Beniamine} presents an
approach to computational induction of inflection classes and suggests
that heteroclisis and overabundance are actually far more wide-spread
than commonly assumed and that monotonic inheritance hierarchies, as
used in HPSG lend themselves naturally towards modelling inflectional
macro- and microclasses.

\subsection{One-to-many phenomena beyond morphology}

As shown in the previous section, one-to-many relations are well
established in morphology.  In this section, we list some example
cases to which the morphological terminology can be applied, at least
on a pretheoretical, descriptive level.

% * Compositionality in syntax and semantics (MS)
%   "The success story in linguistics" 
%One of the success stories in linguistics is Richard Montague's implementation of the 
A key insight at the basis of modern formal semantics is the
\emph{principle of compositionality}, which we show in one of its
standard versions in (\ref{PrincOfCompo}).

\ea Principle of compositionality:\\
The meaning of a complex expression is a function of the meaning of
its component parts and the way in which they are
combined.\label{PrincOfCompo}  \z

This principle captures the insight that speakers of a language can
understand utterances they have never heard if they understand the
words and the structure of these utterances.  Typical formulations
of the principle of compositionality such as (\ref{PrincOfCompo})
make a number of implicit assumptions that point towards a one-to-one
relation between form and meaning.  We shall review two aspects and
some problems with them: First, a function has a unique value for a
given input, second, there is a single relevant level of ``meaning'',
or what \citet{Bach:99} calls the dictum of \emph{one sentence, one
  proposition}.

Turning to the first aspect, the very notion of a \emph{function}
suggests that there is a \emph{unique} interpretation for any given
word-structure combination.  This is not immediately obvious once we
look at ambiguities others than lexical and structural ambiguities.
For example scope ambiguity, see (\ref{ex-scopeamb}), or
collective-distributive ambiguity, see (\ref{ex-coldist}), are not
straightforwardly related to different lexical items or syntactic
structures.

\ea
\begin{xlist}
\ex Most linguists speak at least two languages. \hfill (scope ambiguity)\label{ex-scopeamb}\\
Reading 1: For most linguists, there are at least two languages that they speak. %(Most linguists is bilingual)
\\
Reading 2: There are at least two languages such that most linguistics speak them.
\ex Two students lifted the box. \hfill (collective-distributive ambiguity)\label{ex-coldist}\\
Reading 1: Two students jointly lifted the box.\\
Reading 2: Two students lifted the box separately.
\end{xlist}
\z 

\begin{sloppypar}
  There have been numerous attempts to make the analysis of such data
  compatible with the principle of compositionality. There are three
  standard solution strategies.  First, more syntactic structure can
  be postulated to subsume these cases under structural ambiguity, as
  done in \emph{Montague Grammar} \citep{Montague:PTQ}, or through
  \emph{quantifier raising}, starting from \citet{May:77}.  Second,
  semantic shifting operations can be introduced in order to treat the
  problem as a (systematic) lexical ambiguity.  Prominent examples of
  this include \citet{Link:83}, \citet{Partee:Rooth:83} and
  \emph{Flexible Montague Grammar} \citep{Hendriks:93}.  Third,
  attempts could be made to argue that there is no real ambiguity but
  rather a vagueness, i.e., that the apparent readings are just
  different scenarios that are compatible with the one, very general,
  interpretation of the clauses. This could be done in
  \emph{underspecified semantics}, see \citet{Pinkal:99} and
  \citet{Egg:11} for an overview.
\end{sloppypar}


Let us turn to the second implicit one-to-one aspect of the principle
of compositionality.  It is usually interpreted as expressing the idea
of \emph{one sentence, one proposition}.  \citet{Bach:99} is widely
quoted as explicitly challenging this assumption, in that
whatever is ``said'' should be considered the relevant meaning in the
sense of the principle of compositionality -- in contrast to what is
being communicated implicitly by a conversational implicature.  The
prime examples of sentences with more than one proposition involve
\emph{conventional implicatures} as in the classical example from
\citet{Grice:75} in \REF{ex-grice}.%
\footnote{Grice's example in \REF{ex-grice} violates many of the LSA
  guidelines of linguistic examples, see
  \url{https://www.linguisticsociety.org/resource/lsa-guidelines-nonsexist-usage},
  accessed 2020-03-04.}$^,$%
\footnote{\citet{Bach:99} questions the notion of conventional
  implicature and rather intends to replace it by allowing more than one
  proposition.  }
 
\ea
 He is an Englishman; he is, therefore, brave. 
 \citep[44]{Grice:75}\label{ex-grice}
 \begin{xlist}
 \ex Proposition 1: \mytrans{He is brave.}\label{ex-grice-ai}
 \ex Proposition 2: \mytrans{His being brave is a consequence of his being an Englishman.}\label{ex-grice-ci}
 \end{xlist}
\z 

We indicate the two propositions expressed in \REF{ex-grice} below
the example.  Often, only the proposition in \REF{ex-grice-ai} is
considered what is being ``said'', or asserted.  The proposition in
\REF{ex-grice-ci} is considered non-asserted.  Under the heading of
\emph{projective meaning}, it has been argued that the difference
between asserted content, presupposition, conventional implicature,
and, possibly other types, is not categorical
\citep{Tonhauser:al:13,AnderBois:al:15}.

Formal approaches such as \citet{Potts:05} and \citet{Liu:12} show that
the non-asserted meaning can be computed in parallel to and with the
same techniques as the asserted content.  \citet{Gutzmann2013} provides
examples of lexical items and constructions that contribute to the
non-asserted content only (such as attributive \bsp{damn}) and to both
asserted and non-asserted content -- such as slurs like \bsp{kraut}
with the asserted meaning \mytrans{German} and the non-asserted
meaning of a speaker's negative attitude towards Germans.  This shows
that meaning computation itself is a one-to-many challenge, i.e., that
not only a single, asserted, content needs to be computed, but
 potentially several, projective meaning contributions need to be
computed in parallel.

\begin{sloppypar}
  There are, however, other constellations that are problematic for
  the one-to-one aspects of the principle of compositionality, some of
  which are also addressed in the contributions of \textbf{Sailer \&
    Richter} and \textbf{Bargmann, Gehrke \& Richter} of this volume.
\end{sloppypar}


When we reconsider the list of one-to-many phenomena in morphology, it
is easy to find analogous cases for each of them at the
morphology-syntax interface, in syntax, and at the syntax-semantics
interface.  

%periphrasis
One obvious case is \emph{periphrasis}, i.e., the marking of a
morphosyntactic category (such as tense, number, or case) by means of
several words.  A simple example of this is past tense marking in
Afrikaans: while a few verbs have a past tense form -- such as
\bspT{kan}{can} with the form \bspT{kon}{could} -- most verbs form
their past tense with the auxiliary \bspT{het}{have} and a past
participle, as in \bspT{ge-werk het}{worked have}. Neither the verb
\bsp{het} nor the past participle \bsp{ge-werk} express past tense
when used on their own.

% complex predicates, idioms
We find similar periphrastic behaviour at the syntax-semantics
interface.  Light verb constructions, complex predicates, particle
verbs, or idiomatic expressions are all cases in which a single
meaning is expressed through the use of more than one word, where
none of the words may carry this meaning outside the
combination.  While there is a continuum of transparency in these
cases, we find extreme examples such as the German particle verb
\bspTL{an-geben}{brag}{on-give} or the English idiomatic expression
\bspT{kick the bucket}{die}.

% * Redundancy: e.g. agreement -> NC (MS)
There are many cases of \emph{redundancy}, i.e., the same
morphosyntactic or semantic property is marked on more than one
word. This can be understood as the syntacto-semantic equivalent of
extended exponence. A common pattern is to find the same category
being marked on a substantive word and also by some function word.  In
some varieties of English, for example, we find both a morphological
and a periphrastic marking of the comparative, as in
\REF{more-comp-sa}.

\ea
But I found that in all area of my life where I live the most according my own rules, I feel more stronger. (GloWbE, South Africa)\label{more-comp-sa}
\z 

This constellation also occurs in the second stage in the
\emph{Jespersen cycle} \citep{Jespersen:17}, illustrated with a Frecnh
example in (\ref{ex:Jesp}).  There, an original
negation marker (\textit{ne}) is strengthened through the occurrence of a further
negative item (\textit{pas}).

\ea \label{ex:Jesp}
\gll Je ne dis pas.\\
I NE say not\\
\glt `I don't say' \citep[7]{Jespersen:17}
\z 

The Jespersen cycle has been applied to a number of grammaticalisation
processes, see \citet{Gelderen:11,Gelderen:13} for an overview.  Since
the redundant step belongs to many of the grammaticalisation cycles,
this particular one-to-many stage constitutes a standard case in the
syntactic marking of grammatical categories.

\kommentar{ This pattern is also found in what \citet{denbesten:86}
  calls \emph{negative doubling}, where a designated negative marker
  co-occurs with a negative expression in a clause.  This is
  illustrated with a West Flemish example from
  \citet{haegeman&zanuttini:91} in \REF{neg-doub}.  The negative
  indefinite \bspT{niemand}{nobody} can be considered a negative
  expression.  In the example, there is an additional preverbal marker
  \bsp{en-}, which is restricted to negated clauses and, thus,
  redundantly marks the highest verb in the clause as negated.

\ea \label{neg-doub}
\gll da Valère niemand (en-)kent\\
that Valère nobody EN-knows\\
\glt \mytrans{that Valère does not know anybody.}
\z 

}

Redundant marking outside morphology is also found  in so-called
\emph{concord phenomena}.  The most widely studied is negative
concord, where more than one negative indefinite is used in a clause
without expressing more than one negation
\citep{Jespersen:17,denbesten:86,Zeijlstra:diss}. %, see \REF{ex-nc-intro}.
There is also modal concord as in \REF{mod-conc}, where we find two
modal expressions, here a modal auxiliary and a modal adverb,
expressing the same modality \citep{Zeijlstra:07,Huitink:12}.  We
expect that there may potentially be other concord phenomena at the
syntax-semantics interface.

\vbox{
\ea
%\begin{xlist}
%\ex
%\ex
My eyes must certainly be deceiving me.
\citep[404]{Huitink:12}\label{mod-conc}\\
$=$ My eyes must be deceiving me.\\
$=$ Certainly, my eyes are deceiving me.
%\end{xlist}
\z 
}

% BC: Not sure anymore that relatives are a good case

Cases of redundancy also involve pronouns, as witnessed, inter alia,
by resumption.  In many languages, the extraction site in an unbounded
dependency, such as \bsp{wh}-fronting or relativisation can or must be
marked by a pronominal in situ. For instance in Hausa, questioning the
object of a preposition requires either pied-piping of the
preposition, or else presence of a pronoun in situ, as illustrated by
the example in (\ref{ex:Hau:Res}).

\begin{exe}
  \ex \label{ex:Hau:Res}
  \begin{xlist}
    \ex {\gll dà mèe kikà zoo?\\
      with what you.\textsc{f.sg} come\\
      \glt `With what did you come?' \hfill \citep[521]{jaggar01:_hausa}} 
    \ex {\gll mèe kikà zoo dà shii?\\
      what you.\textsc{f.sg} come with him/it\\
      \glt `What did you come with?' \hfill \citep[521]{jaggar01:_hausa}} 
  \end{xlist}
\end{exe}  

In the case of pied-piping in (\ref{ex:Hau:Res}a), we have a
one-to-one correspondence between participants and their
realisations. 
With resumption in (\ref{ex:Hau:Res}b), hwoever, one participant is actually
realised twice, namely by the fronted \bsp{wh} expression \textit{mèe}
`what' and
by the in situ resumptive pronoun \textit{shii} `him/it'. Unless one assumes
ambiguity between semantically potent ordinary pronouns and
semantically vacuous resumptives, one is confronted with the problem
that a single semantic role is simultaneously filled by two syntactic
complements. However, as pointed out by
\citet{mccloskey02:_resum_succes_cyclic_local_operat}, resumptive
pronouns are non-distinct in shape from the ordinary pronouns of the
language, casting doubts on an ambiguity approach. 


% wh-doubling
We also find cases of doubling of \bsp{wh}-words. 
In Afrikaans long-distance extraction, there can be 
a copy of the extracted \bsp{wh}-phrase at the beginning of  any intermediate clause.
This is shown in \REF{w-kopier}. 
The construction is not restricted to Afrikaans. 
\citet{Hoehle:19wh} discusses analogous data in German, and \citet{Bruening:06} in Passamaquoddy. 


\vbox{
\ea
\gll 
Waarvoor denk julle waarvoor werk ons?\\
wherefore think you wherefore work we\\
\glt \mytrans{What do you think we are working for?}
\citep[725]{Plessis:77}\label{w-kopier}
\z 
}


\begin{sloppypar}
  We would like to mention a final group of redundancy phenomena that
  does not involve functional elements: predicate fronting and cognate
  objects.  For many languages, we find a duplication of a fronted
  predicate, as in the Yiddish example in \REF{yiddish-front} from
  \citet{Kaellgran:Prince:89}.  In this case, a non-finite form of the
  predicate occurs in the fronted position, and the same verb, though
  in a potentially different inflected form, occurs in the rest of the
  clause.
\end{sloppypar}

\ea
\gll leyenen leyent er dos bukh yetst.\\
read\textsc{.inf} reads he the book now\\
\glt \mytrans{As for reading, he's reading the book now.}
\citep[48]{Kaellgran:Prince:89}\label{yiddish-front}
\z 

\begin{sloppypar}
  This phenomenon has been documented at least for Hebrew, Hungarian,
  Brazilian Portuguese, Russian, Spanish, Yiddish \citep{Vicente:09}.
\end{sloppypar}
%\citet{Urogdi:06}

The cognate object construction is a further phenomenon showing
redundancy.  In the prototypical case of this construction, a usually
intransitive verb combines with an NP complement that can be
considered a nominalisation of the verb, see \REF{ex-coc}.  As the
example shows, the NP complement seems to be redundant.  This is,
again, a cross-linguistically very common construction
\citep{Jones:88,Massam:90,Mittwoch:98}.

\ea \label{ex-coc}
Harry lived an uneventful life. 
\\
$=$ Harry lived uneventfully.
\citep[89]{Jones:88}
\z 

% * Polyadic quantification (MS)
We can turn to a different type of one-to-many relations.
In the following cases, several quantificational elements occur in a
sentence but need to be interpreted as a single unit, a \emph{polyadic
  quantifier}. This is illustrated in \REF{ex-different}, from \citet{Keenan:92},
%and \citet{Richter:16}, 
with a paraphrase of the relevant reading.
 \citet{Keenan:92} shows that certain uses of \bsp{different} cannot be accounted for with a combination of ``ordinary'', i.e.\@ monadic, quantifiers. 
This result presents an important challenge 
%to systems of the semantic combinatorics, i.e.\@ 
to systems of semantic combinatorics that assume compositionality.
%the idea of a compositional semantics.
%


\begin{exe}
  \ex
  \label{ex-different}
  Different people like different things.\\
    \mytrans{There are at least two people and for all distinct people
      $x$, $y$ the things that $x$ likes are not exactly the same as
      those that $y$ likes.}
\end{exe}


\begin{sloppypar}
  Various approaches have been proposed to solve this problem:
  \citet{Moltmann:95} and \citet{Beck:06} generate more general
  readings in a compositional way and assume context-sensitive
  mechanisms that will filter out undesired readings.
  \citet{Barker:07} proposes an unusual syntactic structure that will
  guide the interpretation.  \citet{Lahm:2016} uses data on
  \bsp{different} as additional motivation for the use of choice
  functions.  Finally, \citet{Richter:16} employs a non-standard
  mechanism of semantic combinatorics to arrive at an explicitly
  polyadic semantic representation.  If one accepts a polyadic
  analysis, the configuration is similar to the one we found in
  complex predicates: several expressions form an inseparable unit
  together.
\end{sloppypar}

% * Ellipsis
The last one-to-many relation that we would like to mention are elliptical phenomena. 
These include \emph{gapping}, see \REF{ex-gapping}, and \emph{argument cluster coordination}, as in \REF{ex-acc}, both examples are taken from \citet{Kubota:Levine:16-Gapping}.

\ea \label{ex-gap-acc}
\begin{xlist}
\ex
\label{ex-gapping}
Leslie bought a CD, and Robin a book.
%
\ex \label{ex-acc}
I told the same joke to Robin on Friday and to Leslie on Sunday. \citep{Kubota:Levine:16-Gapping}
\end{xlist}
\z 

Gapping is a one-to-many phenomenon in the sense that the verb is mentioned only in the first conjunct but present for interpretation in both conjuncts.
There are numerous approaches to these phenomena. They can, basically, be divided into three groups: (i) phonological deletion approaches 
%Ross 1969: glaube nicht, dass wir das finden! Ross, J.R., 1969. Guess who? In: Binnick, R.I., Davison, A., Green, G.M., Morgan, J.L. (Eds.), Proceedings of the Fifth annual meeting of the Chicago Linguistics Society, Chicago, IL, pp. 252--286.
\citep{Merchant:01,Fox:Lasnik:03}; (ii) approaches assuming a copy at the level of Logical Form \citep{Lobeck:95,Chung:al:95%,Chung:al:10
}; 
%(Lobeck, 1995; Chung et al., 1995, 2010; Chung, 2006, 2013)
(iii) direct interpretation approaches \citep{Ginzburg:Sag:00,Culicover:Jackendoff:05,Kubota:Levine:16-Gapping}. 
%

We hope to have shown in this section that we find one-to-many
phenomena of various types in all modules of grammar and at their
interfaces.  It is common in formal linguistics to try to reduce these
phenomena to one-to-one relations.  The papers in this volume take a
different approach, taking the one-to-many nature of the phenomena at
face value.

\section{Overview of the individual chapters}

The chapters in this volume are grouped together according to the
major linguistic sub-disciplines, starting with morphology, via the
morphology-syntax interface towards syntax and semantics.  

In the second chapter of the volume, \textbf{Beniamine} investigates
the system of inflectional classes across a number of language, using
a data-driven computational approach, which permits to assess the
complexity of morphological systems without any bias from the
analysing linguist.

Beniamine starts off with a comparison of different conceptualisations
of inflection classes, going from simple, flat partitions, as
characteristic of pedagogical grammars, via trees, as advocated in the
theoretical literature, to lattices, i.e.\  multiple
inheritance.  In the discussion of tree-based approaches, he already
notices deviations that would suggest a more general data
structure. 

The main theoretical question addressed in Beniamine's chapter is the
extent to which inflection class systems can be regarded as trees or
rather multiple inheritance hierarchies. Or, put in more linguistic
terms, to what extent inflectional class systems are characterised by
heteroclisis.

Beniamine's method takes as a starting point an ideally complete
lexicon of morphological word forms, paired with the morphosyntactic
features that are expressed. From these, he automatically extracts
morphophonological alternation patterns that relate a lexeme's  word form in
one cell to that in another. These patterns then represent a lexeme's
paradigm as the set of alternations. Full (or partial) identity of 
these alternations across lexemes provides the basis for an empirical
notion of inflection class. 

Using concept analysis, Beniamine automatically constructs more
general superclasses corresponding to the sharing of patterns across
lexemes. If a number of lexemes share all patterns, they form a
microclass, which corresponds to a tree. More abstract classes are
built from microclasses on the basis of partial identity.

Beniamine evaluates the complexity of the concept hierarchies of six
different languages (Arabic, English, French, Russian, Portuguese,
Chatino) using three metrics: (i) the number of concepts, (ii) the
depth of the hierarchy, and (iii) the number of immediately dominating
nodes for each concept, which is an indicator of multiple inheritance.

The results are highly interesting: in all six languages, the number
of concepts clearly surpasses the number of microclasses,
disconfirming the idea of a flat partitioning. The most spectacular
finding, though, is that all systems witness an elevated degree of
multiple inheritance, an average of almost two dominating nodes for
English, and higher for all other languages. Beniamine concludes that
heteroclisis permeates the system and should be considered the norm
rather than the exception. Thus, it seems that inflection class
systems observe a many-to-many organisation that can be captured by
multiple inheritance hierarchies, but neither partitions nor trees.



The contribution by \textbf{Crysmann} addresses a classical
challenge in inflectional morphology, namely an extreme case of
extended (or multiple) exponence in Batsbi (Tsova-Tush), called
exuberant exponence \citep{Harris09}. In this language, the same set
of class (=gender/number) markers can appear multiple times
within a word, as shown in example (\ref{ex:BatsbiMult}).

\begin{exe}
  \ex  \label{ex:BatsbiMult}\gll \textbf{y}-ox-\textbf{y}-$\emptyset$-o-\textbf{y}-anǒ\\
  \textsc{cm}-rip-\textsc{cm-tr-prs-cm-evid1}\\
  \glt `Evidently she ripped it.’ \hfill\citep[277]{Harris09}
  
\end{exe}

What distinguishes exuberant exponence as found in Batsbi from more
common cases of multiple exponence is not just a matter of quantity, or
the fact that multiple marking is alliterative. These are important
properties, yet the most central observation relates to its variable
nature: because only certain stems take the marker, and only certain
affixes (e.g. transitive/intransitive and evidential), we may find
anything between zero and four identical exponents.

The formal analysis Crysmann proposes is carried out in the framework
of Information-based Morphology (=IbM; \citealp{Crysmann:Bonami:2016}) and
exploits the fact that this theory incorporates $m:n$ relations at the
most basic level of organisation, namely realisational rules,
extracting partial generalisations over rules by means of inheritance
in typed feature structures. The analysis capitalises on the dependent
nature of exuberant exponence in Batsbi and shows how IbM permits to
improve over the holistic word-based baseline proposed in
\citet{Harris09}. There is an interesting twist as to how the
one-to-many relation between the morphosyntactic property of class
agreement and its zero to many exponents is captured in the formal
analysis: because both the number and the position of markers depend
on the presence of a particular stem or some other suffixal marker,
multiple exponence is indirect, and so is the locus of the one-to-many
relation: in essence, exponence rules for class markers compose with
those for the stems and markers they depend on, forming many-to-many
rules of exponence that introduce more than one marker corresponding
to more than one function. Technically, this is done by systematic
cross-classification of agreement marking rules for stems and
exponents they depend on. This cross-classification in turn
constitutes another instance of a one-to-many relation, namely at the
level of the formalism (cf. the semi-lattices in Beniamine's chapter).

Thus, the availability of one-to-many relationships at the level of
the underlying logic, as is the case with multiple inheritance
hierarchies, appears to provide a solid foundation to approach
one-to-many relations at the level of descriptions.

The chapter by \textbf{Bonami \& Webelhuth} crosses the boundary
between morphology and syntax by investigating periphrastic tenses in
Czech. Periphrastic realisation describes the situation where
syntactically independent words analytically fill cells in a paradigm
for which there is typically no synthetic realisation. Periphrasis
in itself already constitutes a one-to-many relationship, where more
than one lexeme is involved in the inflectional realisation of a
morphological word.

The particular phenomenon under investigation concerns the past and
conditional, both of which are realised analytically by a participial
form combined with the (clitic) copula in the present or past,
respectively. While the copula is always overtly realised in
predicative constructions, both in its present and past forms, and it
is equally present in all cells of the periphrastic conditional, third
person cells of the past paradigm are characterised by the significant
absence of the ancillary element, an instance of what the authors call
\textit{zero periphrasis}, in analogy with the well-known phenomenon
of zero exponence. 

Bonami \& Webelhuth argue that these particular non-periphrastic
cells in otherwise periphrastic paradigms need to be accounted for in
morphological terms, rather than in terms of a covert copula.
Extending their previous theory of periphrasis
\citep{Bonami14d,Bonami16b,Bonami13}, they propose that zero
periphrasis should be captured at the morphology-syntax interface,
treating third person past as exceptionally non-periphrastic
cells. This mirrors quite neatly the case of non-default
zero-exponence, as found in synthetic inflectional morphology. 

Complex predicates provide one of the classical challenges for the
view that the interface between syntax and the lexicon constitutes a
straightforward one-to-one correspondence. In their chapter,
\textbf{Faghiri \& Samvelian} investigate the syntactic separability
of complex predicates in Persian and explore to what extent complex
predicate status correlates with linearisation properties.  The
authors report the results of two acceptability judgement studies that test
word-order variation. In (\ref{ex-fs-1}), the complex predicate
\textit{v\={a}ks zadan} `to polish' (lit: polish hit) is used.
As can be seen, the nominal and the verbal part of the complex
predicate are adjacent in (\ref{ex-fs-1a}), but can be separated by a
prepositional phrase, see (\ref{ex-fs-1b}).

\ea \label{ex-fs-1}
\begin{xlist}
\ex 
\gll ali be kaf\v{s}-h\={a} v\={a}ks zad\\
Ali to shoe-\textsc{pl} polish hit.\textsc{pst.3sg}\\
\glt `Ali polished the shoes.'\label{ex-fs-1a}
\ex
\gll ali behtarin v\={a}ks=r\={a} be kaf\v{s}-h\={a}  zad\\
Ali best polish=\textsc{ra} to shoe-\textsc{pl} polish hit.\textsc{pst.3sg}\\
\glt `Ali polished the shoes with the best polish.'
\label{ex-fs-1b}
\end{xlist}
\z 

The paper investigates the conditions under which such a
separation is possible and contrasts this with the word order
preferences of syntactic combinations that are not complex
predicates. The studies show that complex predicates behave largely as
one would expect given their syntactically complex form, not given
their semantic or lexicographic unit-like nature. A certain preference
for non-separate occurrence is, however, attested.

In the second chapter on syntax, \textbf{Pozniak, Abeillé \& Hemforth}
explore the use of inverted vs.\ non-inverted subjects with object relatives in
French, as illustrated by the examples in (\ref{ex:FrRelInv}). They
start off by observing that inversion is standardly considered
optional and possibly dispreferred and note that current competence
and performance models alike make conflicting predictions regarding a
preference for or against subject inversion in this context.

\begin{exe}
  \ex \label{ex:FrRelInv}
  \begin{xlist}
    \ex \gll Le médecin [que l'avocat connait] aime courir.\\
    the physician that {the lawyer} knows likes run\\
    \glt `The physician [that the lawyer knows] likes running.'

    \ex
    \gll Le médecin [que connait l'avocat] aime courir.\\
    the physician that knows {the lawyer}  likes run\\
    \glt `The physician [that the lawyer knows] likes running.'
  \end{xlist}
\end{exe}


The main aim of their contribution is to assess not only the relative
acceptability of inversion with object relatives, but also what the
specific use conditions for each of the two variants are that favour
one realisation over the other. They report on three empirical studies
they have conducted to shed light on this issue: a corpus study, an
acceptability judgement task, and a self-paced reading experiment.


In the corpus study they annotated object relatives from the French
Treebank with properties pertaining to the subject, the verb, and the
relativised object, as well as global properties, such as length of
the subject or the relative clause. The data were analysed using
logistic regression. Among the significant factors favouring inversion
they found two subject-related properties, namely intentionality and
length. These were tested in two subsequent experiments: while the
acceptability judgement task confirmed the basic corpus findings regarding
the equal acceptability of inverted and non-inverted subjects in this
construction, the self-paced reading experiments revealed improved
performance with combined factors (length and intentionality), from
which the authors conclude that a proper understanding needs to
acknowledge both distance-oriented processing constraints and semantic
factors, which can be seen as an instance of one-to-many relations at the
level of performance.

\begin{sloppypar}
  The final two chapters of this volume explore one-to-many aspects of
  semantics.  \textbf{Sailer \& Richter} look at the syntax-semantics
  interface and \textbf{Barg\-mann, Gehr\-ke \& Rich\-ter} study the
  simultaneous availability of different levels of interpretation.
\end{sloppypar}

\textbf{Sailer \& Richter} combine two constellations that give rise
to one-to-many correspondences: negative concord (NC) and
coordination.  In NC languages, two negative indefinites may co-occur
in the same clause while a single negation is expressed
semantically. Thus, we observe a one-to-many correspondence in
the sense of a double marking of negation in syntax and a single
negation in the interpretation.  In coordination, we can find the
opposite situation: what appears to be a constituent negation in
syntax can, and sometimes must, be interpreted as a coordination of
two clauses, i.e., the part of the sentence outside the coordinated
constituent occurs only once, but is interpreted several times, once
for each conjunct.

%Slavic and Romance languages are NC languages (ignoring important differences among the various languages), Standard English, German, and Dutch are not. In such non-NC languages, a clause with two negative indefinites cannot be interpreted as a simple, single negation. 
\textbf{Sailer \& Richter} study cases in which two negative indefinite noun phrases are coordinated in a non-NC language, Standard German, as in example (\ref{ex-rs-1}). 

\ea \label{ex-rs-1} 
\gll Alex hat keine Milch und keinen Zucker verrührt.\\
Alex has no milk and no sugar stirred\\
\glt Bi-propositional reading: `Alex didn't stir milk and Alex didn't stir sugar.'\\
     Mono-propositional readings: `Alex didn't stir milk and sugar together.'
\z 

They show that there are, in principle, two readings of such
sentences: a bi-propositional reading and a mono-propositional
reading, i.e., the sentence can be logically characterised by a conjunction of two
negated sentences or by a single negated sentence that contains the
union of the two conjuncts in the scope of negation.  In the
mono-propositional reading, we find the first type of one-to-many
correspondence, in the bi-propositional reading, we find the second
type.

In the last chapter of this volume,
\textbf{Bargmann, Gehrke \& Richter} consider a case of a one-to-many
correspondence that relates a single syntactic form to various levels
of interpretation at the same time.  They discuss data with idioms
expressing the idea of dying in English and German in which the idiom
occurs with a modifier that seems to be interpreted literally rather
than idiomatically.  One of their examples is (\ref{ex-bgr-1}). Here,
the idiom \textit{kick the bucket} `die' is used, but the noun phrase
\textit{the bucket} contains a modifier, \textit{golden}, which is
incompatible with the idiomatic meaning of the expression.

\ea \label{ex-bgr-1}
Venezuela’s Friend of the Working Class, Hugo Chávez, kicked the golden
bucket with an estimated net worth of 2 billion dollars.
\z

The authors argue that the sentence receives two types of
interpretation simultaneously: an idiomatic interpretation
(\textit{Hugo Chávez died}) and a literal interpretation of part of
the idiom (\textit{Hugo Chávez had a golden bucket}).  To make the two
parts of interpretation fit together, the literal interpretation of
the idiom part gives rise to an inference Hugo Chávez was rich.  Taken
together, sentence (\ref{ex-bgr-1}) expresses the idea that Hugo
Chávez died and was very rich.  \textbf{Bargmann et al.}\@ provide a
detailed discussion of naturally occurring examples of this type of
intricate uses of idioms, in which an expression is used in its
idiomatic meaning and, at the same time, part of the idiom is
interpreted literally, like \textit{the bucket} in (\ref{ex-bgr-1}).


It is the central aim of this book to make a strong case for accepting
one-to-many correspondences as an essential property of the interfaces
of natural language grammar. The individual chapters provide detailed
studies of exemplary phenomena to see whether the analytic tools
developed for handling them in one module of grammar are transferable
to other modules, and to work on an integrated approach within a
constraint-based grammar framework.

\section*{Abbreviations}

Examples in this chapter follow the Leipzig glossing rules. We use the
following additional abbreviations, in order of appearance:
\textsc{ne} (French negative particle \textit{ne}), \textsc{cm} (class
marker), \textsc{evid1} (evidential 1), and \textsc{ra} (Persian
particle \textit{ra}).

\section*{Acknowledgements}

The research collected in this volume has been carried out as
part of the bilateral project \textit{One-to-many relations in
  morphology, syntax and semantics}, funded from 2017 to 2018 by a
grant to Berthold Crysmann from the \textit{Ministère de l'éducation
  supérieur, de la recherche et de l'innovation} as part of the
\textit{Parténariat Hubert Curien} of the \textit{Procope} programme
and by a grant from the \textit{Deutscher Akademischer
  Austauschdienst} (DAAD) to Manfred Sailer.

We are gratefully indebted to the 14 anonymous reviewers for their
expertise. A great many thanks also go to Sebastian Nordhoff and Felix
Kopecky for their
technical support in the production of this volume, as well as to the
proofreading volunteers from the Language Science Press
community. Finally, we would like to thank Janina Radó for
copy-editing the final manuscript.

{\sloppy\printbibliography[heading=subbibliography,notkeyword=this]}

\end{document}
